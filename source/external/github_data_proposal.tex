%
\documentclass[12pt,notitlepage]{article}
\usepackage{amssymb}
\usepackage{amsmath}
\usepackage{graphicx}
\usepackage{epstopdf}
\usepackage{pdflscape}
\usepackage{tabularx}
\usepackage{longtable}
\usepackage{array}
\usepackage{dsfont}
\usepackage{float}
\usepackage{booktabs}
\usepackage{tikz}
\usepackage{marvosym}
\usepackage{multirow}
\usepackage{pdflscape}
\usepackage[hyphens]{url}
\usepackage{setspace}
\usepackage{epigraph}
\usepackage{bm}
\usepackage{textcomp}
\usepackage{diagbox}
\usepackage{bbm}
\usepackage{verbatim}
\usepackage[framemethod=tikz]{mdframed}
\usepackage{subcaption}
\usepackage{caption}
\usepackage{lipsum}
\usepackage{mathtools}
\usepackage{scalerel}
\usepackage{stackengine}
\usepackage{amsthm}
\usepackage{epsfig}
\usepackage[
backend=biber,
style=authoryear-comp,
sorting=ynt
]{biblatex}
\usepackage[colorlinks,allcolors=blue]{hyperref}
\usepackage[shortlabels]{enumitem}
\usepackage{subfiles} % Best loaded last in the preamble


\setlength{\epigraphrule}{0pt}
\setlength\parindent{0pt}
\renewcommand{\baselinestretch}{1.25}

\setcounter{MaxMatrixCols}{10}

\newcolumntype{L}[1]{>{\raggedright\let\newline\\\arraybackslash\hspace{0pt}}m{#1}}
\newcolumntype{C}[1]{>{\centering\let\newline\\\arraybackslash\hspace{0pt}}m{#1}}



\newcommand{\I}{\mathbb{I}}
\newcommand{\E}{\mathbb{E}}
\newcommand{\Ll}{\mathrm{L}}
\newcommand{\R}{\mathbb{R}}
\renewcommand{\L}{\mathbb{L}}
\newcommand{\Var}{\mathrm{Var}}
\newcommand{\Cov}{\mathrm{Cov}}
\newcommand{\Corr}{\mathrm{Corr}}
\newcommand{\Prob}{\mathbb{P}}
\newcommand{\supp}{\mathrm{supp}}
\newcommand{\notimplies}{\mathrel{{\ooalign{\hidewidth$\not\phantom{=}$\hidewidth\cr$\implies$}}}}
\newcommand{\var}{\mathrm{var}}
\newcommand{\Bias}{\mathrm{Bias}}
\newcommand{\cov}{\mathrm{cov}}
\newcommand{\corr}{\mathrm{corr}}
\newcommand{\MSE}{\mathrm{MSE}}

\topmargin=-1.5cm \textheight=23cm \oddsidemargin=0.5cm
\evensidemargin=0.5cm \textwidth=15.5cm

\newtheorem{theorem1}{Special Theorem}

\newtheorem{ass}{Assumption}
\newtheorem{definit}{Definition}
\newtheorem{prop}{Proposition}
\newtheorem{thm}{Theorem}
\newtheorem{lem}{Lemma}
\newtheorem{conj}{Conjecture}
\newtheorem{cor}{Corollary}
\newtheorem{rem}{Remark}

\renewcommand{\thesubsection}{\arabic{section}.\arabic{subsection}}
\renewcommand{\thesubsubsection}{\arabic{section}.\arabic{subsection}.\arabic{subsubsection}}

\newcommand\dapprox{\stackrel{\mathclap{\tiny \mbox{d}}}{\approx}}
\newcommand\papprox{\stackrel{\mathclap{\tiny \mbox{p}}}{\approx}}
\newcommand\pconverge{\stackrel{\mathclap{\tiny \mbox{p}}}{\to}}
\newcommand\dconverge{\stackrel{\mathclap{\tiny \mbox{d}}}{\to}}

\addbibresource{source/paper/references.bib}


\newcommand\independent{\protect\mathpalette{\protect\independenT}{\perp}}
\def\independenT#1#2{\mathrel{\rlap{$#1#2$}\mkern2mu{#1#2}}}


\hypersetup{
	pdftitle={undergrad thesis},    % title
	pdfauthor={Chris Liao},     % author
	pdfnewwindow=true,      % links in new window
	colorlinks=true,       % false: boxed links; true: colored links
	linkcolor=blue,          % color of internal links
	citecolor=red,        % color of links to bibliography
	filecolor=black,      % color of file links
	urlcolor=blue           % color of external links
}

\allowdisplaybreaks

\begin{document}
	
	\title{GitHub Data Proposal: Organizational Hierarchy in Open Source Software Development}
	\author{\textsc{Chris Liao}\footnote{cliao@hbs.edu}}
	
	\maketitle

\section*{Project Summary}
The objective of this research project is to understand the relationship between the organizational hierarchy of the Open Source Software (OSS) project and the development of open source software. By combining empirical data on the behavior of OSS contributors and the project with economic theory on knowledge-based organizational hierarchies, I can answer two questions. 
\begin{enumerate}
    \item First, how does the OSS organization allocate responsibilities to different OSS contributors.
    \item Second, when the OSS development environment changes, how does project leadership respond and how does the behavior of OSS contributors evolve. 
\end{enumerate} Studying the OSS organizational hierarchy is important because OSS development relies on OSS contributors, whose behavior is affected and whose role in the organizational hierarchy is directly determined by OSS project leadership (\cite{lerner_simple_2002}, \cite{lerner_economics_2005}). Given the economic importance of OSS, which is used in at least 70\% of all software (\cite{perlow_summary_2022}), there's great value in studying the organizational hierarchy underlying OSS development.  While it is well documented that OSS organizations are organized into hierarchies based on contributor knowledge (\cite{crowston_hierarchy_2006}, \cite{crowston_core_2006}, \cite{lerner_simple_2002}), my research is the first that seeks to apply microdata on OSS development and an economic model of knowledge hierarchy to quantitatively study OSS development.


\section*{Analysis}
I'm interested in testing two hypotheses using data
\begin{itemize}
    \item How do OSS organizations and OSS contributors respond to the departure of highly ranked OSS contributors from the project?
    \item How do OSS organizations and OSS contributors respond to GitHub platform improvements (like issue \& pull request templates) that improve communication?
\end{itemize}
These are hypotheses that economic models of knowledge hierarchies (\cite{garicano_hierarchies_2000}) also study. Adapting existing models for OSS development can also illuminate the economic forces driving change in behavior and generalize the settings in my two hypotheses to broader classes of change. Moreover, using the knowledge hierarchies framework makes my results comparable to existing empirical studies of how traditional firms and their employees respond to shocks to their economic environment (\cite{garicano_hierarchies_2012}, \cite{bloom_distinct_2014}). \\


Next, I list the data that I require for this empirical analysis. I require five categories of data: contributor characteristics, OSS project characteristics, data on issues, data on pull requests, and data on pushes. Since I am analyzing contributor behavior over time, it is necessary to have granular, contributor-level data on how their behavior in a project evolves with their role in the project's hierarchy. In particular, the rank variable, which is a key component of my analysis, is not publicly available. Similarly, since I am studying the organizational hierarchy of a project, it is important to also understand the project itself and understand how that might affect the organizational hierarchy. Finally, data on issues, pull requests, and pushes provide measures of OSS development that I will need to objectively describe the behavior of OSS contributors and quantify project outcomes. Some of the variables that I am describing are not publicly available online, so I am willing and happy to work with anonymized data, and I welcome further discussion on data-specific details!  



\iffalse
\subsubsection{Predicted Empirical Effects - Hierarchical Structure}
I am interested in two metrics related to hierarchical structure
\begin{enumerate}
    \item Span: Ratio of higher to lower ranked contributors at each rank  
    \item Output: How many and what types of problems are being solved by the project
\end{enumerate}
\subsubsection{Predicted Empirical Effects - Contributor Behavior}
I am interested in three metrics related to contributor behavior: . 
\begin{enumerate}
    \item Individual Effort: How much time are contributors investing into the project?
    \item Individual Skill: How skilled is each contributor? 
    \item Individual Value Added: What is the difficulty and value of the problems they're solving?
\end{enumerate}
\fi

\section{Contributor Data} \label{contributor_data}
\begin{itemize}
    \item \textbf{Timespan}: 2011-2024
    \item \textbf{Frequency}: Daily (Monthly)
    \item \textbf{Unit of Observation}: Contributor-project level
    \item \textbf{Observation Set}: Popular Python projects (see attached list)
    \item \textbf{Individuals of Interest}: All contributors who interacted with an issue or pull request
\end{itemize}
\subsection*{Covariates}
Each entry is uniquely identified by \textbf{Anonymous GitHub User ID}, \textbf{Anonymous GitHub Project ID} and \textbf{Date}
\begin{itemize}
    \item \textbf{Anonymous GitHub User ID}
    \item \textbf{Anonymous GitHub Project ID}
    \item \textbf{Date} (Month-Year)
    \item \href{https://docs.GitHub.com/en/organizations/collaborating-with-groups-in-organizations/about-organizations}{GitHub Project Owner Type (whether it's an organization or not)}
    \item \href{https://docs.GitHub.com/en/get-started/learning-about-GitHub/access-permissions-on-GitHub}{Rank}
    \item \href{https://GitHub.com/liaochris?tab=achievements}{\# of GitHub Achievements completed}
    \item Country (when listed on their profile)
    \item \href{https://GitHub.com/liaochris?tab=followers}{\# of followers}
    \item \href{https://GitHub.com/liaochris?tab=following}{\# of following}
    \item \% of all merged PRs (in project's history) authored by them 
    \item \% of all lines of code (in project's history) authored by them 
    \item \# months on GitHub
    \item \# months where they have been involved as a contributor on this project
    \item For the top 10 projects that they have the most PRs in, 
    \begin{enumerate}
        \item Quantity of PRs (all time)
        \item \% of all LOC (in project's history) authored by them 
        \item \# of Stars
    \end{enumerate}
    \item Anonymous GitHub User ID of the contributor who promoted them/invited them
\end{itemize}
% the ones I really need are Rank, achievements, followers, following, \% of all commits and LOC authored

\section{Project Data} \label{project_data}
\begin{itemize}
    \item \textbf{Timespan}: 2011-2024
    \item \textbf{Frequency}: Monthly
    \item \textbf{Unit of Observation}: Project level
    \item \textbf{Observation Set}: All popular Python projects (see list)
\end{itemize}
\subsection*{Covariates}
Each entry is uniquely identified by \textbf{GitHub Project ID} and \textbf{Date}. 
This matches to the data described in Section~\ref{contributor_data} (\nameref{contributor_data}) through \textbf{Anonymous GitHub Project ID}
\begin{itemize}
    \item \textbf{Anonymous GitHub Project ID}
    \item \textbf{Date} (Month-Year)
    \item For each of the top 5 programming languages used in the project, the \% of the codebase that relies on that language
    \item \href{https://docs.github.com/en/repositories/managing-your-repositorys-settings-and-features/customizing-your-repository/classifying-your-repository-with-topics}{Project topics}
    \item Project license type
    \item \# of Forks (cumulative)
    \item \# of Stars (cumulative)
    \item \# of Watchers (cumulative)
    \item \href{https://GitHub.com/pandas-dev/pandas/network/dependencies}{\# Number of dependencies, cumulative}
    \item \href{https://GitHub.com/pandas-dev/pandas/network/dependents}{\# of Dependents, cumulative}
    \item \# of Releases (cumulative)
    \item \# of Tags (cumulative)
    \item Whether reviews are required on the PR
    \item Whether auto-merge is on
    \item Whether the project has a CONTRIBUTING.md or similar file
    \item Whether the substring ``contribut" shows up in the README
    
\end{itemize}
% Theo nes I really need are programming language, topics, forks, stars, watchers, dependencies, dependents

\section{Issue Data} \label{issue_data}
\begin{itemize}
    \item \textbf{Timespan}: 2011-2024
    \item \textbf{Unit of Observation}: Issue-Project level
    \item \textbf{Observation Set}: All popular Python projects (see list)
\end{itemize}
\subsection*{Covariates}
Each entry is uniquely identified by \textbf{Anonymous GitHub Project ID} and \textbf{Anonymous Issue ID}. This matches to the data described in Section~\ref{project_data} (\nameref{project_data}) through \textbf{GitHub Project ID}
\begin{itemize}
    \item \textbf{Anonymous GitHub Project ID}
    \item \textbf{Anonymous Issue ID} (anonymized version of Issue \# on GitHub)
    \item Issue Opener Anonymous GitHub User ID
    \item Issue Closer Anonymous GitHub User ID
    \item Date \& Time of issue opening (first time)
    \item Date \& Time of issue closing (first time)
    \item Whether issue was ever reopened 
    \item Issue current status (open or closed)
    \item A dictionary of all the times the issue tags for an issue were changed/assigned, with the date and time the issue tags were changed, the new cumulative tags assigned and the tag assignee Anonymous GitHub User ID.
    \item A dictionary of all the times the assignees for an issue were changed/assigned, with the date \& time the assignees were changed/assigned, the new cumulative group of assignees and the assigner. All assignees and assigners are identified by their Anonymous GitHub User ID
    \item Linked Anonymous Pull Request ID, if available
    \item Date \& Time Pull Request was linked
    \item Whether an issue development branch was created inside the project and if so, the anonymous branch ID
    \item Number of characters in issue text (in the latest edit)
    \item Latest edit date \& time of issue
    \item Each Anonymous GitHub User ID that was mentioned in the issue text
    \item Whether an Issue Template was used (if there are multiple issue templates, an anonymous id for the issue template based on the issue template filename)
\end{itemize}
% the ones I really need are tags and assignees

\section*{Issue Comment Data}
\begin{itemize}
    \item \textbf{Timespan}: 2011-2024
    \item \textbf{Unit of Observation}: Issue Comment-Project level
    \item \textbf{Observation Set}: All popular Python projects (see list)
\end{itemize}
\subsection*{Covariates}
Each entry is uniquely identified by \textbf{Anonymous GitHub Project ID}, \textbf{Anonymous Issue ID} and \textbf{Anonymous Issue ID Comment \#}. 
This matches to the data described in Section~\ref{issue_data} (\nameref{issue_data}) through \textbf{GitHub Project ID} and \textbf{Issue \#} 
\begin{itemize}
    \item \textbf{Anonymous GitHub Project ID}
    \item \textbf{Anonymous Issue ID}
    \item \textbf{Anonymous Issue ID Comment \#}
    \item Issue Commenter Anonymous GitHub User ID
    \item Date \& Time of Issue Comment
    \item Number of characters in issue comment text (latest edit)
    \item Latest edit date \& time of issue
    \item Each Anonymous GitHub User ID that was mentioned in the issue text
    \item Reactions on Issue Comment (as of data collection)
\end{itemize}

\section{Pull Request Data} \label{pull_request_data}
\begin{itemize}
    \item \textbf{Timespan}: 2011-2024
    \item \textbf{Unit of Observation}: Pull Request-Project level
    \item \textbf{Observation Set}: All popular Python projects (see list)
\end{itemize}
\subsection*{Covariates}
Each entry is uniquely identified by \textbf{Anonymous GitHub Project ID}, \textbf{Anonymous Pull Request ID} 
This matches to the data described in Section~\ref{project_data} (\nameref{project_data}) through \textbf{Anonymous GitHub Project ID} 
\begin{itemize}
    \item \textbf{Anonymous GitHub Project ID}
    \item \textbf{Anonymous Pull Request ID}
    \item Anonymous Issue ID
    \item Pull Request Opener Anonymous GitHub User ID
    \item Pull Request Merger/Closer Anonymous GitHub User ID
    \item Date \& Time of Pull Request opening (first time)
    \item Date \& Time of Pull Request closing or being merged (first time)
    \item Date \& Time the issue was linked
    \item Pull Request current status: Open, Closed or Merged
    \item A dictionary of all the times the tags for a pull request were changed/assigned, the new cumulative tags assigned and the tag assignee Anonymous GitHub User ID.
    \item A dictionary of all the times the assignees for a pull request were changed/assigned, with the date \& time the assignees were changed/assigned, the new cumulative group of assignees and the assigner. All assignees and assigners are identified by their Anonymous GitHub User ID
    \item A dictionary of all the times the reviewers for a pull request were changed/assigned, with the date \& time the reviewers were changed/assigned, the new cumulative group of reviewers and the assigner. All reviewers and assigners are identified by their Anonymous GitHub User ID
    \item Whether the Pull Request took place on a branch in the project and if so, the anonymous branch ID
    \item How many GitHub Actions checks were run and how many passed (latest run)
    \item Number of characters in the Pull Request Text (in the latest edit)
    \item Latest edit date \& time of Pull Request Text
    \item Each Anonymous GitHub User ID that was mentioned in the issue text
    \item The \# of other Pull Requests that were mentioned in the issue text
    \item The \# of other issues that were mentioned in the Pull Request text
    \item Number of commits prior to Pull Request being opened
    \item Number of commits after Pull Request was opened
    \item \# LOC added \& removed in commits prior to Pull Request being opened
    \item \# LOC added \& removed in commits after Pull Request was opened
    \item \# files added \& removed in commits prior to Pull Request being opened
    \item \# files added \& removed in commits after Pull Request was opened
    \item Date and Time of First Commit
    \item Date and Time of Last Commit (prior to Pull Request being opened)
    \item Date and Time of Last Commit (prior to merge or closing)
    \item Total number of PR review comments
    \item Total number of PR comments
    \item For each committer (identified by Anonymous GitHub User ID when possible), the number of commits, the \# of LOC added \& the \# of LOC removed
    \item For each reviewer (referenced by Anonymous GitHub User ID), whether they approved and how many times they requested changes
    \item Pull Request Target Branch: If Main/master, list that; otherwise, refer to the anonymous branch ID
\end{itemize}

\section*{Pull Request Comment Data}
\begin{itemize}
    \item \textbf{Timespan}: 2011-2024
    \item \textbf{Unit of Observation}: Pull Request Comment-Project level
    \item \textbf{Observation Set}: All popular Python projects (see list)
\end{itemize}
\subsection*{Covariates}
Each entry is uniquely identified by \textbf{Anonymous GitHub Project ID}, \textbf{Anonymous Pull Request ID} and \textbf{Anonymous Pull Request Comment ID} 
This matches to the data described in Section~\ref{pull_request_data} (\nameref{pull_request_data}) through \textbf{Anonymous GitHub Project ID} and  \textbf{Anonymous Pull Request ID}. In this dataset, only pull request comments are described (not pull request review comments). 
\begin{itemize}
    \item \textbf{Anonymous GitHub Project ID}
    \item \textbf{Anonymous Pull Request ID}
    \item \textbf{Anonymous Pull Request ID Comment \# }
    \item Pull Request Commenter Anonymous GitHub User ID
    \item Date \& Time of Pull Request Comment
    \item Number of characters in pull request comment text (in the latest edit)
    \item Latest edit date \& time of pull request comment
    \item Each Anonymous GitHub User ID that was mentioned in the pull request comment text
    \item Reactions on Pull Request Comment Text (as of data collection)
\end{itemize}

\section*{Pull Request Review Data}
\begin{itemize}
    \item \textbf{Timespan}: 2011-2024
    \item \textbf{Unit of Observation}: Pull Request Review-Project level
    \item \textbf{Observation Set}: All popular Python projects (see list)
\end{itemize}
\subsection*{Covariates}
Each entry is uniquely identified by \textbf{Anonymous GitHub Project ID}, \textbf{Anonymous Pull Request ID} and \textbf{Anonymous Pull Request Review \#} 
This matches to the data described in Section~\ref{pull_request_data} (\nameref{pull_request_data}) through \textbf{Anonymous GitHub Project ID} and \textbf{Anonymous Pull Request ID}. In this dataset, I describe aggregated pull request reviews
\begin{itemize}
    \item Anonymous GitHub Project ID
    \item Anonymous Pull Request ID
    \item Anonymous Pull Request Review \# 
    \item Pull Request Reviewer
    \item Date \& Time of Pull Request Review Completion
    \item Number of characters in pull request review
    \item Number of comments in pull request review
    \item \# of pull request comments in pull request review
    \item Pull Request Review Result
    \item Each Anonymous GitHub User ID that was mentioned in the pull request review text 
\end{itemize}

\section{Push Data}
\begin{itemize}
    \item \textbf{Timespan}: 2011-2024
    \item \textbf{Frequency}: Monthly
    \item \textbf{Unit of Observation}: Branch-Project level
    \item \textbf{Observation Set}: All popular Python projects (see list)
\end{itemize}
\subsection*{Covariates}
Each entry is uniquely identified by \textbf{Anonymous GitHub Project ID}, \textbf{Anonymous Branch ID} and \textbf{Date}
This matches to the data described in Section~\ref{project_data} (\nameref{project_data}) through \textbf{Anonymous GitHub Project ID} and \textbf{Date}. In this dataset, I describe aggregated pull request reviews
\begin{itemize}
    \item \textbf{Anonymous GitHub Project ID}
    \item \textbf{Master, Main, or Anonymous Branch ID}
    \item \textbf{Date (Month-Year)}
    \item For each \textbf{Anonymous GitHub User ID} who committed to the branch in the past month, 
    \begin{itemize}
        \item Number of commits
        \item Number of LOC added/deleted (cumulative)
        \item Number of files added, deleted, changed \& removed
    \end{itemize}
    \item Whether the branch is protected
\end{itemize}



\end{document}