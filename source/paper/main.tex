%
\documentclass[12pt,notitlepage]{article}
\usepackage{amssymb}
\usepackage{amsmath}
\usepackage{graphicx}
\usepackage{epstopdf}
\usepackage[pdftex,dvipsnames]{xcolor}
\setlength{\marginparwidth}{2cm}
\usepackage[colorinlistoftodos,prependcaption,textsize=small]{todonotes}
\usepackage{ragged2e}
\usepackage{xargs}
\usepackage{csquotes}
\usepackage{tabularx}
\usepackage{longtable}
\usepackage{arydshln}
\usepackage{array}
\usepackage{dsfont}
\usepackage{booktabs}
\usepackage{xparse}
\usepackage{tikz}
\usepackage{marvosym}
\usepackage{multirow}
\usepackage{pdflscape}
\usepackage[hyphens]{url}
\usepackage{setspace}
\usepackage{epigraph}
\usepackage{bm}
\usepackage{textcomp}
\usepackage{diagbox}
\usepackage{bbm}
\usepackage{verbatim}
\usepackage[framemethod=tikz]{mdframed}
\usepackage{subcaption}
\captionsetup[sub]{subrefformat=parens}
\DeclareCaptionLabelFormat{subpanel}{Panel~(#2):}
\captionsetup[sub]{labelformat=subpanel, labelsep=space}
\usepackage[authoryear]{natbib}

\usepackage{caption}
\usepackage{lipsum}
\usepackage{mathtools}
\usepackage{scalerel}
\usepackage{stackengine}
\usepackage{amsthm}
\usepackage{epsfig}

\usepackage[colorlinks,allcolors=blue]{hyperref}
\usepackage[shortlabels]{enumitem}
\setlength{\epigraphrule}{0pt}
\renewcommand{\baselinestretch}{1.25}

\setcounter{MaxMatrixCols}{10}

\newcolumntype{L}[1]{>{\raggedright\let\newline\\arraybackslash\hspace{0pt}}m{#1}}
\newcolumntype{C}[1]{>{\centering\let\newline\\arraybackslash\hspace{0pt}}m{#1}}
\newcolumntype{R}[1]{>{\raggedleft\let\newline\\arraybackslash\hspace{0pt}}m{#1}}

\usepackage{subfiles} % Best loaded last in the preamble


\newcommand{\I}{\mathbb{I}}
\newcommand{\E}{\mathbb{E}}
\newcommand{\Ll}{\mathrm{L}}
\newcommand{\R}{\mathbb{R}}
\renewcommand{\L}{\mathbb{L}}
\newcommand{\Var}{\mathrm{Var}}
\newcommand{\Cov}{\mathrm{Cov}}
\newcommand{\Corr}{\mathrm{Corr}}
\newcommand{\Prob}{\mathbb{P}}
\newcommand{\supp}{\mathrm{supp}}
\newcommand{\notimplies}{\mathrel{{\ooalign{\hidewidth$\not\phantom{=}$\hidewidth\cr$\implies$}}}}
\newcommand{\var}{\mathrm{var}}
\newcommand{\Bias}{\mathrm{Bias}}
\newcommand{\cov}{\mathrm{cov}}
\newcommand{\corr}{\mathrm{corr}}
\newcommand{\MSE}{\mathrm{MSE}}
\let\OldTodo\todo

\definecolor{brightpink}{rgb}{1.0, 0.0, 0.5}

\RenewDocumentCommand{\todo}{O{} m}{\OldTodo[#1]{\textbf{TODO}: #2}}
\newcommandx{\thiswillnotshow}[2][1=]{\OldTodo[disable,#1]{#2}}
\newcommandx{\askjesse}[2][1=]{\OldTodo[linecolor=Plum,backgroundcolor=Plum!25,bordercolor=Plum,#1]{\textbf{{Ask Jesse:}} #2}}
\newcommandx{\telljesse}[2][1=]{\OldTodo[linecolor=brightpink,backgroundcolor=brightpink!25,bordercolor=brightpink,#1]{\textbf{{Tell Jesse:}} #2}}
\newcommandx{\longterm}[2][1=]{\OldTodo[linecolor=Blue,backgroundcolor=Blue!25,bordercolor=Blue,#1]{\textbf{{Long-term:}} #2}}
\newcommandx{\donow}[2][1=]{\OldTodo[linecolor=Green,backgroundcolor=Green!25,bordercolor=Green,#1]{\textbf{{Do Now:}} #2}}


\topmargin=-1.5cm \textheight=23cm \oddsidemargin=0.5cm
\evensidemargin=0.5cm \textwidth=15.5cm

\newtheorem{theorem1}{Special Theorem}

\newtheorem{ass}{Assumption}
\newtheorem{definit}{Definition}
\newtheorem{prop}{Proposition}
\newtheorem{thm}{Theorem}
\newtheorem{lem}{Lemma}
\newtheorem{conj}{Conjecture}
\newtheorem{cor}{Corollary}
\newtheorem{rem}{Remark}

\renewcommand{\thesubsection}{\arabic{section}.\arabic{subsection}}
\renewcommand{\thesubsubsection}{\arabic{section}.\arabic{subsection}.\arabic{subsubsection}}

\newcommand\dapprox{\stackrel{\mathclap{\tiny \mbox{d}}}{\approx}}
\newcommand\papprox{\stackrel{\mathclap{\tiny \mbox{p}}}{\approx}}
\newcommand\pconverge{\stackrel{\mathclap{\tiny \mbox{p}}}{\to}}
\newcommand\dconverge{\stackrel{\mathclap{\tiny \mbox{d}}}{\to}}


\newcommand\independent{\protect\mathpalette{\protect\independenT}{\perp}}
\def\independenT#1#2{\mathrel{\rlap{$#1#2$}\mkern2mu{#1#2}}}

\onehalfspacing
\newtheorem{theorem}{Theorem}
\newtheorem{assumption}{Assumption}
\newtheorem{corollary}[theorem]{Corollary}
\newtheorem{proposition}{Proposition}

\newtheorem{hyp}{Hypothesis}
\newtheorem{subhyp}{Hypothesis}[hyp]
\renewcommand{\thesubhyp}{\thehyp\alph{subhyp}}

\newcommand{\red}[1]{{\color{red} #1}}
\newcommand{\blue}[1]{{\color{blue} #1}}


% required by modelsummary
\usepackage{tabularray}
\usepackage{float}
\usepackage{graphicx}
\usepackage{codehigh}
\usepackage[normalem]{ulem}
\UseTblrLibrary{booktabs}
\UseTblrLibrary{siunitx}
\newcommand{\tinytableTabularrayUnderline}[1]{\underline{#1}}
\newcommand{\tinytableTabularrayStrikeout}[1]{\sout{#1}}
\NewTableCommand{\tinytableDefineColor}[3]{\definecolor{#1}{#2}{#3}}


\begin{document}

\begin{titlepage}
\title{Organizational Resilience: Evidence from Open Source Software}
\author{Christopher Liao\thanks{I am indebted to my advisors Jesse Shapiro and Ali Hortaçsu for their support and guidance. I also thank Matthew Lee Chen, Krishna Dasari,  Jeff Gortmaker, Ruru Hoong, Noah Sobel-Lewin, Anjali Pullabhotla, Ruby Zhang, Emanuel Scherz and members of the Harvard Economics Predoc Workshop and JMSLab for helpful comments and suggestions. E-mail: cliao@hbs.edu}}
\date{\today}
\maketitle

\begin{abstract}
\noindent 
A large body of work in academia and industry has produced insights on what organizational practices induce organizational resilience. 
I systematically examine, in a large scale empirical study of open source software organizations, whether and which organizational practices affect resilience. 
I define resilience as how software development and release activity change after a key member permanently leaves the organization.
When the data are divided into balanced halves based on the adoption of a single organizational practice, these practices explain little variation in resilience.
However, splitting rules derived from flexible machine learning models that capture interactions among pre-departure practices can divide organizations into balanced halves that differ markedly in resilience.
Organizations whose members share expertise and maintain a positive atmosphere tend to be more resilient; surprisingly, those that are unwelcoming to newcomers and have a negative atmosphere are also more resilient.
\vspace{0in}\\
\end{abstract}

\setcounter{page}{0}
\thispagestyle{empty}
\end{titlepage}
\pagebreak \newpage

\section{Introduction} \label{sec:intro}
% \textbf{Paragraphs 1-2: Motivation. After reading these paragraphs a reader in any field of economics should believe that if you answer your research question your paper will make an important contribution.}
All organizations face challenges. 
Some crumble.
Others are able to weather the storm and bounce back. 
In doing so, they demonstrate their resilience. 
Building a resilient organization that successfully navigates disruptions is a topic of great interest. 
Witness, for instance, the myriad of organizational resilience insights produced by consulting companies such as McKinsey (\citealt{
maor_raising_2022}, \citealt{maor_foster_2023}, \citealt{kristensen_building_2025}), academic papers in fields ranging from business research (\citealt{duchek_organizational_2020}, \citealt{hillmann_organizational_2021}) to health policy (\citealt{barasa_what_2018}) or teaching cases in industries from journalism (\citealt{dutton_heart_2010}) to cloud technology services (\citealt{tang_business_2025}) and healthcare (\citealt{atkinson_organizational_2023}). 
But when brought to the data, do these insights actually contribute to improved organizational resilience? 
And if so, what are the organizational practices that enable an organization ``to maintain or restore an acceptable level of functioning despite perturbations or failures'' (\citealt{robert_organizational_2009})?

A wide variety of case studies and surveys have found that certain organizational practices bolster organizational resilience. 
Examples include investing in talent (\citealt{maor_raising_2022}), promoting positive communication (\citealt{luthans_psychological_2006}), establishing shared knowledge via resource or knowledge redundancy (\citealt{sheffi_supply_2005}, \citealt{rashid_systematic_2019}) and instituting  routines (\citealt{suarez_building_2020}). 
However, to date, there exists little empirical work on what organizational practices contribute to organizational resilience, despite extensive research and strong business interest in the topic.  
A systematic empirical analysis is valuable for determining whether commonly held beliefs about organizational resilience are generalizable at scale.

% \textbf{Paragraphs 3-4: Challenges. These paragraphs explain why your research question has not already been answered, i.e., what are the central challenges a researcher must tackle to answer this question.}

Conducting a systematic empirical analysis of the relationship between organizational practices and organizational resilience requires addressing two challenges: obtaining data on organizational practices and measuring organizational resilience.

The first challenge arises because detailed, systematic measures of organizational practices are rarely available. 
Even studies that leverage rich administrative data on individuals and firms provide limited insight into the inner workings of organizations (\citealt{jaravel_team-specific_2018}, \citealt{jager_how_2022}). 
When data on management practices is available, it comes in the form of organization-level surveys  (\citealt{bloom_management_2012}). 
This means researchers lack detailed information on the members of the organization, such as what they work on or how they work, that would enable researchers to analyze the impact of organizational practices. 
% Such detail is essential, as the actions of individual members both embody organizational practices and reveal how those practices shape an organization’s operations.

The second challenge is identifying a general measure of organizational resilience that meaningfully captures how organizations respond to disruptions.
Measuring organizational resilience requires exposing organizations to a disruption. 
Moreover, an organization's resilience should be assessed using outcomes that are both comparable across and matter to organizations. 
Ideally, organizations are exposed to the same type of disruption, to ensure that differences in organizational resilience will not confound differences in the effect of organizational practices with differences in the nature of the disruption. 
% \textbf{Paragraph 5: This Paper. This paragraph states in a nutshell what the paper accomplishes and how. }

In this paper, I study the relationship between organizational practices and organizational resilience using evidence from open source software (OSS) organizations, where detailed member-level data on activities is available. 
I leverage the abrupt departure of key members from the organization as the disruption that enables the measurement of organizational resilience. 
I measure organizational resilience by evaluating the causal effect of key departures on two categories of organizational outcomes: software development and software release activities.
I investigate how differences in the adoption of three categories of organizational practices affect organizational resilience: practices that build knowledge redundancy, invest in organizational member talent, and establish problem-solving routines. 
These categories are selected based on their relevance in the literature,

% \longterm[inline]{Paragraphs 6-7: Model. Summarize the key formal assumptions you will maintain in your analysis.

% \textbf{Paragraphs 8-9: Data. Explain where you obtain your data and how you measure the concepts that are central to your study}

OSS is a category of software that is freely available for use and modification. 
OSS is pervasive; a recent report found that 97\% of all software (including commercial types) included OSS code (\citealt{fred_bals_six_2025}). 
Prominent examples of OSS include the operating system Linux, the operating system of choice for 96.4\% of the top one million web servers (\citealt{w3cook_os_2015})\footnote{Ubuntu, CentOS, Debian, Fedora, SUSE and Redhat are all Linux-based} and the machine learning framework PyTorch, which is used by 63\% of all organizations training machine learning models (\citealt{lawson_shaping_2024}). 
Data from the GitHub Archive\footnote{GitHub is the world's largest host of OSS development repositories. The GitHub Archive is a third-party project to ``to record the public GitHub timeline, archive it, and make it easily accessible for further analysis'' (\citealt{GitHub_archive_GitHub_2025}).} provide activity-level data on the ongoings of OSS organizations.
I use this data to identify key members, measure software development activity and characterize organizational practices. 


% \textbf{Paragraphs 10-11: Methods. Explain how you take your model to the data and how you overcome the challenges you raised in paragraphs 3-4.}
Defining organizational resilience as the causal effect of key departures on organizational software development outcomes allows me to examine how OSS organizations respond to a common disruption---abruptly losing a key member---and evaluates them along a shared goal: developing software. 
Measuring organizational resilience is challenging because leaving the organization is, in part, a choice made by members. 
This means that the choice to depart may be driven by unobservable organizational trends that also affect outcomes. 
My aim is to identify a set of key departures that are plausibly exogenous to unobservable organizational trends.
Evidence from the literature suggests that abrupt key departures, which occur when a key member suddenly and permanently ceases all involvement with the organization, tend to be motivated by shocks to a member's time budget, such as a major life events, as opposed to unobservable organizational trends such as dissatisfaction with organizational leadership (\citealt{miller_why_2019}). 
Existing research from the literature that does not explicitly filter on abrupt key departures do find that departure tends to be motivated by organizational characteristics, such as dissatisfaction with the organization's management or disengagement from the organization's social community (\citealt{hannon_retaining_2008}, \citealt{constantinou_empirical_2017}, \citealt{miller_why_2019}), emphasizing the importance of the departure's abrupt nature. 

To characterize knowledge redundancy, I measure the extent of collaboration within organizations and the breadth of the tasks taken on by members.
To measure organizational investment in talent, I evaluate the characteristics of feedback provided to members, the cohesiveness of the organization's social structure, and the frequency of internal advancement. 
To measure organizational routines, I examine the extent to which organizations codify processes for problem solving. 
There are 37 different practices in total across the three categories. 
% \textbf{Paragraphs 12-13: Findings. Describe the key findings. Make sure they connect clearly to the motivation in paragraphs 1-2.}
% \todo[inline]{It would be nice to place a number on "impacts" caused by departure (what's the headline number?)}

The analysis proceeds in four steps. 
First, I estimate organizational resilience using an event study design.
I find that in line with results from the literature, key departures have significant negative impacts on all software development outcomes. 
Next, I provide descriptive evidence that organizational resilience does vary when I subset organizations by their pre-departure organizational practices, although the evidence is not strong. 

In the third step, I use the \textit{generalized random forest event study} to develop the \textit{optimal split rule} that classifies organizations as more or less resilient based on their pre-departure organizational practices. 
The generalized random forest event study, which I develop in a separate note in \href{https://drive.google.com/file/d/1brRivFrurkeuNHZHLgT_iuaZx_kuhXAz/view?usp=share_link}{\textbf{Liao (Forthcoming)}}, shows how the tools developed in \cite{athey_generalized_2019} can be used to estimate dynamic heterogeneous treatment effects in an event study with staggered treatment (\citealt{sun_estimating_2021}). 
I find that the optimal split rule performs exceedingly well; on average, less resilient organizations experience declines between 0.68 and 2.02 standard deviations larger than more resilient organizations across all organizational outcomes. 
In the final step, I transform the optimal split rule into an interpretable lower-dimensional rule that incorporates between 1 and 3 split layers using a policy learning algorithm (\citealt{athey_policy_2021}).
I call this the \textit{policy tree split rule}. 
Even under this highly regularized algorithm, I find that the policy tree split rule can still separate organizations into two subsets whose difference in resilience is statistically meaningful. 
Less resilient organizations still experience declines between 0.22 and 0.96 standard deviations larger than more resilient organizations across all organizational outcomes. 
One (perhaps not so surprising) finding is from the more interpretable rule is that more resilient organizations tend to have members with higher levels of shared expertise and conversations that are more positive.
One (more surprising) finding is that organizations that are more negative and unwelcoming to newcomers are also more resilient. 


% \textbf{Paragraphs 14-15: Literature. Lay out the two main ways your paper contributes to the literature. Each paragraph should center around one contribution and should explain precisely how your paper differs from the most closely related recent work.}

A large body of conceptual and empirical research on organizational resilience exists in the management literature (see review papers by \citealt{annarelli_strategic_2016} and  \citealt{hillmann_organizational_2021}), although most empirical work relies on case study or survey methods (see Table 2 in \citealt{annarelli_strategic_2016}). 
Two central challenges to empirical work are defining organizational resilience and identifying appropriate ways to measure it. 
I define organizational resilience as the capacity of organizations to respond to disruptions or shocks.\footnote{Work from the literature that has also used this definition include \cite{horne_iii_coming_1997}, \cite{kantur_organizational_2012}, \cite{boin_resilient_2013}, and \cite{sahebjamnia_integrated_2015}. 
Additional definitions widely used by the literature are listed in Table 2 of \cite{hillmann_organizational_2021}.} 
I measure organizational resilience by analyzing organizational outcomes observed \textit{ex post} the disruptive event.
Other work that adopt similar definitions and measures of organizational resilience include \cite{ortiz-de-mandojana_long-term_2016} and \cite{desjardine_bouncing_2019}, who examine the relationship between an organizations’ social and environmental practices (SEPs) and post-financial crisis performance and \cite{gittell_relationships_2006}, who examine the relationship between an organization in the airline industry's financial flexibility and performance post 9/11.
This paper differs in two key respects from prior work. 
First, I focus on a different category of organizational characteristics: organizational practices describing problem-solving, as opposed to financial characteristics or secondary goals like SEPs.
Second, I study organization-specific departures, as opposed to aggregate shocks that affect all organizations within a market at once. 
Because departures are more common events for organizations than large aggregate shocks, my findings will be more generalizable. 
This is especially true for OSS organizations, which depend on volunteers and thus face frequent turnover issues.
Moreover, aggregate shocks will affect all organizations operating in the same market, so quantifying how much an organization's resilience is attributable to its individual characteristics versus declines in its market competitiveness is difficult. 

This paper lies at the intersection of two bodies of work in economics.
The first body is empirical work studying organizational structure and practices (\citealt{bloom_measuring_2007}, \citealt{bloom_management_2012}). 
My work differs because I am specifically interested in understanding the contribution of organizational practices towards organizational resilience.
I do so by leveraging abrupt departures, which are plausibly exogenous departures; detail on the dynamics of member turnover is typically unobservable using organizational-level data.
This relates to a second body of empirical work that leverages plausibly exogenous departures to study various outcomes of interest, such as labor demand (\citealt{jager_how_2022}) and collaborations among inventors (\citealt{agrawal_how_2008}, \citealt{jaravel_team-specific_2018}, \citealt{azoulay_does_2019}), academics (\citealt{waldinger_quality_2010}, \citealt{oettl_reconceptualizing_2012}), scientists (\citealt{khanna_aftermath_2021}) and doctors (\citealt{chen_team-specific_2021}). 
However, even papers that leverage large-scale administrative datasets (\citealt{jaravel_team-specific_2018}, \citealt{jager_how_2022}) cannot describe in detail what practices different organizations or teams adopt and how those practices affect organizational resilience
My paper fills this gap by leveraging the rich data from open source software to understand both the extent and the details of how organizational practices contribute to resilience. 


What makes OSS organizations resilient post-departure is also an active research topic in the information systems and software engineering literature. 
\cite{rashid_systematic_2019} provides a detailed review of that topic. 
There has been extensive conceptual, survey and case study work on the prevalence and ways OSS organizations can mitigate either departures or their impact (\citealt{von_krogh_community_2003}, \citealt{robles_evolution_2005}, \citealt{hannon_retaining_2008}, \citealt{xu_volunteers_2010}, \citealt{yu_empirical_2012}, \citealt{rashid_exploring_2017}, \citealt{miller_why_2019}).
One outcome prior empirical work has used to quantify the impact of departures is ``potential damage to the codebase'' (\citealt{izquierdo-cortazar_using_2009}, \citealt{rigby_quantifying_2016}, \citealt{nassif_revisiting_2017}). 
\citealt{rigby_quantifying_2016}, which is most closely related to this paper, finds that the presence of “successors” who worked on tasks similar to the departed member can mitigate the repercussions of departures. 
However, this paper differs in several ways.
First, I focus on a variety of organizational practices, as opposed to just one. 
Second, whereas \cite{rigby_quantifying_2016} analyzes all departures from just 5 organizations, my analysis includes 699 organizations.
Finally, while \cite{rigby_quantifying_2016} considers hypothetical impacts based on potential codebase damage and possible mitigation through a successor, I quantify, using realized outcomes of organizational output, the actual impact of organizational practices. 

\section{Background and Data on Open Source Software} \label{sec:data}
\subsection{Background} \label{sec:data_background}
I analyze an important subset of the current OSS ecosystem: Python libraries. 
Python is the world's most popular programming language (\citealt{paul_jansen_tiobe_2025}). 
Python libraries--collections of reusable toolkits providing prewritten code for specific tasks--are a major reason for Python’s popularity, as they provide extensive functionality across diverse domains.
The development of most Python libraries occurs on GitHub, a cloud-based platform for software development and version control. 

The unit of observation in my analysis is the organization.
I define an organization as the GitHub project that handles development for one or more Python libraries. 
On GitHub, to initiate discussion about a topic, such as a software bug or feature request, a member of the organization opens an \textit{issue} thread. 
Members participate in the issue thread by posting \textit{issue comments}. 
To propose a change to the codebase, such as a software bug fix, members will open a \textit{pull request} thread. 
Members ask general questions about a pull request, such as the overarching goal, by posting \textit{pull request comments}.
Members ask questions about specific code changes, such as whether a specific line of code can be simplified, by posting \textit{pull request review comments} and decide what action should be taken by leaving \textit{pull request reviews}. 
If the organization's higher-ranking members believe the proposed code changes have value, typically, following some back and forth discussion, the pull request will be merged into the codebase. 
Changes made to the codebase are not immediately available to software users; instead, every so often, the organization releases an updated version of the Python library incorporating the new code changes. 
Anyone who participates in any of the aforementioned activities is classified as a member of the organization. 


The organizational outcomes of interest are software development and software update activity. 
Within the first category of outcomes, pull request activity, I am interested in the number of pull requests opened and merged. 
I will henceforth refer to this category as pull request-related activity. 
Within the second category of outcomes, software update activity, I am interested in the total number of new software releases and the number of major and minor software releases. 
Pull request activity is a more proximate category of outcomes  because pull requests are opened/merged at higher frequency and more closely map to an organization's overall level of activity.
In contrast, software release activity is a more downstream category of outcomes because releases are observed less frequently.
Software releases are akin to an organization's ``final output'' because releases often bundle the result of many merged pull requests. 
Ultimately, software users or other stakeholders are likely more interested in software updates, which deliver new features and address bugs, than the frequency of pull requests being opened and merged, which do not directly affect their usage of the software. 
I consider major and minor software releases as a separate outcome because those release types contain the most substantive changes to the software. 

I observe the aforementioned activity, including the member performing each activity and the date and content of their contributions, via the GitHub Archive (\citealt{GitHub_archive_GitHub_2025}).
In Section~\ref{sec:key_members}, I define the organization's key members and what defines an abrupt departure. 
I also defend the plausible exogeneity of these departures to unobservable organizational trends. 
In Section~\ref{sec:org_practices}, I describe the organizational practices of interest and how I use GitHub activity data to measure adoption. 
In Section~\ref{sec:treatment_sample}, I describe the final sample used in the analysis. 

\subsection{Key Members and Abrupt Departures}\label{sec:key_members}
\subsubsection{Defining key members}
Identifying the organization's key members requires a measure of each member’s prominence.
I define a member's prominence based on the organization's social network. 
The organization's social network is an undirected weighted graph constructed based on conversations between members.
The graph's nodes represent members, and edge weights between two members reflect the extent of their interactions. 
Each graph is constructed from activity data spanning a six month time period, corresponding to either the first or second half of the calendar year.
%\longterm[inline]{Appendix Section~\ref{sec:app_data} provides additional detail about the social network's construction. }


A member’s prominence is measured by their degree centrality in the social network.
A member's degree centrality is the number of unique members with whom they interacted during that time period.
In any given period, members who rank among the top three by degree centrality are defined as \textit{single-period key members}.\footnote{All members whose degree centrality is equal to or greater than that of the third-ranked member are included, so the number of single-period key members may exceed three when ties occur.}  
Members who are single-period key members for at least three consecutive periods are classified as \textit{key members} beginning in the third period.

My definition of an OSS organization's social network follows existing work by \cite{crowston_hierarchy_2006}. 
My definition of what it means to be a \textit{key member} is informed by two aspects of OSS organizations known from the literature. 
First, key members of OSS organizations interact with many more people than other members, which I capture by defining key members based on their degree centrality. 
The aim of non-key members of the OSS organization is typically to share the problems they are encountering or propose features they would like to see. 
Their goal is not to engage with the broader community; hence, they only interact with a select few of the organization's members (\citealt{hippel_open_2003}). 
In contrast, key members will participate in discussions started by many members; hence, I expect them to have high degree centrality.
Second, even though OSS organizations have no predetermined structure and community structure can vary substantially from project to project, the literature has found that a small minority of members account for a majority of the organization's activity (\citealt{mockus_two_2002}, \citealt{crowston_hierarchy_2006}). 
My measure captures this aspect by restricting key members to those with the three highest degree centrality scores in at least three consecutive periods.

\subsubsection{Abrupt Departures and their Plausible Exogeneity} \label{sec:departures_exogeneity}
I next define when a key member abruptly departs an organization. 
To qualify as having abruptly departed in period $t$, a key member in period $t$ must no longer have any recorded activity in the organization in all future periods. 
Existing survey evidence shows that abrupt departures occur because of ``some kind of transition (e.g., switching jobs or leaving academia)" or because members were ``having children or getting married" (\citealt{miller_why_2019}). 
The definition of abrupt departure that I build on from \cite{miller_why_2019} also requires key members to have been highly active for at least 3 consecutive 6-month periods. 

I believe that the motivations behind abrupt departures of key members is unrelated and hence plausibly exogenous to unobservable organizational trends that affect organizational outcomes. 
First, as noted by \cite{miller_why_2019}, abrupt departures are often driven by major life changes. 
These are, by definition, unrelated to an OSS organization's unobservable trends.
Since many OSS members are volunteers and contribute as a hobby, when major life changes occur, OSS contribution is often dropped as a consequence of there being only 24 hours in a day.
If one contributes to OSS as part of their academic program, they may stop contributing once they graduate because it is no longer required. 
Graduation is typically a fixed date that members cannot adjust around due to the costs of tuition (for undergraduate students) and limited funding (for graduate students).
It also seems implausible that the timing of one's marriage or having children, important milestone events in one's life, is related to unobservable organizational trends in one's hobby. 

Even if one contributes to OSS as part of their job because their company develops OSS, when they switch jobs, their departure is completely driven by the job switch as company rules likely require disengagement from company projects. 
Some developers also contribute to external OSS organizations as part of their job because OSS organizations contain tools valuable to their work at the job. 
However, not all employers support OSS contributions to external organizations, so job changes can also lead to disengagement even from projects unrelated to one's former employer. 
It seems reasonable to assume that job changes, which can occur for much more important personal reasons such as financial circumstances or changes in work location (compared to the OSS one works on), are unrelated to the unobservable trends of the OSS they worked on.
Even if the company's underlying trends caused the departure, for companies that do produce OSS, OSS development is typically is a small part of its operations because OSS as a product often does not bring in revenue.

The existing literature does cite organizational reasons, such as the absence of peer support, that might motivate departures from an organization. 
However, the key reason why these studies are not a threat to exogeneity is that they consider general departures that do not necessarily involve either key members or immediate and permanent disengagement. 
For example, \cite{miller_why_2019}, when discussing abrupt departures, find that
\begin{displayquote}
occupational reasons such as major life changes (e.g., getting a new job or leaving school) were the most cited (with 106 citations), significantly more than lacking peer support or losing interest that are more commonly discussed in the literature (\citealt{miller_why_2019}).
\end{displayquote}
Commonly cited reasons for departures from the literature that may be related to unobservable organizational trends include social factors or role changes (\citealt{iaffaldano_preliminary_2019}). 
I do not believe that my definition of departures includes departures driven by social reasons because social reasons are most often cited as barriers to joining or becoming an integral part of the project (\citealt{bosu_impact_2014}, \citealt{steinmacher_let_2019}).
In contrast, my analysis focuses on the departure of key members who have already been integral to the project for at least 18 months.
Sometimes, members have stop contributing to code changes because their role within the organization changes. 
However, since my definition of departure requires total disengagement, I do not think that ``role departures'' driven by unobservable organizational changes would be captured by my definition.
Even if someone stops writing as much code, I would still expect them, as a member of the project, to engage in discussion. 
This is a key way my definition differs from \cite{miller_why_2019}, who still permit low levels of engagement post-departure. 


\subsection{Organizational Practices}\label{sec:org_practices}
I am interested in the effects of three categories of organizational practices: knowledge redundancy among members of the organization, investment by the organization in talent and organizational routines. 
The choice was inspired by existing research in both the organizational resilience and OSS literature. 
For example, three of the four concepts that \cite{duchek_organizational_2020} describes as ``general attributes that may facilitate an organization's resilience'' map to my categories.\footnote{See section 2.2 of \cite{duchek_organizational_2020}. I rephrase what \cite{duchek_organizational_2020} terms ``redundancy,'' ``positive relationships,'' and ``specific organizational strategies'' as knowledge redundancy, investment in talent, and organizational routines, respectively. The fourth type is adequate resources, which is less applicable to OSS organizations as most organizations are made up of volunteers who collaborate remotely.}
\cite{rashid_systematic_2019} highlights disaster-mitigation practices that can be categorized as bolstering knowledge redundancy, increasing investment in talent and establishing organizational routines as potential mitigation strategies.\footnote{See section 6.2 of \cite{rashid_systematic_2019}. The knowledge redundancy practices are ``pair programming and shared code ownership,'' ``successor[s],'' and ``uniform knowledge distribution.'' The investment in talent practices are ``removal of knowledge barriers'' and ``improving code review feedback time for non-cores.''. The organizational routine is ``gamification''.}

The following sections describe the specific practices I use to characterize knowledge redundancy, talent investment and the adoption of organizational routines within OSS organizations. 

\subsubsection{Knowledge Redundancy}
Knowledge redundancy in an organization arises when multiple individuals possess the same knowledge required to solve a task. 
Organizational resilience can be enhanced via knowledge redundancy because it means the ability to solve a task is not lost when one member leaves. 
Since each member's knowledge cannot be directly observed, I proxy for it using three broad subcategories: assessing how often members collaborate, how often their activities undergo external review, and using a text-based method to quantify each member in the organization's problem-solving expertise. 

The first subcategory measures the extent of collaboration among members. 
Collaboration contributes to knowledge redundancy because it exposes multiple members to a task's problem solving process.
One metric I use to measure the adoption level of collaboration is the average number of members per discussion thread. 
Panel A of Table~\ref{tab:knowledge_redundancy_metrics:collab} lists all measurements associated with this subcategory. 

The second subcategory measures the extent to which a given member's codebase changes are reviewed by other members. 
High ranking members can make codebase changes without external review, but this is considered poor software engineering practice. 
External review contributes to knowledge redundancy because it ensures that multiple people have seen, reviewed, and possibly understand how codebase changes will alter the software.
One measurement I use to measure the adoption of external review is the proportion of merged pull requests that receive an ``approving'' pull request review.
Panel B of Table~\ref{tab:knowledge_redundancy_metrics:external} lists all measurements associated with this subcategory. 

The third subcategory uses each member's text discussions to produce organization-level aggregates of average and shared expertise. 
Here, the content of text discussions proxy for expertise. 
I define the average level of expertise as one minus the average cosine similarity between different discussion comments written by the same member, using \texttt{sentence-transformers/all-mpnet-base-v2} embeddings (\citealt{song_mpnet_2020}).\footnote{I transform cosine similarity values to be between 0 and 1. }
When members have more similar comments, they have lower expertise. 

To evaluate overlapping expertise, conceptually, I can compare the similarity of content written by different members. 
My measure of overlapping expertise is the ratio of the average cosine similarity between comments written by different members to the average level of expertise previously described. 
The numerator measures how much different members overlap in expertise; more similar comments indicate more shared expertise.
I include the denominator because overlap in expertise is inherently relative to each member's level of expertise. 
Suppose we have two organizations and both organizations have the same numerator, but in the first organization, the average level of expertise among members is higher.
The first organization has much lower shared expertise because as a proportion of each member's expertise, less is shared. 
Panel C of Table~\ref{tab:knowledge_redundancy_metrics:expertise} lists all measurements associated with the individual and shared expertise category. 

\askjesse[inline]{Would I benefit from using math here to define the text similarity metric?}

\subsubsection{Investment in members}
Organizations that invest in their members encourage learning by members and aim to foster positive environments. 
Existing research has shown for example, that resilient organizations invest in talent (\citealt{maor_raising_2022}) and that social ties within the organization matter (\citealt{luthans_psychological_2006}). 
I measure organizational investment in members by examining the characteristics of feedback they receive, the extent to which new members have opportunities to take on greater responsibility, and the cohesiveness of the organization’s social network.

The first subcategory measures the frequency, timeliness, and positivity of feedback received by members. 
For example, existing research shows that new developers are discouraged from continuing to participate in project when they experience delays in receiving feedback (\citealt{bosu_impact_2014}). 
I measure the timeliness of feedback by examining, for all discussion responses, the average number of days it takes to receive feedback. 
Panel A of Table~\ref{tab:talent_investment_metrics:response} lists all measurements associated with this subcategory. 

The second subcategory measures whether members have opportunities to take on greater responsibility. 
For example, on GitHub, only higher-level members within the organization have the authority to merge pull requests. 
Hence, I measure, in a given 6-month time period, the proportion of members who merged a pull request for the first time, out of all members who had joined in the previous 18 months and had never merged a pull request. 
Conceptually, being granted the ability to merge a pull request proxies for promotion. 
Panel B of Table~\ref{tab:talent_investment_metrics:growth} lists all measurements associated with this subcategory. 

The final subcategory examines how cohesive the organization's social network is. 
To measure this, I use the clustering coefficient, which is used to measure cohesiveness in networks (\citealt{wasserman_social_1994}, \citealt{serrano_zanetti_quantitative_2012}).
This is the only measure used to characterize the organization's cohesiveness and Panel C of Table~\ref{tab:talent_investment_metrics:cohesiveness} provides additional measurement details. 


\subsubsection{Organizational routines}
Organizations may also have routines it uses to solve problems.
Routines can contribute to organizational resilience by providing plans organizations can adhere to during times of crisis. 
Evidence from the literature suggests that in situations as wide ranging from surmounting Mount Everest (\citealt{suarez_building_2020}) to governing the Roman Empire (\citealt{carmeli_capture_2011}), organizational resilience is enhanced by having routines or combinations of routines in place. 
I classify organizational routines into three subcategories: assignment-based routines, problem-solving routines and initiation routines. 

The first subcategory measures whether organizations explicitly assign responsibilities to project members or add clarifying labels to tasks. 
Such actions enhance organizational resilience by eliminating ambiguity about responsibility for tasks.
For example, some projects have a file called the CODEOWNERS inside the software codebase, which formally delineates responsibility for different parts of the codebase to particular members. 
When a pull request is opened, members of the project responsible for the affected parts of the code are automatically assigned as pull request reviewers. 
This can be used to enable members to continue contributing, even in the absence of a member that explicitly coordinates task assignment. 
Panel A of Table~\ref{tab:org_routine_metrics:assignment} lists all measurements associated with this category. 

The second subcategory measures whether organizations systematize problem solving by requiring that the comment initiating any discussion contain certain pieces of information.
On GitHub, organizations can do this by creating issue and pull request templates.
For example, an organization's pull request template can require that prior to opening a pull request, members must have ensured that all unit tests have passed and that the documentation relevant to the code change has been updated. 
This enables organizations to continue maintaining high standards for pull requests, even in the absence of a leading member who might have implicitly enforced these standards in their pull request review. 
Panel B of Table~\ref{tab:org_routine_metrics:problem-solving} provides additional detail about measurements associated with this subcategory. 


The third category of metrics measures whether organizations adopt routines to assist new members joining the organization. 
Some organizations have contributing guides that introduce members to the project and ``good first issues" that are considered suitable tasks for new members to try and tackle.
These allow new members to independently learn about how they can help solve tasks as part of the organization. 
These may be especially valuable post-departure, when existing members in the organization may have less time to initiate new members because they are preoccupied with taking on the departed's tasks.  
Panel C of Table~\ref{tab:org_routine_metrics:initiation} provides additional detail about measurements associated with this subcategory. 


\subsection{Defining Treatment and the Final Sample} \label{sec:treatment_sample}
My sample spans 10 years from 2015-2024. 
Each time period $t$ in my analysis is 6 months. 
Treated organizations are those that experience at least one key member departure during the sample period; control organizations never experience a key member departure. 
I define the treatment date $E_i$ for an organization $i$ as the last period where the first key member who leaves is present. 
For control organizations that never experience a key member departure, $E_i = \infty$.
Any organization where $E_i = e$ is a member of the treatment cohort $e$. 

For each control organization, I assign a \textit{quasi-treatment date} $Q_i$, drawn from the empirical distribution of treatment dates among treated units conditional on the organization's first appearance in the sample. 
The quasi-treatment date $Q_i$ for treated units is equal to its treatment date. 
The concept of quasi-treatment date allows me to assign control organizations reference periods.
Event time for all organizations can thus be defined relative to the quasi-treatment date as $k=t-Q_i$.\footnote{I abuse notation slightly not specifying that event time $k_i$ is organization-specific to maintain the typical convention used}
I define the time periods where $k<0$ as an organization's pre-period, and $k>0$ as an organization's post period. 

The organizational outcomes $Y_{i,t}$ represent, in each period $t$, the number of pull requests opened, pull requests merged, new software releases, and major or minor software releases.
To ensure comparability across organizations, I standardize each organization’s outcome values using the organization-level mean and standard deviation from its five most recent pre-periods ($k = -5$ to $k = -1$).
Hence, my standardized outcomes $Y_{i,t}^{SD}$ are in units of standard deviations. 
For each organization in my sample, I measure adoption of each organizational practice $X_i = (X_i^1, \cdots, X_i^J)$ by calculating each of the $J$ metrics described in Section~\ref{sec:org_practices} over the five most recent pre-periods. 

I impose the following sample restrictions. 
First, all organizations must have experienced at most one key member departure during the sample period, eliminating contamination problems arising from multiple treatments.
Second, all organizations must have exactly one key member at the time of the quasi-treatment date. 
This ensures that comparisons between treated and control organizations are not biased by differences in the distribution of key members across treated and control organizations. 
Third, organizations must be observed for at least 5 pre- and post-periods.
Finally, all organizations must be a GitHub project corresponding to at least one widely-used Python library. 
I define a Python library as widely-used if, in any month between September 2018 and December 2024, it was downloaded at least 10,000 times, according to the Python Package Index (\citealt{the_Python_software_foundation_pypi_2025}). 
The sample features 699 organizations and includes major GitHub projects, including \texttt{pytorch/pytorch}, the flagship deep learning framework; \texttt{googleapis/google-cloud-Python}, the Python Google Cloud software development kit (SDK); and \texttt{boto/boto3}, the Python Amazon Web Services SDK. 

\section{Illustrative Event Studies} \label{sec:event_study}
\subsection{Organizational Resilience}\label{sec:org_resilience}
In this section, I estimate organizational resilience, defined as the causal effect of a key departure on organizational outcomes.
This analysis is meant to illustrate the empirical effect of departures and introduce the event study that serves as the backbone of my analysis. 

Following \cite{sun_estimating_2021}, I estimate the regression
\begin{equation}
    Y_{i,t}^{SD} = \alpha_i + \lambda_t + \sum_{e \neq \infty} \sum_{k \neq -1} \delta_{e, k} Z_{i,t}^{e,k}  + \epsilon_{i,t} \label{eq:event_study}
\end{equation}
where $Z_{i, t}^{e,k}$ equals 1 if 1) organization $i$ is at event time $k$ in period $t$ and 2) belongs to treatment cohort $e$, and 0 otherwise. 
I include organization fixed effects $\alpha_i$ and time fixed effects $\lambda_t$.
Standard errors are clustered at the organization level. 

Equation~\ref{eq:event_study} is robust to treatment effect heterogeneity across treatment cohorts $e$. 
To simplify the analysis, post-estimation, I aggregate the cohort-specific estimates $\hat{\delta}_{e,k}$ into an event-time average defined as the following: 
\begin{equation}\label{eq:event_time}
    \hat{\delta}_k = \sum_{e} \hat{\delta}_{e,k} \Pr(E_i = e \mid \text{treatment cohort } e \text{ is observed at event time } k)
\end{equation}

Under two identifying assumptions, $\hat{\delta}_{e, k}$ is an unbiased and consistent estimator of the causal effect of a key member departure on organizational outcomes $k$ periods post-treatment, for organizations in treatment cohort $e$. 
If the two assumptions hold, since the sample shares of each treatment cohort are also unbiased and consistent estimators for the population share, $\hat{\delta}_k$ will also be unbiased and consistent. 

\begin{assumption}[Parallel Trends]\label{ass:parallel_trends}
\text{For all time periods $s \neq t$}, 
\[
\E[Y_{i,s}^{SD,\infty} - Y_{i,t}^{SD, \infty}\mid E_i = e] 
\quad \text{is the same across all treatment cohorts $e$}
\]
\end{assumption}
$Y_{i,s}^{SD,\infty}$ is the value of $Y_{i,s}^{SD}$ in the event that $i$ is never treated. 
I believe this is reasonable because as discussed in Section~\ref{sec:departures_exogeneity}, since my treatment is whether an organization experiences an abrupt departure, the decision to depart is unrelated to unobservable organizational trends. 
Hence, outcomes for treated and control organizations should not differ along unobservable organizational trends. 


\begin{assumption}[No Anticipation]\label{ass:no_ant}
\[
\E[Y_{i,s}^{SD,\infty} - Y_{i,s}^{SD}\mid E_i = e]  = 0
\quad \text{for all pre-periods periods $s < Q_i$}
\]
\end{assumption}
One way the no anticipation assumption could be violated would be if the organization knows both that departure is occurring in advance and adjusts its behavior sufficiently far in advance. 
I think this is unlikely because each time period in my analysis is six months and the treatment period is the last period where the member is present.
Hence, violations would require the organization to start adjusting at least six months prior to the actual departure, conditional on even having advance notice. 
For example, I find it unlikely that the anticipation assumption affects organizations whose key member departures occur because the member experienced a job transition, a central core of the departees my measure captures. 
This is because in the technology industry, where most OSS members are employed, the standard notice period is two weeks, so it seems unlikely that the organizations would start adjusting six months prior.

The departures I observe are also permanent. 
I find it plausible that members who are taking temporary, but long-term breaks (such as parental leave) would wish to inform their supervisors of breaks several periods in advance to maintain goodwill.
However, given that in all cases, I can see that departed members have completely disengaged for at least 2.5 years, I do not see a strong reason why these members would fall into the category of those taking temporary long-term breaks.

Figure~\ref{fig:event_study} presents an event study plot of $\hat{\delta}_k$ for 5 pre- and post-periods. 
I find that the departure of a key member has a statistically significant negative effect on all organizational outcomes starting in the first treatment period, when the key member has left.
On average, in the 5 post-periods, a key member departure reduces the number of pull requests opened (merged) by 0.924 (0.960) standard deviations. 
In contrast, over the same horizon, a key member departure reduces the number of new (major and minor) software releases by 0.484 (0.385) standard deviations. 
Moreover, the decline in software releases levels out fairly quickly. 
Given the smaller decline in all new software releases compared to pull request activity, and the even comparatively smaller decline in major and minor software releases, it would be interesting to explore, in future work, how the characteristics of software releases are changing. 
For example, given the trends, do future software releases contain less new features, as a consequence of the decline in pull request activity?


In the bottom left corner of each event study plot, I report the p-value of the pre-trend Wald test $\delta_k = 0$ for the 5 plotted pre-periods $k$ (following the suggestion in \citealt{freyaldenhoven_visualization_2021}). 
Across all organizational outcomes, I fail to reject the null hypothesis that pre-trends do not exist. 
Moreover, pre-trend coefficients are closest to 0 during the pre-periods closes to the actual treatment date. 
This is reassuring because I would expect that the unobservable organizational trends that pose the greatest threats to identification are those visible, as pre-trends, in the periods immediately prior to treatment. 


\subsection{Descriptive Evidence on the Explanatory Power of Organizational Practices} \label{sec:human_event_study}
In this section, I will provide descriptive evidence that variation in the adoption of organizational practices does explain variation in organizational resilience. 
To do this, I use a \textit{median splitting rule} to subset organizations based off organizational practice adoption levels $X_i$. 
The median splitting rule says that for each organizational practice $j$, organization $i$ belongs to the ``low'' adoption subset if adoption level $X_{i,j}$ is below the median value, across all organizations, and the high adoption subset otherwise. 

Figure~\ref{fig:event_study_human} depicts event study estimates of $\hat{\delta}_k$ from estimating Equation~\ref{eq:event_study} separately for the ``high'' and ``low'' subsets of three selected organizational practices: whether organization discussions were positive\footnote{I measure sentiment based off the VADER score from \cite{hutto_vader_2014}, which was specifically designed for sentiment analysis on social media platforms}, whether members of the organization tended to work only on issues, which are discussion-based tasks (as opposed to also working on code-development tasks) and whether organizations assigned assignees to tasks.
In each event study plot, three p-values are reported. 
The first two report the p-value of the pre-trend Wald test estimated separately on the ``high'' and ``low'' subsets.
The last value reports the p-value from a Wald test under the null hypothesis that the post-treatment trajectories of the treatment effect following a key member’s departure are identical between the ``high'' and ``low'' subsets in all post-periods. Formally, define for any event time $k$, the event-time estimates $\hat{\delta}_k^{high},\hat{\delta}_k^{low}$ obtained from estimating Equation~\ref{eq:event_study} and Equation~\ref{eq:event_time} separately on the ``high'' and ``low'' subsets.  
I then test the joint null hypothesis: 
\begin{equation} \label{eq:wald}
H_0\colon \hat{\delta}_k^{high} - \hat{\delta}_k^{low} = 0 \text{ for all $1 \leq k \leq 5$}
\end{equation}

I find that variation in the adoption of organizational practices has some explanatory power, although the differences in reslience are not large. 
Moreover, some of the findings contradict my expectations based off the literature.
Panel 1 of Figure~\ref{fig:event_study_human} is one such surprising finding; organizations where the conversation tended to be less positive were more resilient.
Panel 2 of Figure~\ref{fig:event_study_human} suggests that organizations where a greater share of members worked only on issues tended to be less resilient.
This suggests that resilient organizations are those where members have experience across multiple tasks types. 
Not all organizational practices have explanatory power.
Panel 3 of Figure~\ref{fig:event_study_human} suggests organizations that tend to adopt a formal system of assigning individuals to tasks do not tend to be systematically more resilient, across most outcomes. 

A flaw with Figure~\ref{fig:event_study_human} is that it only presents a few hypotheses about the impact of adopting organizational practices. 
Figure~\ref{fig:event_study_att_human} summarizes, for all 37 organizational practices, the difference in the post-period ATT between the ``high'' and ``low'' subset for all four organizational outcomes. 
I find that for most organizational practices, the difference in the post-period ATT between the ``high'' and ``low'' subset across all  organizational outcomes is not statistically significant. 
Moreover, even when the difference is statistically significant, the results are typically not consistent across all outcomes or the difference in point estimates is either large and very noisy, or economically small. 

Two assumptions are required to interpret $\hat{\delta}_k^{high}$ and $\hat{\delta}_k^{low}$ (and hence, the ATT's) as causal effects.
These are that Assumption~\ref{ass:parallel_trends} and Assumption~\ref{ass:no_ant} from Section~\ref{sec:org_resilience} both hold for each of the two subsets.
A sufficient and stronger condition is that Assumption~\ref{ass:parallel_trends} and Assumption~\ref{ass:no_ant} hold conditional on any vector of organizational practices $X_i$. 
I will justify the stronger assumptions here; the reason will become apparent in the next section. 

\begin{assumption}[Conditional Parallel Trends]\label{ass:parallel_trends_cond}
\text{For all time periods $s \neq t$}, 
\[
\E[Y_{i,s}^{SD}(\infty) - Y_{i,t}^{SD}(\infty) \mid E_i = e, X_i = x] 
\quad \text{is the same across all treatment cohorts $e$}
\]
\end{assumption}
\begin{assumption}[Conditional No Anticipation]\label{ass:no_ant_cond}
\[
\E[Y_{i,s}^{SD}(\infty) - Y_{i,s}^{SD}\mid E_i = e, X_i = x]  = 0
\quad \text{for all pre-periods periods $s < Q_i$}
\]
\end{assumption}
$Y_{i,t}^{SD}(e)$ is potential outcomes notation for the outcome of organization $i$ at time $t$ if it is part of treatment cohort $e$. 
Conditional parallel trends is a strong assumption, but I believe it is plausible for two reasons. 
First, since abrupt key member departures are driven by major life changes that are not affected by one's participation in OSS development, it seems reasonable to assume that unobservable organizational trends do not affect one's participation decision, even conditional on organizational practices. 
Second, Assumption~\ref{ass:parallel_trends_cond} might be even more plausible than Assumption~\ref{ass:parallel_trends}, because we control for organizational practices, such as social cohesiveness, that the literature has described as motivating general departures. 
For the conditional anticipation assumption, while one might expect that organizations with certain characteristics might require longer notice periods, I still find it unlikely that these notice periods would be anywhere close to six months. 

For some subsets in Figure~\ref{fig:event_study_human}, I can reject the null hypothesis that there are no pre-trends. 
However, even in such cases, the pre-trends tend to trend closer to 0 as event time approaches treatment time. 
Rejection of the pre-trends test is driven by pre-trends from event times far from treatment time. 
As discussed in Section~\ref{sec:org_resilience}, I expect the pre-periods closest to treatment time to provide the most information about whether unobservable organizational trends might cause parallel trends violations. 

One conceptual note is that the plausibility of conditional parallel trends is a separate task from understanding whether the effect of adopting certain organizational practices is causal. 
Whether adoption is causal depends on the underlying reasons for why variation in organizational practice adoption exists. 

\subsection{Flaws of the Median Split Rule}\label{sec:median_split}
Empirically, Figure~\ref{fig:event_study_att_human} shows that the median split rule from Section~\ref{sec:human_event_study} does not provide conclusive statements about how organizational practices affect organizational resilience. 
My ability to understand how organizational practices affect organizational resilience using the median split rule is limited for several reasons. I discuss two below. 

First, the median split rule is poorly suited for discovering how organizational practices interact to affect organizational resilience. 
This is a well-known problem that arises due to the curse of dimensionality.
The number of possible interactions between organizational practices increases factorially in the number of variables in each interaction.
Moreover, naively reporting covariates combinations where I observe statistically significant differences across subgroups using the Wald test from Equation~\ref{eq:wald} complicates inference due to multiplicity of testing. 
Interactions are important because organizations are complex and multiple factors may work in tandem to contribute to their resilience. 

Second, since the median split rule does not use a data-driven approach to subset organizations, some subsets with heterogeneity may be omitted. 
This is because it is not necessarily true (and likely incorrect) to assume ex-ante that for any given organizational practice, splitting along the median and using two subsets is the best way to model induced heterogeneity in organizational resilience. 
For example, suppose that organizational resilience is highest for only values at the left and right tail of the adoption distribution for a particular organizational practice. 
In this case, the Wald test from Equation~\ref{eq:wald} may report that the coefficients for event studies run on the two subsets produced by the median split rule are statistically indistinguishable.
Assuming based off this that no heterogeneity exists is incorrect; the relationship between adoption of the organizational practice and organizational resilience is just nonlinear. 

These two reasons combined suggest that developing an improved splitting rule may yield dividends. 
\section{Machine Learning Optimal Splits}
Conceptually, my goal is to estimate heterogeneous treatment effects. 
One standard approach for heterogeneous treatment effects estimation is to apply the generalized random forest from \cite{athey_generalized_2019}.
The generalized random forest estimates heterogeneous treatment effects by adaptively splitting the covariate space to group together observations with comparable treatment effects, before averaging treatment effects within these groups.
This splitting procedure is done thousands of times, each on a different random subsample of the dataset; the set of splits associated with each random subsample is called a tree (hence the forest moniker). 
This procedure addresses both flaws discussed in Section~\ref{sec:median_split}, as generalized random forests capture complex interactions and adaptively split the covariate space to group observations based off treatment effect similarity.

However, I cannot apply the generalized random forest out-of-the-box because I am estimating dynamic treatment effects and treatment adoption is staggered. 
To deal with this, in a separate note, I develop the \textit{generalized random forest event study}.
\href{https://drive.google.com/file/d/1brRivFrurkeuNHZHLgT_iuaZx_kuhXAz/view?usp=share_link}{\textbf{Liao (Forthcoming)}} shows that estimating heterogeneous treatment effects in event study settings with staggered adoption and unit-invariant covariates is equivalent to estimating a special case of the conditional linear model with binary regressors and panel-data aware propensity scores. 
The conditional linear model estimator is unbiased and consistent when estimated using generalized random forests (\citealt{athey_generalized_2019}), so the estimated event study heterogeneous treatment effects inherit these desirable properties under assumptions tailored to my panel data setting.

\href{https://drive.google.com/file/d/1brRivFrurkeuNHZHLgT_iuaZx_kuhXAz/view?usp=share_link}{\textbf{Liao (Forthcoming)}} call this procedure the generalized random forest event study. Section~\ref{sec:assumptions_grf_es} provides a brief overview of the model and describes the identifying assumptions. 
Section~\ref{sec:optimal_split} describes and applies it an improved splitting rule 
using the generalized random forest event study.
Section~\ref{sec:policy_tree} uses the generalized random forest event study splits to learn about which organizational practices contribute to organizational resilience. 

\subsection{Applying the Generalized Random Forest Event Study}\label{sec:assumptions_grf_es}
I model the standardized organizational outcome $Y_{i,t}^{SD}$ as
\begin{equation}\label{eq:event_study_ml}
Y_{i,t}^{SD} = \alpha_i + \lambda_t + \sum_{e \neq \infty} \sum_{k \neq -1} \delta_{e, k}(X_i) Z_{i,t}^{e,k} + \varepsilon_{i,t} 
\end{equation}
Since my aim is to estimate the local parametric function $\delta_{e,k}(x)$ as opposed to an average treatment effect, in addition to conditional parallel trends and conditional no anticipation, identification requires a conditional independence assumption. 
I also assume that the regularity conditions from Section 3 of \cite{athey_generalized_2019}  hold. 

\begin{assumption}[Conditional exogeneity]\label{ass:no_confound} For all organizations $(i, j)$ and time periods $(s, t)$
$$\E[\varepsilon_{i,t} \mid \{X_j\}_{j}, \{Z_{j, s}\}_{(j, s)} ]$$
\text{ where $Z_{j, s}$ is a vector of $Z_{j, s}^{e, k}$ for all $e\neq \infty$ and $k \neq 1$.}
\end{assumption}
I believe this assumption is plausible for two reasons.
First, each key member's departure is unrelated to underlying organizational trends that would affect organizational outcomes. 
Moreover, since OSS organizations operate independently, it seems unlikely that the organizational practices or key member departures of one would affect the other. 
While OSS organizations do exist within the same ecosystem, my model accounts for aggregate shocks, such as temporal changes in Python popularity, via $\lambda_t$. 

One way to make the model in Equation~\ref{eq:event_study_ml} even more flexible is to replace time fixed effects $\lambda_t$ with flexible time-varying functions of covariates $f_t(X_i)$.
These would represent baseline heterogeneity induced by covariates across time. 
I think it is unlikely that this more flexible model, which is also more difficult to estimate, would lead to substantial improvements for two reasons. 
First, since covariates are unit invariant, I already embed a time-invariant notion of baseline heterogeneity induced by covariates via $\alpha_i$, which can be rewritten as $f(X_i)$ without cost.
Second, I find it unlikely that organizational practices would have systematically different aggregate effects in, for example, 2023 than 2016. 

The procedure I use to estimate the heterogeneous treatment effects $\hat{\delta}_{e,k}(x)$ follows \cite{athey_generalized_2019}.\footnote{I use the default hyperparameters as of version 2.4.0.}
As discussed in \href{https://drive.google.com/file/d/1brRivFrurkeuNHZHLgT_iuaZx_kuhXAz/view?usp=share_link}{\textbf{Liao (Forthcoming)}}, I use panel-aware propensity scores in the estimation.
I estimate the model using a $k$-fold procedure. 
$k-1$ folds of the data are used to train the model.
I then use the trained model to estimate $\hat{\delta}_{e,k}(x)$ for the hold-out fold. 
This is done $k$ times, on the $k$ different possible splits. 
Using a $k-$fold procedure ensures that small biases from the outcome and propensity score models used for estimation will not affect $\hat{\delta}_{e,k}(x)$.
For each organization's cohort-event time estimates $\hat{\delta}_{e,k}(X_i)$, I calculate the doubly robust equivalent  $\hat{\psi}_{e, k}(X_i)$ using the same $k$-fold procedure and panel-aware propensity scores. 
I adapt \cite{uysal_dr_2015}, who describe propensity scores estimation with multivalued treatments, for my setting with staggered treatments. 
For each organization, I then calculate the doubly robust post-period treatment effect, 
\begin{equation}\label{eq:doubly_robust}
    \hat{\psi}(X_i) = \frac{1}{5} \sum_{k >0}^{5} \hat{\psi}_{e, k}(X_i)
\end{equation}

\subsection{The Optimal Split Rule}\label{sec:optimal_split}

To assess whether organizational practices affect organizational resilience, I define the \textit{optimal split rule} to subset organizations into ``high'' and ``low'' resilience subsets. 
Using just the standardized pull requests merged outcome, I estimate the treatment cohort event time heterogeneous treatment effect $\hat{\delta}_{e, k}(X_i)$.
I then transform it into the doubly robust post-period treatment effect $\hat{\psi}(X_i)$. 
Organizations are classified into the ``low'' resilience subset when $\hat{\psi}(X_i)$ is below the median value across all organizations, and the ``high'' resilience subset otherwise. 
I use the doubly robust scores rather than the heterogeneous treatment effect estimates $\hat{\delta}_{e,k}(x)$ because they are robust to either outcome or treatment propensity model misspecification. 

Figure~\ref{fig:event_study_optimal} depicts estimates of Equation~\ref{eq:event_study} estimated separately on the ``high'' and ``low'' subsets determined using the optimal split rule. 
I find that organizational resilience does vary substantially based on pre-departure organizational practices. 
The difference is both statistically significant and economically large. 
On average, across all 5 post-periods, the difference between the number of pull requests opened (merged) by organizations in the ``high'' versus ``low'' resilience subset is 1.998 (2.019) standard deviations.
On average, across all 5 post-periods, the difference between the number of new (major and minor) software releases by organizations in the ``high'' versus ``low'' resilience subset is 1.144 (0.680) standard deviations.
Since the heterogeneous treatment effects used for classification are hold-out fold predictions, we know organizational practices actually have explanatory power and are not fitting to training-sample noise. 
As a consequence of splitting on doubly robust scores from an estimation that only uses pull requests merged, the ``high'' and ``low'' subsets are the same in all plots, so the same organizational practices induce resilience across many outcomes. 

One concern with Figure~\ref{fig:event_study_optimal} is that in the event study plots using pull-request activity as an outcome, I observe statistically significant pre-trends. 
These pre-trends are concerning for two reasons.
First, they are statistically significant in the pre-periods close to the departure.
Second, they suggest that pre-departure, organizations in the ``high'' (``low'') subset that eventually experience departure are more (``less'') productive that their never-treated counterparts. 
This suggests that the magnitude of the treatment effects is biased upwards, the treatment effect likely includes both the causal effect of the departure, and the unobservable trend affecting the organization's outcomes. as observed in Panel~\ref{fig:event_study_prs_merged_optimal}.
Although the pre-trend magnitudes are much smaller in magnitude than the actual difference in effects and do not follow any perceptible upwards or downwards trend, future work should consider adopting methods from \cite{rambachan_pt_2022} to perform robust inference under parallel trends assumption violations. 


\subsection{Identifying Organizational Practices that Matter} \label{sec:policy_tree}
\subsubsection{The Policy Tree Split Rule}
While Figure~\ref{fig:event_study_optimal} illustrates that organizational practices do explain substantial variation in organizational resilience, the second goal of the paper, which is what practices matter most, remains unanswered. 
The vast difference observed in Figure~\ref{fig:event_study_optimal} (in contrast to the results in Figure~\ref{fig:event_study_att_human}) suggests that interactions and potential nonlinearities between covariates matter. 
Widely-used interpretability approaches for forest-based methods such as the SHAP value or reporting what \% of trees use an organizational practice when splitting, are limited in their ability to describe interactions of importance. 
Instead, in this section, I use a \textit{policy tree} (\citealt{athey_policy_2021}) to provide a lower-dimensional summary of the \textit{generalized random forest event study}. 


Given a pre-specified tree depth $k$, the policy tree identifies the mapping $\pi(X_i) $ that maps organizational practices to either a low or high resilience subset by solving the following
\begin{equation}
\pi^*=\arg\max_{\pi\in\Pi_k};\frac{1}{n}\sum_{i=1}^n \Gamma_i\big(\pi(X_i)\big).
\end{equation}
$\Gamma_i (\cdot)\in\mathbb R^2$ denotes the organization-level reward vector for assigning that unit to the ``low'' or ``high'' resilience subset. 

The policy tree determines $\pi^*$ by partitioning the space of organizational practices via individual organizational practice splits; the number of candidate partitions grows with the number of features and with tree depth $k$.
I call the policy tree's assignments the \textit{policy tree split rule}.
I specify that the reward $\Gamma_i(x_i)$ is the doubly robust estimate of the post-period treatment effect $\hat{\psi}(X_i)$ if the organization is classified as belonging to the ``high'' subset, and 0 if classified into the ``low'' subset. 
Although the policy tree is limited by only being able to use up to $2^{k}-1$ covariates, it is not required to have a balanced ``high'' and ``low'' subset, unlike the previous rules which split along the median of a distribution. 

\subsubsection{Applying the Policy Tree Split Rule}
I will first show how the policy tree performs in a k-fold estimation procedure. 
I will then train a policy tree on the full sample to show how the policy split rule's decision process. 

Panel~\ref{fig:event_study_policy_tree} depicts estimates of Equation~\ref{eq:event_study} for the ``low'' and ``high'' resilience subset, as determined by the policy tree split rule, for trees of depth 1, 2, and 3. 
Recall again that across all outcomes, the doubly robust post-period treatment effects used to determine the splits are estimated using just one outcome: pull requests merged. 
Panel 1 of Figure~\ref{fig:event_study_policy_tree} shows that when we deviate from splitting along the median, splitting along one organizational practice can still identify meaningful variation in organizational resilience. 
The difference in average post-period treatment effects for the high vs. low resilience subset is statistically significant for all outcomes except the number of major and minor software releases.
Although substantially smaller than differences observed in Figure~\ref{fig:event_study_optimal}, the differences are still economically meaningful for pull-request related activity. In standard deviations, the ``high''-``low'' differences are 0.855 for pull requests opened, 0.662 for pull requests merged, 0.225 for new software releases and 0.313 for major and minor software releases. 

Panel 2 of Figure~\ref{fig:event_study_policy_tree} shows that using two organizational practices can improve the difference in performance. 
The differences are both economically meaningful and statistically significant for all outcomes except for major and minor software releases. In standard deviations, the ``high''-``low'' differences are 0.953 for pull requests opened, 0.795 for pull requests merged, 0.674 for new software releases and 0.217 for major and minor software releases. 
Panel 3 of Figure~\ref{fig:event_study_policy_tree} shows that the policy tree split is likely overfitting to noise in the data when using 3 layers, as differences between the ``high'' and ``low'' resilience subsets are statistically insignificant. 

Figure~\ref{fig:policy_tree} depicts the organizational practices that the policy tree identifies as distinguishing more and less resilient organizations.
The 1-layer policy tree shows that organizations with a high text \textit{similarity ratios} are more resilient. 
This measure proxies for the level of overlapping expertise between members. 
In the tree with 2 layers, I find that organizations with both high text similarity ratios and high \textit{overall sentiment} are resilient. 
An organization's overall sentiment is higher when conversations in the organization tend to be more positive. 
I also find that organizations with low overall sentiment, and no \textit{contributing guide} to initiate newcomers, are resilient. 
This latter result is quite surprising, as both of these practices contribute to a more insular and unwelcoming project, which the literature does not find to contribute to organizational resilience. 
One potential explanation is that newcomers, who have less experience, may actually worsen a project that is already suffering from a departure. 
Taken together, these results suggest that while some of the literature's proposed organizational practices, such as knowledge redundancy and positive communication, do strengthen resilience, there are also resilient organizations that deviate substantially from these recommendations. 

\section{Conclusion}
I develop measures of organizational resilience and practice using detailed activity-level data on open source software projects.
I show, via a novel method combining event study and heterogeneous treatment effect estimation, that differences in organizational practices explain substantial variation in organizational resilience. 
Some of the literature's hypotheses on what makes an organization resilience are supported and some are rejected. 

There are many interesting directions for future research.
I can expand the set of organizational practices to include data on code changes and characteristics of important project members, and the set of organizational outcomes to include software release characteristics such as quality, security and usage. 
I can enrich the analysis by shining light on how organizational practices contribute to resilience and whether the adoption of organizational practices includes tradeoffs.
I can strengthen the plausibility of identifying assumptions by providing empirical evidence on the circumstances surrounding abrupt departures, such as whether they coincide with job transitions or graduation, and whether they co-occur with confounding organizational changes or circumstances such as a large reorganization or having just met a major milestone. 

\newpage
\begin{longtable}{>{\raggedright\arraybackslash}p{0.2\textwidth} p{0.8\textwidth}}
\caption{Knowledge Redundancy Metrics}\label{tab:knowledge_redundancy_metrics} \\
\toprule
\textbf{Name} & \textbf{Definition} \\
\midrule
\endfirsthead

\caption*{\textbf{Table \ref{tab:knowledge_redundancy_metrics} (continued):} Knowledge Redundancy Metrics} \\
\toprule
\textbf{Name} & \textbf{Definition} \\
\midrule
\endhead

\bottomrule
\multicolumn{2}{r}{\textit{Continued on next page}} \\
\endfoot

\bottomrule
\endlastfoot

\multicolumn{2}{l}{\phantomsection\label{tab:knowledge_redundancy_metrics:collab}\textbf{Panel A}: Collaboration}
\\%
\noalign{\vskip-0.6ex}
\midrule
Average discussants & Average \# of discussants per issue or pull request thread \\
Multiple discussants & \% of issue or pull request threads with multiple discussants \\
Review concentration & Organization-level Herfindahl–Hirschman Index based on each member's share of pull request review comments \\
Comment concentration & Organization-level Herfindahl–Hirschman Index based on each member's share of issue or pull request comments \\
Issue thread concentration & Weighted average of thread-level Herfindahl–Hirschman Indices based on members’ shares of comments within issue threads, weighted by each thread’s total comment count \\
Pull request comment concentration & Weighted average of thread-level Herfindahl–Hirschman Indices based on members’ shares of pull request comments within pull request threads, weighted by each thread’s total comment count \\
Pull request review concentration & Weighted average of thread-level Herfindahl–Hirschman Indices based on members’ shares of pull request reviews within pull request review threads, weighted by each thread’s total comment count \\
Pull request review comment concentration & Weighted average of thread-level Herfindahl–Hirschman Indices based on members’ shares of pull request review comments within pull request review threads, weighted by each thread’s total comment count \\
\midrule

\multicolumn{2}{l}{\phantomsection\label{tab:knowledge_redundancy_metrics:external}\textbf{Panel B}: External Review}%
\\
\noalign{\vskip-0.6ex} 
\midrule
Approved review share & \% of merged pull requests that received at least one ``approved'' pull request review \\
Any review share & \% of merged pull requests that received any pull request review \\
Discussion volume & Average number of discussion activities per pull request, consisting of pull request comments, review comments, and reviews \\
\midrule

\multicolumn{2}{l}{\phantomsection\label{tab:knowledge_redundancy_metrics:expertise}\textbf{Panel C}: Expertise Level and Overlap}%
\\
\noalign{\vskip-0.6ex} 
\midrule
Issue only share & \% of members involved in only issue threads \\
Pull request only share & \% of members involved in only pull request threads \\
Both issue and pull request share & \% of members involved in both issue and pull request threads \\
Unique activity types & Average number of unique activities each member engages in, across the following nine options: opening, commenting on, or closing issues; and opening, commenting on, leaving review comments, reviewing, or merging pull requests \\
Same author similarity & Average cosine similarity between embeddings from (\citealt{song_mpnet_2020}) of up to 1,000 pairs of distinct comments authored by the same member \\
Different author similarity & Average cosine similarity between embeddings from \cite{song_mpnet_2020} of up to 1,000 pairs of distinct comments authored by different members \\
Similarity ratio & Ratio of \textbf{Same author similarity} to \textbf{Different author similarity} \\

\end{longtable}

\vspace{-0.5em} % tighten vertical space
\begin{minipage}{\textwidth}
\footnotesize
\setlength{\parskip}{0pt}
\setlength{\baselineskip}{0.9\baselineskip}
% \textit{Notes:} Placeholder for table notes. 
\end{minipage}
\newpage
\begin{longtable}{>{\raggedright\arraybackslash}p{0.2\textwidth} p{0.8\textwidth}}
\caption{Talent Investment Metrics}\label{tab:talent_investment_metrics} \\
\toprule
\textbf{Name} & \textbf{Definition} \\
\midrule
\endfirsthead

\caption*{\textbf{Table \ref{tab:talent_investment_metrics} (continued):} Talent Investment Metrics} \\
\toprule
\textbf{Name} & \textbf{Definition} \\
\midrule
\endhead

\bottomrule
\multicolumn{2}{r}{\textit{Continued on next page}} \\
\endfoot

\bottomrule
\endlastfoot

\multicolumn{2}{l}{\phantomsection\label{tab:talent_investment_metrics:response}\textbf{Panel A}: Feedback characteristics}
\\%
\noalign{\vskip-0.6ex}
\midrule
Response rate & Percentage of discussion threads that received a response, excluding pull request reviews \\
Response time & Average \# of days per response \\
Response positivity & Average positivity score of all discussion, according to VADER (\citealt{hutto_vader_2014}) \\
Response negativity & Average negative score of all discussion, according to VADER (\citealt{hutto_vader_2014}) \\
Overall sentiment & Compound sentiment score of all discussion, according to VADER (\citealt{hutto_vader_2014}) \\
Pull request response quantity & Average amount of discussion (consisting of pull request comments, review comments and reviews) per pull request by everyone except the pull request opener \\
\midrule

\multicolumn{2}{l}{\phantomsection\label{tab:talent_investment_metrics:growth}\textbf{Panel B}: Growth opportunities}%
\\
\noalign{\vskip-0.6ex} 
\midrule
Advanced to merging & \% of members who joined in the past 18 months and had never merged a pull request before, who merged one for the first time in this period \\
Advanced to reviewing & \% of members who joined in the past 18 months and had never reviewed a pull request before, who reviewed one for the first time in this period \\
Advanced to merging or reviewing & \% of members who joined in the past 18 months and had never merged or reviewed a pull request before, who merged or reviewed one for the first time in this period \\
\midrule

\multicolumn{2}{l}{\phantomsection\label{tab:talent_investment_metrics:cohesiveness}\textbf{Panel C}: Social network cohesiveness}%
\\
\noalign{\vskip-0.6ex} 
\midrule
Clustering coefficient & Weighted clustering coefficient of the organization's social network (\citealt{saramaki_generalizations_2007}) \\
\end{longtable}

\vspace{-0.5em} % tighten vertical space
\begin{minipage}{\textwidth}
\footnotesize
\setlength{\parskip}{0pt}
\setlength{\baselineskip}{0.9\baselineskip}
%\textit{Notes:} Placeholder for table notes. 
\end{minipage}


\newpage
\begin{longtable}{>{\raggedright\arraybackslash}p{0.2\textwidth} p{0.8\textwidth}}
\caption{Organizational Routine Metrics}\label{tab:org_routine_metrics} \\
\toprule
\textbf{Name} & \textbf{Definition} \\
\midrule
\endfirsthead

\caption*{\textbf{Table \ref{tab:org_routine_metrics} (continued):} Talent Investment Metrics} \\
\toprule
\textbf{Name} & \textbf{Definition} \\
\midrule
\endhead

\bottomrule
\multicolumn{2}{r}{\textit{Continued on next page}} \\
\endfoot

\bottomrule
\endlastfoot

\multicolumn{2}{l}{\phantomsection\label{tab:org_routine_metrics:assignment}\textbf{Panel A}: Assignment-based routines}
\\%
\noalign{\vskip-0.6ex}
\midrule
Assigned Reviewers & Average \% of pull requests with assigned reviewers \\
Assigned Assignee & Average \% of issues or pull requests with a member assigned to work on it \\
Assigned Tag & Average \% of issues or pull requests with tags labelling \\
Has Codeowners & Whether an organization has a CODEOWNERS file on GitHub defining which members are in charge of the codebase \\ 


\midrule

\multicolumn{2}{l}{\phantomsection\label{tab:org_routine_metrics:problem-solving}\textbf{Panel B}: Problem-solving routines}%
\\
\noalign{\vskip-0.6ex} 
\midrule
Has issue template & Whether an organization uses issue templates to systematize inquiries and discussion  \\ 
Has pull request template & Whether an organization uses issue templates to systematize proposed codebase changes \\ 


\midrule

\multicolumn{2}{l}{\phantomsection\label{tab:org_routine_metrics:initiation}\textbf{Panel C}: Initiation routines}%
\\
\noalign{\vskip-0.6ex} 
\midrule
Has contributing guide & Whether an organization has a dedicated guide to onboarding potential new members \\
Has code of conduct & Whether an organization has a dedicated guide for to ``define community standards, signal a welcoming and inclusive project, and outline procedures for handling abuse'' (\citealt{github_adding_2025}) \\
Has good first issues & Whether an organization has dedicated problems specifically for new contributors to work on \\
\end{longtable}

\vspace{-0.5em} % tighten vertical space
\begin{minipage}{\textwidth}
\footnotesize
\setlength{\parskip}{0pt}
\setlength{\baselineskip}{0.9\baselineskip}
%\textit{Notes:} Placeholder for table notes. 
\end{minipage}


\begin{figure}[t]
  \caption{Average Organizational Resilience}
    \label{fig:event_study}
    \centering
    \begin{minipage}[b]{0.48\textwidth}
        \centering
        \subcaption{Pull requests opened} \label{fig:event_study_prs_opened}
        \includegraphics[width=\textwidth]{output/analysis/event_study/important_topk_exact1/rolling5/nevertreated/github_outcomes/pull_request_opened_norm}
    \end{minipage}
    \hfill
    \begin{minipage}[b]{0.48\textwidth}
        \centering
        \subcaption{Pull requests merged} \label{fig:event_study_prs_merged}
        \includegraphics[width=\textwidth]{output/analysis/event_study/important_topk_exact1/rolling5/nevertreated/github_outcomes/pull_request_merged_norm.png}
    \end{minipage}
    \hfill
    \begin{minipage}[b]{0.48\textwidth}
        \centering
        \subcaption{New software releases} \label{fig:event_study_releases}
        \includegraphics[width=\textwidth]{output/analysis/event_study/important_topk_exact1/rolling5/nevertreated/downstream_outcomes/major_minor_release_count_norm}
    \end{minipage}
    \begin{minipage}[b]{0.48\textwidth}
        \centering
        \subcaption{Major and minor software releases} \label{fig:event_study_major_releases}
        \includegraphics[width=\textwidth]{output/analysis/event_study/important_topk_exact1/rolling5/nevertreated/downstream_outcomes/overall_new_release_count_norm.png}
    \end{minipage}

  \begin{minipage}{1\textwidth}
    \textbf{Figure notes:} Panel~\subref{fig:event_study_prs_opened} depicts event study estimates from \cite{sun_estimating_2021} showing the impact of key member departures on the number of pull requests opened. Each outcome is standardized by the organization’s mean and standard deviation over the five pre-periods. Confidence intervals are 95\% confidence intervals calculated using asymptotic standard errors clustered at the organization level. Panel~\subref{fig:event_study_prs_merged},~\subref{fig:event_study_releases} and~\subref{fig:event_study_major_releases} replace the outcome in Panel~\subref{fig:event_study_prs_opened} with pull requests merged, new software releases, and major and minor software releases.
\end{minipage}
  

\end{figure}
\begin{figure}[t]
  \centering
  % Disable automatic "(1a)" label so only "Panel (1a)" appears
  \captionsetup[subfigure]{position=top,font=normalsize,labelfont={},textfont={},labelformat=empty,labelsep=none}
  \caption{Average Organizational Resilience Using the Median-Split Rule}
  \label{fig:event_study_human}
  \setlength{\tabcolsep}{2pt}

  \newcounter{panelrow}

  \begin{tabular}{@{}m{0.05\textwidth}*{4}{>{\centering\arraybackslash}m{0.23\textwidth}}@{}}
    & Pull requests opened & Pull requests merged & New software releases & Major \& minor releases \\[-0.5ex]

    % Row 2: (2a–2d)
    \centering\rotatebox{90}{\parbox{0.15\textwidth}{\centering\normalfont\normalsize Overall sentiment}} &
    \stepcounter{panelrow}\setcounter{subfigure}{0}\renewcommand{\thesubfigure}{\arabic{panelrow}\alph{subfigure}}%
    \begin{subfigure}[t]{0.23\textwidth}\centering
      \caption{Panel (\thesubfigure)}
      \includegraphics[width=\linewidth]{output/analysis/event_study/important_topk_exact1/rolling5/nevertreated/talent_investment/feedback/response_sentiment/ov_sentiment_avg/pull_request_opened_norm.png}
    \end{subfigure} &
    \setcounter{subfigure}{1}\renewcommand{\thesubfigure}{\arabic{panelrow}\alph{subfigure}}%
    \begin{subfigure}[t]{0.23\textwidth}\centering
      \caption{Panel (\thesubfigure)}
      \includegraphics[width=\linewidth]{output/analysis/event_study/important_topk_exact1/rolling5/nevertreated/talent_investment/feedback/response_sentiment/ov_sentiment_avg/pull_request_merged_norm.png}
    \end{subfigure} &
    \setcounter{subfigure}{2}\renewcommand{\thesubfigure}{\arabic{panelrow}\alph{subfigure}}%
    \begin{subfigure}[t]{0.23\textwidth}\centering
      \caption{Panel (\thesubfigure)}
      \includegraphics[width=\linewidth]{output/analysis/event_study/important_topk_exact1/rolling5/nevertreated/talent_investment/feedback/response_sentiment/ov_sentiment_avg/overall_new_release_count_norm.png}
    \end{subfigure} &
    \setcounter{subfigure}{3}\renewcommand{\thesubfigure}{\arabic{panelrow}\alph{subfigure}}%
    \begin{subfigure}[t]{0.23\textwidth}\centering
      \caption{Panel (\thesubfigure)}
      \includegraphics[width=\linewidth]{output/analysis/event_study/important_topk_exact1/rolling5/nevertreated/talent_investment/feedback/response_sentiment/ov_sentiment_avg/major_minor_release_count_norm.png}
    \end{subfigure} \\[-1.8ex]

    % Row 3: (3a–3d)
    \centering\rotatebox{90}{\parbox{0.15\textwidth}{\centering\normalfont\normalsize Issue only share}} &
    \stepcounter{panelrow}\setcounter{subfigure}{0}\renewcommand{\thesubfigure}{\arabic{panelrow}\alph{subfigure}}%
    \begin{subfigure}[t]{0.23\textwidth}\centering
      \caption{Panel (\thesubfigure)}
      \includegraphics[width=\linewidth]{output/analysis/event_study/important_topk_exact1/rolling5/nevertreated/knowledge_redundancy/broad_expertise/issue_pr_share/share_issue_only/pull_request_opened_norm.png}
    \end{subfigure} &
    \setcounter{subfigure}{1}\renewcommand{\thesubfigure}{\arabic{panelrow}\alph{subfigure}}%
    \begin{subfigure}[t]{0.23\textwidth}\centering
      \caption{Panel (\thesubfigure)}
      \includegraphics[width=\linewidth]{output/analysis/event_study/important_topk_exact1/rolling5/nevertreated/knowledge_redundancy/broad_expertise/issue_pr_share/share_issue_only/pull_request_merged_norm.png}
    \end{subfigure} &
    \setcounter{subfigure}{2}\renewcommand{\thesubfigure}{\arabic{panelrow}\alph{subfigure}}%
    \begin{subfigure}[t]{0.23\textwidth}\centering
      \caption{Panel (\thesubfigure)}
      \includegraphics[width=\linewidth]{output/analysis/event_study/important_topk_exact1/rolling5/nevertreated/knowledge_redundancy/broad_expertise/issue_pr_share/share_issue_only/overall_new_release_count_norm.png}
    \end{subfigure} &
    \setcounter{subfigure}{3}\renewcommand{\thesubfigure}{\arabic{panelrow}\alph{subfigure}}%
    \begin{subfigure}[t]{0.23\textwidth}\centering
      \caption{Panel (\thesubfigure)}
      \includegraphics[width=\linewidth]{output/analysis/event_study/important_topk_exact1/rolling5/nevertreated/knowledge_redundancy/broad_expertise/issue_pr_share/share_issue_only/major_minor_release_count_norm.png}
    \end{subfigure} \\

    
    % Row 1: (1a–1d)
    \centering\rotatebox{90}{\parbox{0.15\textwidth}{\centering\normalfont\normalsize Assigned Asignee}} &
    \stepcounter{panelrow}\setcounter{subfigure}{0}\renewcommand{\thesubfigure}{\arabic{panelrow}\alph{subfigure}}%
    \begin{subfigure}[t]{0.23\textwidth}\centering
      \caption{Panel (\thesubfigure)}
      \includegraphics[width=\linewidth]{output/analysis/event_study/important_topk_exact1/rolling5/nevertreated/organizational_routines/organized_assignment/discussion_assignees/mean_has_assignee/pull_request_opened_norm.png}
    \end{subfigure} &
    \setcounter{subfigure}{1}\renewcommand{\thesubfigure}{\arabic{panelrow}\alph{subfigure}}%
    \begin{subfigure}[t]{0.23\textwidth}\centering
      \caption{Panel (\thesubfigure)}
      \includegraphics[width=\linewidth]{output/analysis/event_study/important_topk_exact1/rolling5/nevertreated/organizational_routines/organized_assignment/discussion_assignees/mean_has_assignee/pull_request_merged_norm.png}
    \end{subfigure} &
    \setcounter{subfigure}{2}\renewcommand{\thesubfigure}{\arabic{panelrow}\alph{subfigure}}%
    \begin{subfigure}[t]{0.23\textwidth}\centering
      \caption{Panel (\thesubfigure)}
      \includegraphics[width=\linewidth]{output/analysis/event_study/important_topk_exact1/rolling5/nevertreated/organizational_routines/organized_assignment/discussion_assignees/mean_has_assignee/overall_new_release_count_norm.png}
    \end{subfigure} &
    \setcounter{subfigure}{3}\renewcommand{\thesubfigure}{\arabic{panelrow}\alph{subfigure}}%
    \begin{subfigure}[t]{0.23\textwidth}\centering
      \caption{Panel (\thesubfigure)}
      \includegraphics[width=\linewidth]{output/analysis/event_study/important_topk_exact1/rolling5/nevertreated/organizational_routines/organized_assignment/discussion_assignees/mean_has_assignee/major_minor_release_count_norm.png}
    \end{subfigure} 
  \end{tabular}
  \begin{minipage}{1\textwidth}
  \textbf{Figure notes:} Panel (1a) depicts separate event study estimates from \cite{sun_estimating_2021} showing the impact of key member departures on the number of pull requests opened, for organizations with below and above median overall sentiment. Each outcome is standardized by the organization’s mean and standard deviation over the five pre-periods. Confidence intervals are 95\% confidence intervals calculated using asymptotic standard errors clustered at the organization level. Panel (1b), (1c) and (1d) replace the outcome in Panel (1a) with pull requests merged, new software releases, and major and minor software releases. All plots in Panel (2) depict separate estimates for organizations below and above the median share of members who only work on issues. All plots in Panel (3) depict separate estimates for organizations below and above the median share of problems with assigned members. 

\end{minipage}
  
\end{figure}

\begin{figure}[t]
  \caption{High-low ATT difference by organizational practices and outcomes}
\label{fig:event_study_att_human}
    \centering
    \begin{minipage}[b]{1.1\textwidth}
        \centering
        \includegraphics[width=\textwidth]{output/analysis/event_study/important_topk_exact1/rolling5/att_by_practice_outcomes_grouped.png}
    \end{minipage}

  \begin{minipage}{1\textwidth}
    \textbf{Figure notes:} Each row depicts the difference between the ATT of event study estimates over the 5 post-periods for organizations with above and below median adoption of the organizational characteristic, for four organizational outcomes: pull requests opened, pull requests merged, new software releases, and major and minor software releases.
    Confidence intervals are 95\% confidence intervals for the difference in ATT estimates. 
    For rows without difference points, there were not sufficient observations in either the above median or below median subset to estimate both event studies.     
\end{minipage}
  

\end{figure}
\begin{figure}[t]
    \caption{Average Organizational Resilience Using the Optimal-Split Rule}
    \label{fig:event_study_optimal}
    \centering
    \begin{minipage}[b]{0.48\textwidth}
        \centering
        \subcaption{Pull requests opened} \label{fig:event_study_prs_opened_optimal}
        \includegraphics[width=\textwidth]{output/analysis/event_study_personalization/important_topk_exact1/rolling5/nevertreated/pull_request_opened_norm_split_pull_request_merged_lm_forest_observed_att_dr_group.png}
    \end{minipage}
    \hfill
    \begin{minipage}[b]{0.48\textwidth}
        \centering
        \subcaption{Pull requests merged} \label{fig:event_study_prs_merged_optimal}
        \includegraphics[width=\textwidth]{output/analysis/event_study_personalization/important_topk_exact1/rolling5/nevertreated/pull_request_merged_norm_split_pull_request_merged_lm_forest_observed_att_dr_group.png}
    \end{minipage}
    \hfill
    \begin{minipage}[b]{0.48\textwidth}
        \centering
        \subcaption{New software releases} \label{fig:event_study_releases_optimal}
        \includegraphics[width=\textwidth]{output/analysis/event_study_personalization/important_topk_exact1/rolling5/nevertreated/overall_new_release_count_norm_split_pull_request_merged_lm_forest_observed_att_dr_group.png}
    \end{minipage}
    \begin{minipage}[b]{0.48\textwidth}
        \centering
        \subcaption{Major and minor software releases} \label{fig:event_study_major_releases_optimal}
        \includegraphics[width=\textwidth]{output/analysis/event_study_personalization/important_topk_exact1/rolling5/nevertreated/major_minor_release_count_norm_split_pull_request_merged_lm_forest_observed_att_dr_group.png}
    \end{minipage}

  \begin{minipage}{1\textwidth}
    \textbf{Figure notes:} Panel~\subref{fig:event_study_prs_opened_optimal} depicts separate event study estimates from \cite{sun_estimating_2021} showing the impact of key member departures on the number of pull requests opened, for organizations classified as ``low'' or ``high'' resilience based on the optimal split rule. Each outcome is standardized by the organization’s mean and standard deviation over the five pre-periods. Confidence intervals are 95\% confidence intervals calculated using asymptotic standard errors clustered at the organization level. Panel~\subref{fig:event_study_prs_merged_optimal},~\subref{fig:event_study_releases_optimal} and~\subref{fig:event_study_major_releases_optimal} replace the outcome in Panel~\subref{fig:event_study_prs_opened_optimal} with pull requests merged, new software releases, and major and minor software releases.
\end{minipage}
  
  

\end{figure}
\begin{figure}[t]
  \centering
  % Disable automatic "(1a)" label so only "Panel (1a)" appears
  \captionsetup[subfigure]{position=top,font=normalsize,labelfont={},textfont={},labelformat=empty,labelsep=none}
  \caption{Organizational Resilience Using the Policy Tree Assignments}
  \label{fig:event_study_policy_tree}
  \setlength{\tabcolsep}{2pt}

    \setcounter{panelrow}{0}

  \begin{tabular}{@{}m{0.05\textwidth}*{4}{>{\centering\arraybackslash}m{0.23\textwidth}}@{}}
    & Pull requests opened & Pull requests merged & New software releases & Major \& minor releases \\[-0.5ex]

    % Row 2: (2a–2d)
    \centering\rotatebox{90}{\parbox{0.15\textwidth}{\centering\normalfont\normalsize Depth 1}} &
    \stepcounter{panelrow}\setcounter{subfigure}{0}\renewcommand{\thesubfigure}{\arabic{panelrow}\alph{subfigure}}%
    \begin{subfigure}[t]{0.23\textwidth}\centering
      \caption{Panel (\thesubfigure)}
      \includegraphics[width=\linewidth]{output/analysis/event_study_personalization/important_topk_exact1/rolling5/nevertreated_policy_tree_kfold_depth1/pull_request_opened_norm_rolling5_kfold_depth1_lm_forest_kfold_depth1_observed.png}
    \end{subfigure} &
    \setcounter{subfigure}{1}\renewcommand{\thesubfigure}{\arabic{panelrow}\alph{subfigure}}%
    \begin{subfigure}[t]{0.23\textwidth}\centering
      \caption{Panel (\thesubfigure)}
      \includegraphics[width=\linewidth]{output/analysis/event_study_personalization/important_topk_exact1/rolling5/nevertreated_policy_tree_kfold_depth1/pull_request_merged_norm_rolling5_kfold_depth1_lm_forest_kfold_depth1_observed.png}
    \end{subfigure} &
    \setcounter{subfigure}{2}\renewcommand{\thesubfigure}{\arabic{panelrow}\alph{subfigure}}%
    \begin{subfigure}[t]{0.23\textwidth}\centering
      \caption{Panel (\thesubfigure)}
      \includegraphics[width=\linewidth]{output/analysis/event_study_personalization/important_topk_exact1/rolling5/nevertreated_policy_tree_kfold_depth1/overall_new_release_count_norm_rolling5_kfold_depth1_lm_forest_kfold_depth1_observed.png}
    \end{subfigure} &
    \setcounter{subfigure}{3}\renewcommand{\thesubfigure}{\arabic{panelrow}\alph{subfigure}}%
    \begin{subfigure}[t]{0.23\textwidth}\centering
      \caption{Panel (\thesubfigure)}
      \includegraphics[width=\linewidth]{output/analysis/event_study_personalization/important_topk_exact1/rolling5/nevertreated_policy_tree_kfold_depth1/major_minor_release_count_norm_rolling5_kfold_depth1_lm_forest_kfold_depth1_observed.png}
    \end{subfigure} \\[-1.8ex]



    % Row 2: (2a–2d)
    \centering\rotatebox{90}{\parbox{0.15\textwidth}{\centering\normalfont\normalsize Depth 2}} &
    \stepcounter{panelrow}\setcounter{subfigure}{0}\renewcommand{\thesubfigure}{\arabic{panelrow}\alph{subfigure}}%
    \begin{subfigure}[t]{0.23\textwidth}\centering
      \caption{Panel (\thesubfigure)}
      \includegraphics[width=\linewidth]{output/analysis/event_study_personalization/important_topk_exact1/rolling5/nevertreated_policy_tree_kfold_depth2/pull_request_opened_norm_rolling5_kfold_depth2_lm_forest_kfold_depth2_observed.png}
    \end{subfigure} &
    \setcounter{subfigure}{1}\renewcommand{\thesubfigure}{\arabic{panelrow}\alph{subfigure}}%
    \begin{subfigure}[t]{0.23\textwidth}\centering
      \caption{Panel (\thesubfigure)}
      \includegraphics[width=\linewidth]{output/analysis/event_study_personalization/important_topk_exact1/rolling5/nevertreated_policy_tree_kfold_depth2/pull_request_merged_norm_rolling5_kfold_depth2_lm_forest_kfold_depth2_observed.png}
    \end{subfigure} &
    \setcounter{subfigure}{2}\renewcommand{\thesubfigure}{\arabic{panelrow}\alph{subfigure}}%
    \begin{subfigure}[t]{0.23\textwidth}\centering
      \caption{Panel (\thesubfigure)}
      \includegraphics[width=\linewidth]{output/analysis/event_study_personalization/important_topk_exact1/rolling5/nevertreated_policy_tree_kfold_depth2/overall_new_release_count_norm_rolling5_kfold_depth2_lm_forest_kfold_depth2_observed.png}
    \end{subfigure} &
    \setcounter{subfigure}{3}\renewcommand{\thesubfigure}{\arabic{panelrow}\alph{subfigure}}%
    \begin{subfigure}[t]{0.23\textwidth}\centering
      \caption{Panel (\thesubfigure)}
      \includegraphics[width=\linewidth]{output/analysis/event_study_personalization/important_topk_exact1/rolling5/nevertreated_policy_tree_kfold_depth2/major_minor_release_count_norm_rolling5_kfold_depth2_lm_forest_kfold_depth2_observed.png}
    \end{subfigure} \\[-1.8ex]

    
    % Row 2: (2a–2d)
    \centering\rotatebox{90}{\parbox{0.15\textwidth}{\centering\normalfont\normalsize Depth 3}} &
    \stepcounter{panelrow}\setcounter{subfigure}{0}\renewcommand{\thesubfigure}{\arabic{panelrow}\alph{subfigure}}%
    \begin{subfigure}[t]{0.23\textwidth}\centering
      \caption{Panel (\thesubfigure)}
      \includegraphics[width=\linewidth]{output/analysis/event_study_personalization/important_topk_exact1/rolling5/nevertreated_policy_tree_kfold_depth3_split50/pull_request_opened_norm_rolling5_kfold_depth3_split50_lm_forest_kfold_depth3_split50_observed.png}
    \end{subfigure} &
    \setcounter{subfigure}{1}\renewcommand{\thesubfigure}{\arabic{panelrow}\alph{subfigure}}%
    \begin{subfigure}[t]{0.23\textwidth}\centering
      \caption{Panel (\thesubfigure)}
      \includegraphics[width=\linewidth]{output/analysis/event_study_personalization/important_topk_exact1/rolling5/nevertreated_policy_tree_kfold_depth3_split50/pull_request_merged_norm_rolling5_kfold_depth3_split50_lm_forest_kfold_depth3_split50_observed.png}
    \end{subfigure} &
    \setcounter{subfigure}{2}\renewcommand{\thesubfigure}{\arabic{panelrow}\alph{subfigure}}%
    \begin{subfigure}[t]{0.23\textwidth}\centering
      \caption{Panel (\thesubfigure)}
      \includegraphics[width=\linewidth]{output/analysis/event_study_personalization/important_topk_exact1/rolling5/nevertreated_policy_tree_kfold_depth3_split50/overall_new_release_count_norm_rolling5_kfold_depth3_split50_lm_forest_kfold_depth3_split50_observed.png}
    \end{subfigure} &
    \setcounter{subfigure}{3}\renewcommand{\thesubfigure}{\arabic{panelrow}\alph{subfigure}}%
    \begin{subfigure}[t]{0.23\textwidth}\centering
      \caption{Panel (\thesubfigure)}
      \includegraphics[width=\linewidth]{output/analysis/event_study_personalization/important_topk_exact1/rolling5/nevertreated_policy_tree_kfold_depth3_split50/major_minor_release_count_norm_rolling5_kfold_depth3_split50_lm_forest_kfold_depth3_split50_observed.png}
    \end{subfigure} 
  \end{tabular}
  \begin{minipage}{1\textwidth}
  \textbf{Figure notes:} Panel (1a) depicts separate event study estimates from \cite{sun_estimating_2021} showing the impact of key member departures on the number of pull requests opened, for organizations classified as ``high'' or ``low'' resilience based on the policy split rule for depth 1 policy trees. Each outcome is standardized by the organization’s mean and standard deviation over the five pre-periods. Confidence intervals are 95\% confidence intervals calculated using asymptotic standard errors clustered at the organization level. Panel (1b), (1c) and (1d) replace the outcome in Panel (1a) with pull requests merged, new software releases, and major and minor software releases. All plots in Panel (2) depict separate estimates for organizations classified as ``high'' or ``low'' resilience based on the policy split rule for depth 2 policy trees. All plots in Panel (3) depict separate estimates for organizations classified as ``high'' or ``low'' resilience based on the policy split rule for depth 3 policy trees.
  
\end{minipage}
  
\end{figure}

\begin{figure}[t]
  \caption{Policy Tree-Split Rule}
  \telljesse[inline]{Sorry in advance about the formatting here}
    \label{fig:policy_tree}
    \centering
    \begin{minipage}[b]{0.4\textwidth}
        \centering
        \subcaption{Depth 1 policy tree} \label{fig:policy_tree_depth1}
        \includegraphics[width=\textwidth]{temp/policytree_depth1.png}
    \end{minipage}
    \hfill
    \begin{minipage}[b]{0.8\textwidth}
        \centering
        \subcaption{Depth 2 policy tree} \label{fig:policy_tree_depth2}
        \includegraphics[width=\textwidth]{temp/policytree_depth2.png}
    \end{minipage}

  \begin{minipage}{1\textwidth}
    \textbf{Figure notes:} Panel~\subref{fig:policy_tree_depth1} depicts a depth 1 policy tree trained on all observations. Panel~\subref{fig:policy_tree_depth1} depicts a depth 2 policy tree trained on all observations. 
\end{minipage}
  

\end{figure}

\clearpage
\bibliographystyle{source/paper/aea}
\bibliography{source/paper/references}


\clearpage
\onehalfspacing
\appendix
\renewcommand{\thesection}{\Alph{section}}
\renewcommand{\thesubsection}{\thesection.\arabic{subsection}}
\section{Additional Details on Data Construction}\label{sec:app_data}
\todo[inline]{I really don't like the wording here but fine to leave as is}
The social network for each organization in a given time period is constructed from all discussion threads that occurred within that six month period. Each discussion thread corresponds to a distinct conversational space on GitHub. There are three types of discussion threads: an *issue thread* (an issue post and its comments), a *pull request thread* (a pull request post and subsequent pull request comments), or a *pull request review thread* (a pull request post and the review comments attached to specific code lines). Within each thread, comments are ordered chronologically, with the first comment representing the opening post that initiates the discussion. For every subsequent comment, I identify the most recent prior comment made by a different contributor and treat this as a directed interaction from the current commenter (sender) to that prior participant (receiver). If the comment occurs in a review thread and is the first review comment, it is treated as a reply to the pull request opener. This procedure ensures that interactions represent meaningful exchanges between distinct contributors rather than self-replies or system events. Actor identifiers are standardized by converting all numeric and string IDs to a canonical string format, and duplicate comment records are removed to avoid inflating edge counts.

These pairwise interactions are then aggregated into undirected, weighted graphs for each time period. Each node represents a unique contributor active during that window, and an undirected edge connects two contributors if at least one interaction occurred between them in either direction. The edge weight equals the total number of interactions recorded between the pair in that period, capturing the intensity of their conversational exchange. 

\section{List of todos (unordered)}
My aim is to organize these into an ordered list of todos

\begin{enumerate}
    \item Use microdata on OSS organizations and members to directly show how the organizational practices in the lower-dimensional rule contribute to or detract from organizational resilience. 
    \item Incorporate a Case study on a few organizations and how they responded
    \item Figure out how much of the data is captured by the torrent and perform a validation exercise (\href{https://GitHub.com/igrigorik/gharchive.org/issues/290\#issuecomment-1923017346}{one of a few known reported data outages}).
    \item I still wonder if visuals of GitHub would help illuminate things. MY data is a strength but I'm not showing any of it
    \item For each measurement and data construction piece, make sure bots are excluded in the code and make that clear in the text
    \item See if there's a way I can leverage the fact that I know whose departing and use that to motivate organizational practices
    \item Focus on investment in talent or communication with non-key members
    \item Is there a way I can dynamically examine CODEOWNERS (ie: are people being removed when they should be???)
    \item It would be nice if the organizational routines I'm considering focus more on disaster preparation as opposed to just ordinary problem solving routines. 
    \item Autofill key numbers
    \item What are ways I can test no anticipation???
    \item Why do PRs decline more than releases?
    \item Separate event studies for each subgroup in the 2... 
    \item Test 3 treatments: neutral, low and high. 
    \item Can I show that I'm actually only getting "one" departure by showing it's unlikely organizations were affected by departures in their pre-period?
    \item Can I use more granular time periods
    \item Summary statistics: average outcomes, number of periods, Average \# of members, Facts abt the departed, Picture of a graph
    \item Read this\href{https://pubs.aeaweb.org/doi/pdfplus/10.1257/jep.36.3.211}{paper} in order to understand how to motivate the paper using descriptive statistics
    \item \longterm[inline]{Once I get the WRDS access, see if I can incorporate the picture of the ES graph of corporate departues for my chosen specification and subset to the main subset of organizations I'm using? Highlights how many of these are corporate related}
    \item \longterm[inline]{\textbf{Show this is true}: Abrupt departures aren't associated with changes in demand for the departed member's services - the number of issues opened, forks, stars and downloads are still increasing.  Moreover, the departure date is unrelated to the proximity to major software releases/updates}
    \item  \longterm[inline]{Can I show that organizational characteristics that might lead to dissatisfaction/change aren't the driving reason for departure/aren't systematically changing in a way that would explain their departure? For example, I don't see declines or increases in member count prior to their departure (signalling broad organizational changes) or changes in the distribution of communication z score/important members (which also suggests changes in how people are communicating/working) or in the sentiment of their communications (are they getting burnt out?)}
    \item knowledge redundancy table: Problem-level HHI should include opening issue/comment?
    \item knowledge redundancy table: Average unique types should aggregate review comments and reviews — maybe all activities should.





\end{enumerate}

\section{How I map PyPi organizations to GitHub repositories}

\section{How I match repo names to repo ids in cases where identity changes}

\end{document}