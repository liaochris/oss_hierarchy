%
\documentclass[12pt,notitlepage]{article}
\usepackage{amssymb}
\usepackage{amsmath}
\usepackage{yhmath}
\usepackage{graphicx}
\usepackage{epstopdf}
\usepackage{pdflscape}
\usepackage{tabularx}
\usepackage{longtable}
\usepackage{array}
\usepackage{dsfont}
\usepackage{float}
\usepackage{booktabs}
\usepackage{tikz}
\usepackage{marvosym}
\usepackage{multirow}
\usepackage{pdflscape}
\usepackage[hyphens]{url}
\usepackage{setspace}
\usepackage{epigraph}
\usepackage{bm}
\usepackage{textcomp}
\usepackage{diagbox}
\usepackage{bbm}
\usepackage{verbatim}
\usepackage[framemethod=tikz]{mdframed}
\usepackage{subcaption}
\usepackage{caption}
\usepackage{lipsum}
\usepackage{mathtools}
\usepackage{scalerel}
\usepackage{stackengine}
\usepackage{amsthm}

\usepackage[
backend=biber,
style=authoryear-comp,
sorting=ynt
]{biblatex}
\addbibresource{references.bib} %Import the bibliography file
\usepackage{hyperref}
\usepackage[shortlabels]{enumitem}
\setlength{\epigraphrule}{0pt}
\setlength\parindent{0pt}
\renewcommand{\baselinestretch}{1.25}

\setcounter{MaxMatrixCols}{10}

\newcolumntype{L}[1]{>{\raggedright\let\newline\\\arraybackslash\hspace{0pt}}m{#1}}
\newcolumntype{C}[1]{>{\centering\let\newline\\\arraybackslash\hspace{0pt}}m{#1}}



\newcommand{\I}{\mathbb{I}}
\newcommand{\E}{\mathbb{E}}
\newcommand{\Ll}{\mathrm{L}}
\newcommand{\R}{\mathbb{R}}
\renewcommand{\L}{\mathbb{L}}
\newcommand{\Var}{\mathrm{Var}}
\newcommand{\Cov}{\mathrm{Cov}}
\newcommand{\Corr}{\mathrm{Corr}}
\newcommand{\Prob}{\mathbb{P}}
\newcommand{\supp}{\mathrm{supp}}
\newcommand{\notimplies}{\mathrel{{\ooalign{\hidewidth$\not\phantom{=}$\hidewidth\cr$\implies$}}}}
\newcommand{\var}{\mathrm{var}}
\newcommand{\Bias}{\mathrm{Bias}}
\newcommand{\cov}{\mathrm{cov}}
\newcommand{\corr}{\mathrm{corr}}
\newcommand{\MSE}{\mathrm{MSE}}

\topmargin=-1.5cm \textheight=23cm \oddsidemargin=0.5cm
\evensidemargin=0.5cm \textwidth=15.5cm

\newtheorem{theorem1}{Special Theorem}

\newtheorem{ass}{Assumption}
\newtheorem{definit}{Definition}
\newtheorem{prop}{Proposition}
\newtheorem{thm}{Theorem}
\newtheorem{lem}{Lemma}
\newtheorem{conj}{Conjecture}
\newtheorem{cor}{Corollary}
\newtheorem{rem}{Remark}

\renewcommand{\thesubsection}{\arabic{section}.\arabic{subsection}}
\renewcommand{\thesubsubsection}{\arabic{section}.\arabic{subsection}.\arabic{subsubsection}}

\newcommand\dapprox{\stackrel{\mathclap{\tiny \mbox{d}}}{\approx}}
\newcommand\papprox{\stackrel{\mathclap{\tiny \mbox{p}}}{\approx}}
\newcommand\pconverge{\stackrel{\mathclap{\tiny \mbox{p}}}{\to}}
\newcommand\dconverge{\stackrel{\mathclap{\tiny \mbox{d}}}{\to}}

\addbibresource{references.bib}


\newcommand\independent{\protect\mathpalette{\protect\independenT}{\perp}}
\def\independenT#1#2{\mathrel{\rlap{$#1#2$}\mkern2mu{#1#2}}}

\usepackage{epsfig,hyperref}

\hypersetup{
	pdftitle={undergrad thesis},    % title
	pdfauthor={Chris Liao},     % author
	pdfnewwindow=true,      % links in new window
	colorlinks=true,       % false: boxed links; true: colored links
	linkcolor=blue,          % color of internal links
	citecolor=red,        % color of links to bibliography
	filecolor=black,      % color of file links
	urlcolor=blue           % color of external links
}

\allowdisplaybreaks

\begin{document}
\textbf{Hierarchy/Core in OSS}
\begin{enumerate}
    \item \cite{crowston_core_2006}
    \begin{itemize}
        \item Provides examples of \"Academic case studies of FLOSS Projects" that have hierarchical structures
        \item \"At the center of the onion are the core developers, who contribute most of the code and oversee the design and evolution of the project. In the next ring out are the co-developers who submit patches (e.g., bug fixes), which are reviewed and checked in by core developers."
    \end{itemize}
    \item \cite{crowston_hierarchy_2006}
    \begin{itemize}
        \item FLOSS projects are not uniformly centralized/hierarchical
        \item Projects are highly connected
        \item Lots of one way communication in projects
        \item Projects become more decentralized as project size grows (expect there to be more write rank contributors as project size increases)
    \end{itemize}
    \item \cite{wessel_github_2023}
    \begin{itemize}
        \item Categorizes actions across four categories: utilities, continuous integration, code quality, and deployment
        \begin{itemize}
            \item \textcolor{red}{Look into other tables from https://link.springer.com/article/10.1007/s10664-023-10369-w/tables/1}
        \end{itemize}
        \item increase in \% of merged PRs (but then decreases over time), decrease in \% of unmerged PRs (continues to decreases over time) 
        \item decrease in comments in merged PRs (but increases over time), increase in comments in unmerged PRs (that also decreases over time) 
        \item faster close to merged PRs (but gets slower over time), faster close to unmerged PRs 
        \item more commits during review (but decreases over time) for merged PRs, more commits during review for unmerged PRs
        \item Does project-level analysis, does not consider heterogeneity between programmer rank 
    \end{itemize}
    \item \cite{rashid_exploring_2017} 
    \begin{itemize}
        \item OSS project contributors, particularly new ones, encounter increased knowledge acquiition costs when turnover occurs 
    \end{itemize}
    \item \cite{avelino_abandonment_2019}
    \begin{itemize}
        \item Examine how knowledge acquisition costs change when projects are "abandoned", by examining truck factor detachment
        \item \textcolor{red}{con for my work is tat you can't measure hierarchical changes in totally defunct projects}
    \end{itemize}
    \item \cite{hata_characteristics_2015}
    \begin{itemize}
        \item How to Contribute document is very important for projects
    \end{itemize}
    \item \cite{rigby_quantifying_2016}
    \begin{itemize}
        \item Truck factor might exaggerate the damage caused by developers leaving, using git blame (to see who changed last line of code in a file) might be better. 
    \end{itemize}
    \item \cite{mcdonald_performance_2013} - OSS contributors care more about project visilbility, contributor count than code quality as a metric of success. 
\end{enumerate}

\end{document}