\documentclass[source/paper/main.tex]{subfiles}
\begin{document}
\begin{enumerate}
    \item \cite{garicano_hierarchies_2000} assume time spent helping can be nonstochastic because of LLN - note that time spent helping is a function of $F(k^r)$. Note also that \cite{garicano_hierarchies_2000} assumes linear utility by the firm because the firm is just optimizing output. I will also assume the nonstochasticity of helping, so that $t_h^w$ is a deterministic constant
\end{enumerate}
Difference between $u(P(t_p^r) F(k^r))$ and $P(t_p^r) u(F(k^r))$ is twofold.
\begin{enumerate}
    \item The marginal benefit of $t_p^r$ in $u(P(t_p^r) F(k^r))$ is affected by the curvature of $u$ while the marginal benefit of $t_p^r$ in $P(t_p^r) u(F(k^r))$ is affected by the shape of $u$ 
    \item In $u(t_p^rF(k^r))$, our marginal utility is affected by the number of problems we solve correctly, $P(t_p^r)F(k^r)$. In $P(t_p^r) u(F(k^r))$, our marginal utility is affected by effort and knowledge separately. 
    \item It seems like it's reasonable to separate \% of problems solved from number of problems attempted. Even omitting costs from the conversation, we don't want to treat the satisfaction someone gets from solving 1 problem (after trying to solve 10) the same way someone gets from solving 1 problem (after trying to solve 1). 
\end{enumerate}
Let $P_s$ be the proportion solved, with $P_s \sim N(F(k^r), \frac{F(k^r)(1-F(k^r))}{t_p^r})$. Since $$E[P(t_p^r)u(P_s))] = P(t_p^r)E[u(P_s)]$$ and we can rewrite $$E[u(P_s)] \approx u(F(k^r)) + \frac{u''(F(k^r))F(k^r)(1-F(k^r))}{2t_p^r}$$
then the read rank programmer solves 
$$\max_{k^r, t_p^r} P(t_p^r)\left(u(F(k^r)) + \frac{u''(F(k^r))F(k^r)(1-F(k^r))}{2t_p^r}\right) - c_r(t_p^r, k^r)$$
with first order conditions
\begin{align*}
    [k^r] &: P(t_p^r)f(k^r)\left(u'(F(k^r)) + \frac{u'''(F(k^r))F(k^r)(1-F(k^r)) + u''(F(k^r))  (1-2F(k^r))}{t_p^r}\right)= \frac{\partial c_r}{\partial k^r}\\
    [t_p^r] &: P'(t_p^r) u(F(k^r)) + \frac{(2P'(t_p^r)t_p^r + P(t_p^r))(F(k^r)(1-F(k^r)))}{4(t_p^r)^2}u''(F(k^r)) = \frac{\partial c_r}{\partial t_p^r}
\end{align*}
Assuming risk neutrality about $u_r$, then $u'' = u''' = 0$ so 
$$P(t_p^r)f(k^r)\alpha_r = \frac{\partial c_r}{\partial k^r} \qquad P'(t_p^r)\alpha_r F(k^r) = \frac{\partial c_r}{\partial t_p^r}$$
\begin{itemize}
    \item Suppose $u_r = \alpha_r F(k^r)$
    \item Suppose $P(t_p^r) = P t_p^r$
    \item Suppose $F(k^r) = 1-e^{-(k^r - 1)}$
\end{itemize}


Example $c_r(k^r, t_p^r) = \alpha_k (k^r)^2 + \alpha_p (t_p^r)^2$\\
$$\frac{P(t_p^r) f(k^r) }{P'(t_p^r) F(k^r)} = \frac{\frac{\partial c_r}{\partial k^r}}{\frac{\partial c_r}{\partial t_p^r}} \iff \frac{t_p^r f(k^r) }{F(k^r)} = \frac{\frac{\partial c_r}{\partial k^r}}{\frac{\partial c_r}{\partial t_p^r}} \iff \frac{t_p^r f(k^r) }{F(k^r)} = \frac{\alpha_k k^r}{\alpha_p t_p^r} \iff t_p^r = \sqrt{\frac{\alpha_k}{\alpha_p} k^r (e^{k^r-1} - 1)} $$
Thus, the problem becomes
\begin{align*}
    &\max_{t_p^r, k^r} P  t_p^r  u(F(k^r)) - \alpha_k (k^r)^2 - \alpha_p (t_p^r)^2 \\
    &= \max_{k^r} P\alpha_r(1-e^{-(k^r-1)}) \sqrt{\frac{\alpha_k}{\alpha_p} k^r (e^{(k^r-1)} - 1)}  -\alpha_k (k^r)^2 - \alpha_k k^r (e^{(k^r-1)} - 1)
\end{align*}
Thus, at the optimal $k^r$, 
\begin{align*}
    &\frac{\partial }{\partial k^r} \left(P\alpha_r(1-e^{-(k^r-1)}) \sqrt{\frac{\alpha_k}{\alpha_p} k^r (e^{(k^r-1)} - 1)}\right) = \frac{\partial }{\partial k^r} \left( \alpha_k (k^r)^2 - \alpha_k k^r (e^{(k^r-1)} - 1) \right) = 0 \\
    &\iff P \alpha_r \left(\frac{1}{\alpha_k \alpha_p}\right)^{\frac12 } (1 - e^{-(k^r-1)}) = 2 \sqrt{k^r (e^{(k^r-1)} - 1)}
\end{align*}
and we find that
\begin{align}
    \frac{2\sqrt{k^r(e^{(k^r-1)} - 1)}}{1-e^{-(k^r-1)}} &= \frac{P\alpha_r}{\sqrt{\alpha_k \alpha_p}} \label{knowledge_quadratic} \\
    t_p^r &= \frac{P\alpha_r}{2\alpha_p} (1-e^{-(k^r-1)}) \label{production_effort_quadratic}
\end{align}
I've omitted the algebra for the optimal $k^r$ calculation and included it in 
\href{run:source/paper/math/quadratic.pdf}{quadratic.pdf}
When $\alpha_k$, the cost of knowledge acquisition increases, $k^r$ decreases as $\frac{2\sqrt{k^r(e^{(k^r-1)} - 1)}}{1-e^{-(k^r-1)}} $ is increasing in $k^r$, per \ref{knowledge_quadratic}. \textcolor{red}{Add note about the part where it's decreasing and how that's not relevant}. Production effort decreases because while the marginal rate of substitution means we shift effort towards production, the negative effect of the increased cost in knowledge and the reduction in production's marginal benefit (because of a reduction in knowledge) is greater.  

\subsection{}
Now that I think about it, the type of contributor that receives increasing marginal benefits from increasing production effort isn't going to be the type of contributor that would care little about reducing production effort by a lot. Thus, the "above" case is unlikely to happen (\textcolor{red}{should formalize what the above case means}). 

As such, we should expect that production decreases far less when the cost of knowledge acquisition increases and production has increasing marginal value. My hypothesis is also that $P(t_p^r)$ for $P$ with increasing marginal value is less than that of $P(t_p^r) = P t_p^r$, and so consequently, knowledge acquisition decreases more than it does in the linear case. My reasoning is that because they care about production more, and production/knowledge are complements and knowledge is adversely affected by increases in $\alpha_k$, and because they care more, they'll be more adversely affected. Thus, they experience greater (relative) loss in knowledge and less (relative) loss in production. 
\subsection{}
\textbf{Differences in $t_p^r$ functional form}
Let us consider an example where the marginal value of $t_p^r$ is increasing in $t_p^r$. 

- Set up the scenario
- Interested in understanding whether knowledge decreases more or less, and whether production decreases more or less
1) When knowledge acquisition costs increase, knowledge decreases in both cases, no matter what. Basic economic principle - when costs increase, consumption decreases
2) How much it decreases depends on how production levels respond. Since $P' e^{t_p^r} = P t_p^r$, if production didn't respond to knowledge, the two cases have identical marginal benefit curves
3) The thing is that production does respond to knowledge because production and knowledge are complements. Consequently, decreases in knowledge also decrease production. What we're interested in is the magnitude of the decrease when $P(t_p^r) = P' e^{t_p^r}$ vs. $P(t_p^r) = P t_p^r$. 

Suppose $P' e^{t_p^r}$ is above $P t_p^r$; that is, for all $1<t < t_p^r$, $P' e^t > Pt$. Then, the individual with $P(t') = P' e^{t'}$ can attain the same benefit from production as someone with $P(t'') = P t''$ at much lower levels of production effort so that $t' < t''$. \textcolor{red}{But now the problem is I don't know whether $P' e^{t'} \gtrless P t''$ even if I know $t' < t''$. The same is true with the opposite in the below curve case}

\textcolor{red}{The intuition is helpful but I think I need to do the math to actually show what's going on. Using a marginally decreasing functional form is similar - it's pinned down by the above/below curve case}




\textbf{Then, I think we can describe how this would change with $u'', u'''$ that are non zero}\\
First of all, I really just want to assume away the $u''$ term. I can assume quadratic utility, or I can assume that $t_p^r$ is sufficiently large (which I prefer). 

\end{document}