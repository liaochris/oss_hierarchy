\documentclass[source/paper/main.tex]{subfiles}
\begin{document}
\textbf{Outline} from \href{https://github.com/liaochris/oss_hierarchy/issues/2#issuecomment-2589020215}{github}
\begin{enumerate}
    \item Identifying departure
        \begin{enumerate}
            \item Defending endogeneity of departure
        \end{enumerate}
    \item Applying causal forest to event studies
    \begin{enumerate}
        \item Describe procedure
        \item How we identify heterogeneity
    \end{enumerate}
\end{enumerate}

\subsection{Identifying departure}

\subsection{Applying causal forest to event studies}
Literature
\begin{itemize}
    \item \textbf{Empirical: } \cite{iyengar_impact_2022}, \cite{cui_tax-induced_2022}, \cite{guo_effect_2021}
    \item \textbf{Theoretical:} \href{https://pure.au.dk/ws/portalfiles/portal/198663023/Nicolaj_Mu_hlbach_PhD_dissertation.pdf}{See Section 2.3.3 for causal forest event study}, \href{https://arxiv.org/pdf/1610.01271}{GRF}
\end{itemize}

Note that this procedure would be very simple if I used a diff-in-diff with one pre-period and one treatment period because the moment condition would be fairly straightforward (just solve the diff-in-diff equation??) and the transformation of the outcomes/treatment would just be differences. 

\subsubsection{Goal}
Each project $i$ is observed at each time period $t$ and we observe outcomes $Y_{it}$, treatment status $D_{it}$ and covariates $X_{it} \in \mathcal{X}$. If project $i$ is treated at time $g$, then $i \in G_g$. If project $i$ is never treated, then $i \in C$. Define $N_C = |C|, N_g = |G_g|$

Our goal is to estimate $ATT_{k}(x)$, the treatment effect after $k$ periods. Formally, 
\begin{align}
    ATT_{k}(x) = E[ATT_{g,t}(x) \mid x, t-g = k] \text{ for all } x \in \mathcal{X}
\end{align}
I first show how we estimate $ATT_{g,t}(x)$ for all valid $(g, t)$. Attaining a point estimate $\widehat{ATT}_{k}(x)$ can be done by calculating a simple weighted average. Inference on $\widehat{ATT}_{k}(x)$ can be done in two ways. One approach is to use the multivariate delta method (which I will not do because the distributions of the different components of $\widehat{ATT}_{k}(x)$ are correlated since they use the same set of controls). Another approach is to use the bayesian bootstrap (which sounds very computationally intensive so perhaps I will use the delta method).

We know by the definition of the conditional ATT the relevant moment condition is
\begin{align}
    E[Y_{i,t}(1, x) - Y_{i,t}(0, x) - ATT_{g,t}(x) \mid x, i \in G_g] = 0
\end{align}
where $Y_{i,t}(T_{i,t}, x)$ is the potential outcome notation for an individual $i$ in treatment group $G_g$ at time $t$ with treatment $T_{i,t}$. Definitionally, $T_{i,t} = 1$ if $g \geq t$. Now, for some diff-in-diff magic. Assuming $t\geq g$, then \textcolor{red}{Write out conditional parallel trends assumption}
\begin{align*}
    &E[Y_{i,t}(1, x) - Y_{i,t}(0, x)  \mid x, i \in G_g]\\
    &= E[Y_{i,t}(1, x) \mid x] - E[Y_{i,t}(0, x) \mid x, i \in G_g]  \\
    &= E[Y_{i,t}(x) \mid x, i \in G_g] - E[Y_{i,t}(0, x) \mid x, i \in G_g] \\
    &= E[Y_{i,t}(x) \mid x, i \in G_g] - (E[Y_{i,g-1}(0, x) \mid x, i \in G_g] + E[Y_{i',t}(0, x) - Y_{i',g-1}(0, x) \mid x, i' \in C] )\\
    &\text{by the parallel trends assumption, using never-treated as a control}\\
    &= E[Y_{i,t}(x) - Y_{i,g-1}(x) \mid x, i \in G_g] - E[Y_{i',t}(x) - Y_{i',g-1}(x) \mid x, i' \in C] )
\end{align*}
Moreover, note that $\E[ATT(x) \mid x] = ATT(x)$
Following the framework of GRF, we can say that
\begin{align*}
    \widehat{ATT}_{g,t}(x) \text{ solves } \frac{1}{N_g}  \sum_{i \in G_g} \alpha_i(X_{i, g-1}) \left(Y_{i,t} - Y_{i,g-1} -  \frac{1}{N_C} \sum_{i' \in C} (Y_{i',t} - Y_{i',g-1})\right) = \widehat{ATT}(x)
\end{align*}
Note that the causal forest-determined weights $\alpha_i(X_{i, g-1})$ use, for the treated group, the last pre-period covariates to determine the weights. Those are the covariates we will be calculating CATEs with respect to. \textcolor{red}{How can I connect the moment more clearly to the estimation procedure?}

To calculate CATEs, we require that the outcome $Y$ fed into the causal forest is independent of treatment $T$ conditional on covariates $X$. Conditional parallel trends tells us that
\begin{align}
    Y_{i, t} - Y_{i, g-1} \independent T_{i,t} \mid X_{i, g-1}
\end{align}
Thus, differencing our outcome wrt to the time $g-1$ outcome will allow us to properly estimate our parameter of interest. 

Our currently estimated parameters are $\delta_k$, where $k$ is the time period relative to the treatment. Our goal is to estimate $\delta_k(x)$ for some $x \in \mathcal{X}$, which is the heterogeneous treatment effect. Now, define a moment equation $\psi $ such that
\begin{align}
    E[\psi_{\delta_k(x)}(( \boldsymbol{Y_i, D_i})) \mid X_i = x] = 0, \forall x \in \mathcal{X} 
\end{align}
is equivalent to identifying $\delta_k(x)$. \textbf{I believe} that the moment equation will instruct us how to transform \textbf{$Y_i, D_i$} such that the inputs into the causal forest $g_{y,t}(Y_i), g_{d,t}(D_i)$ are conditionally exogenous; that is
\begin{align}
    g_{y,t}(Y_i) \independent g_{d,t}(D_i) \mid X_{it} 
\end{align}
This should be the same as estimating the event-study specification that also produces $\delta_k$ with special project-time weights. Once I have the moment $\psi$ and transformations $g$ defined, I should think about whether I need to change how my controls enter. 

Once I have all the $\delta_k(x)$'s, I should aggregate them to $\delta_k$ but make sure I'm using a doubly robust estimator? (\href{https://grf-labs.github.io/grf/reference/get_scores.causal_forest.html}{DR from GRF guide, but not sure if this is correct}) and Chernozhukov et. al 2018 might be helpful

Here's a few cool things about this
\begin{enumerate}
    \item Following the procedure above means I can adapt use one causal forest routine (for each $k$). This is not something I see other papers do, and actually is quite important because connecting your moment condition (which relies on the conditional exogeneity assumption) to the inputs into the causal forest ensures that you properly estimate quantities
    \item Consistency and asymptotics should immediately follow from the generalized random forest paper
    \item This \textbf{should} allow me to use time-varying controls, whch is not something I've seen before in applications of causal forest to event studies
\end{enumerate}

\subsubsection{Next Steps}
\begin{enumerate}
    \item \textbf{Moment condition}: Which event study estimator am I going to use (Abraham and Sun, Shapiro), and how do I transform that into my moment condition and the transformation of the treatment + outcome. \textbf{I should probably test Abraham and Sun on my current event studies before deciding}
    \item \textbf{Data collection}: Obtain all covariates, divide into hierarchy measures and controls
    \item \textbf{Code}: Write out what code would be \textbf{supposing I had all my variables}
    \item \textbf{Paper}: Outline the methods section in detail (Here's what we do, this is why we do it)
\end{enumerate}

\end{document}