\documentclass[source/paper/main.tex]{subfiles}
\begin{document}

\textbf{Paragraphs 1-2: Motivation. After reading these paragraphs a reader in any field of economics should believe that if you answer your research question your paper will make an important contribution.}

\begin{enumerate}
    \item OSS is economically important. Cite usage + value statistic
    \item A major problem encountered by OSS is developer turnover. Cite project damage statistic 
    %Andreas Schilling, Sven Laumer, and Tim Weitzel. Who will remain? an evaluation of actual person-job and person-team fit to predict developer retention in FLOSS projects. In HICCS, pages 3446–3455. IEEE, 2012.
    \item Development of an OSS project can be conceptualized as the strategy adopted by an organization. Economics tells us that an organization's structure affects its strategy and outcomes. This suggests that we can say something interesting about how organizational structure allows organizations to be more resilient against a departure
\end{enumerate}

\textbf{Paragraphs 3-4: Challenges. These paragraphs explain why your research question has not already been answered, i.e., what are the central challenges a researcher must tackle to answer this question.}

Empirical challenges
\begin{enumerate}
    \item Exogenous departures
    \item Identifying combinations of organizational structures that have an effect 
    \item Understanding the mechanisms by which these organizational structures affect OSS-development
    \item Relating short-term outcomes to long-term OSS-development outcomes
\end{enumerate}
Current work: Current literature does not consider the complementarity of organizational structures. Research on impact of organizational structures only provides suggestive evidence on the mechanism by which organizational structures have impacts. The focus of research has been on codebase related outcomes or "project survival" as opposed to more relevant outcomes such as usage. 

Model of hierarchy that tells us (some subset of the below) how these organizational features affect project outcomes
\begin{enumerate}
    \item Degree of overlap between individuals who participate in several layers of problem solving on the hierarchy (existing theory assumes one person is on one layer)
    \item How work is split between individuals in a layer of problem solving (existing theory assumes each person does the same amount of work)
    \item Degree of cooperation, such as \# of people you work with on your layer (existing theory assumes no cooperation within layer)
    \item Extent of communication, such as the \# of people you communicate with below/above your layer of problem solving
    \item How individuals are “promoted” to higher layers of problem solving.
\end{enumerate}
Existing theory tells us how organizations adapt their organizational structure and strategy in response to external shocks. However, it makes certain assumptions that prevents us from learning about how that adaption depends on existing organizational structure. One example of a puzzle is that we don't know how "degree of overlap" affects response and subsequent outcomes - does it have a positive effect (because they share more knowledge) or negative effect (because their departure affects more areas of problem-solving)?

\textbf{Paragraph 5: This Paper. This paragraph states in a nutshell what the paper accomplishes and how. }

Goals
\begin{enumerate}
    \item Identify key groups of organizational structures that in conjunction affect 
    \item Describe underlying mechanism of the organizational structure, which we show using novel microdata on OSS development from GitHub
    \item Effects on 1) more basic problem solving measures and 2) long-term project outcomes
    \item Combine this to produce revised theory of hierarchy that tells us how organizational adaption depends on existing organizational structure 
\end{enumerate}


\textbf{Paragraphs 6-7: Model. Summarize the key formal assumptions you will maintain in your analysis.}
Don't know yet, if this is about the economic model

\textbf{Paragraphs 8-9: Data. Explain where you obtain your data and how you measure the concepts that are central to your study.}

\begin{itemize}
    \item Data: GitHub
    \item Setting: Major Python projects, from 2015 onwards
    \item Define major contributor
    \item Define departure
    \item Define problem-solving outcomes and downstream outcomes. If necessary, define microdata that allows us to validate impact on problem-solving outcomes. 
    \item Define how I measure organizational structures
\end{itemize}

\textbf{Paragraphs 10-11: Methods. Explain how you take your model to the data and how you overcome the challenges you raised in paragraphs 3-4.}
Key Challenges
\begin{enumerate}
    \item Endogeneity between effect of organizational structure and departure: causal forest constructs control (to limit selection on observables), \textcolor{red}{How does "rapid" departure suggest that departures are not endogenous to unobservables? } 
    \item Combinations: use causal forest
    \item Mechanisms: microdata + \textcolor{red}{fill in how I do this}
    \item Short to long-term outcomes: \textcolor{red}{fill in how I do this}
\end{enumerate}

\textbf{Paragraphs 12-13: Findings. Describe the key findings. Make sure they connect clearly to the motivation in paragraphs 1-2.}
A few interesting organizational structure effect combinations, such as
\begin{itemize}
    \item Increased overlap is bad, except when individuals communicate and cooperate a lot
    \item Negative effects of unequal work split are mitigated by cooperation, which suggests knowledge diffusion
    \item Rapid promotion has less negative effects when it occurs during departure because problem solving benefit outweighs cost of increased mistake rate
\end{itemize}

\textbf{Paragraphs 14-15: Literature. Lay out the two main ways your paper contributes to the literature. Each paragraph should center around one contribution and should explain precisely how your paper differs from the most closely related recent work.}

\begin{enumerate}
    \item Empirics: large scale analysis of departure's impact and how it varies across orgs
    \item Model: how orgs respond when constrained
\end{enumerate}
\end{document}