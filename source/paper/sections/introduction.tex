\documentclass[source/paper/main.tex]{subfiles}
\begin{document}

\textbf{Paragraphs 1-2: Motivation. After reading these paragraphs a reader in any field of economics should believe that if you answer your research question your paper will make an important contribution.}

\begin{enumerate}
    \item OSS is economically important. Cite usage + value statistic
    \item A major problem encountered by OSS is developer turnover. Cite project damage statistic 
    %Andreas Schilling, Sven Laumer, and Tim Weitzel. Who will remain? an evaluation of actual person-job and person-team fit to predict developer retention in FLOSS projects. In HICCS, pages 3446–3455. IEEE, 2012.
    \item Development of an OSS project can be conceptualized as the strategy adopted by an organization. Economics tells us that an organization's structure affects its strategy and outcomes. This suggests that we can say something interesting about how organizational structure allows organizations to be more resilient against a departure
\end{enumerate}

\textbf{old text, may or may not be helpful} \\
The development of open source software (OSS), software that can be accessed, modified and redistributed without additional cost, is an \$8.8 trillion industry (\cite{hoffmann_value_2024}). Open source software is a component of 70-90\% of all software used and is an integral component in the technological workflow of businesses across a wide range of industries, from the tech sector to auto and retail (\cite{nagle_open_2017}). 


Since OSS is distributed for free, organizations that develop OSS cannot generate revenue from OSS by selling software licenses. 


A major problem encountered by these organizations is developer turnover; they rely on developers who join the organization to help contribute to the creation and maintenance of OSS. Developers join for a myriad of reasons; while some are formally employed by the organization 


Instead, OSS developers help contribute to the creation and maintenance of OSS for a myriad of reasons, such as personal software use, as part of their employment, for their own personal enjoyment, or to develop or signal their programming ability. 


many OSS developers are volunteers whose ability to contribute is much more variable than someone whose job is to develop OSS. Departures are caused by a variety of reasons, such as major life changes that reduce free time or a loss of motivation. Moreover, developer turnover may even be a problem for OSS projects maintained by companies that employ developers to work on them, given the high turnover rates in the technology industry. 

Developer turnover negatively affects a project's survival likelihood and code quality; approximately 80\% of all OSS project failures can be attributed to turnover related problems. Software is a form of economic capital that depreciates over time and when less labor is available to maintain it, the backlog of unanswered user questions and unreviewed code improvements grows. Even onboarding new developers may become more difficult if the work done by the contributor they're taking over for is not clearly documented and organized. 

Open source software development can be modeled using economic models of knowledge hierarchies, but these tell us how organizations should optimally adapt in the presence of negative shocks, not how an organization's outcomes or adaptation ability depends on its organizational structure. There is a substantial literature in information systems on developer turnover in OSS and  the existing work on mitigating the impact of developer turnover provides useful case studies for a few projects, but is largely descriptive and limited in scale and scope of analysis. 


\textbf{Paragraphs 3-4: Challenges. These paragraphs explain why your research question has not already been answered, i.e., what are the central challenges a researcher must tackle to answer this question.}
\textcolor{red}{I feel like these aren't really challenges...}
Understanding how an OSS project's organizational structure contributes to resilience in the face of developer turnover is a two-step process. First, a set of important contributors to OSS projects and the subset of those contributors that ultimately depart the project must be identified. To ensure the causal validity of my analysis, it is crucial that the specific timing of the contributor's departure is quasi-random and unrelated to organizational structure characteristics. For example, departures caused by external factors unrelated to the project such as changing jobs or the sudden loss of free time are more likely to have quasi-random timing. The second step is to identify causally what key organizational structures distinguish resilient from less resilient organizations. Collecting microdata on OSS contributions can help show the direct pathway by which particular organizational characteristics cause resilience. 

\qquad The described empirical analysis will help inform an economic model of organizational hierarchy that describes how OSS projects respond to negative shocks like contributor departures. The existing economic model of hierarchy can inform us how organizations redistribute employees to new roles when faced with negative shocks such as developer turnover and a decrease in the accessibility of knowledge. However, an open puzzle is how the existing organizational structure of an OSS project might have differential effects on problem solving ability when projects encounter greater barriers to redistributing contributors. This better captures the reality of OSS development, as some OSS contributors are volunteers who cannot be assigned a mandate and even firms may encounter hiring or promotion frictions when reorganizing teams in the short-term.


\qquad An economic analysis done using the framework above will allow us to understand why some OSS projects persist and even thrive despite the departure of major contributors, while others slowly fade into irrelevance or end up abandoned. Given the global economic value of OSS, it is important for developers to understand how they can protect their organization, and for users of OSS to understand what types of organizations they can rely on to produce the software solutions they need. 

\textbf{Paragraph 5: This Paper. This paragraph states in a nutshell what the paper accomplishes and how. }

In this paper, I analyze the empirical effect of contributor departure on OSS development and how projects are affected differentially depending on their organizational structure. I do so using novel microdata on OSS development from GitHub, the world's largest OSS hosting platform. I then use these insights to develop a formal economic model that can describe more broadly, how hierarchical organizations that face frictions to reallocation respond to the departure of key employees. 

\textbf{Paragraphs 6-7: Model. Summarize the key formal assumptions you will maintain in your analysis.}
Don't know yet, if this is about the economic model

\textbf{Paragraphs 8-9: Data. Explain where you obtain your data and how you measure the concepts that are central to your study.}
\begin{itemize}
    \item Data: GitHub
    \item Setting: Major Python projects, from 2015 onwards
    \item Major Contributor: 18 months (3 6-month periods) where they are in the 75th percentile or above of contributors
    \item Departure: Within 6 months of no longer being a "major contributor", their contributions are 0 (forever)
    \item I study a variety of outcomes; outcomes that relate to a projects problem solving ability directly (how many questions they answer, code changes they review) and consumer-side outcomes (downloads) 
\end{itemize}

\textbf{Paragraphs 10-11: Methods. Explain how you take your model to the data and how you overcome the challenges you raised in paragraphs 3-4.}
What are the confounders and how do I deal with them? 

\textbf{Paragraphs 12-13: Findings. Describe the key findings. Make sure they connect clearly to the motivation in paragraphs 1-2.}


\textbf{Paragraphs 14-15: Literature. Lay out the two main ways your paper contributes to the literature. Each paragraph should center around one contribution and should explain precisely how your paper differs from the most closely related recent work.}
\begin{enumerate}
    \item Empirics: large scale analysis of departure's impact and how it varies across orgs
    \item Model: how orgs respond when constrained
\end{enumerate}
\end{document}