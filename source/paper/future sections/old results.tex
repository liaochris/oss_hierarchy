
\subsection{}
Rsults
\subsubsection{}
\begin{enumerate}
    \item First, I show how effects vary when we consider how "clustered" a project is. 
    \todo[inline]{Discuss the specifics of overlap definition, incorporate it into the data and methods section}
    \begin{itemize}
        \item When considering proportions: In the short term, highly clustered projects are no different from less clustered projects. However, in the long-term, highly clustered projects do better.
        \item When considering values: The long-term negative effect is equally unclear in both cases. However, there are two things I'm noticing. 
        \begin{itemize}
            \item Highly clustered projects do better relative to their baseline mean but the magnitude of their decrease (in count terms) is larger. This suggests that these are larger projects
            \item If the proportional trend is decreasing but the count trend is constant, that suggests that the projects that are most affected are the smallest ones
        \end{itemize}
    \end{itemize}
    \item What explains this? Note that up to so far, our analysis of project-level outcomes means we may be confounding the departure's effect on others with the departure's direct effect (through loss of activity) is to examine how average activity and activity participant count evolves. 
    \begin{itemize}
        \item In relative proportions: Highly clustered project contributors experience smaller declines in average activity over the short and long-term. Moreover, they experience smaller long-term declines in activity participant count. 
        \item In count proportions: When we consider average activity, we see similar trends (but not with activity participant count). We also see strong evidence of declines in count, although it's still not statistically significant.    
    \end{itemize}
    \item One outstanding issue is that  pre-trend coefficients include contributions by the departed, so we would naturally expect to see declines because of their absence. \begin{itemize}
        \item On the one hand, that they aren't replaced might suggest that their departure was abrupt and unanticipated. 
        \item On the other hand, using a full sample of contributors that includes the departed to construct aggregate outcomes doesn't tell us that much about how everyone else is affected 
    \end{itemize}
\end{enumerate}
\todo[inline]{hidden below is discussion about individual departure overlap}
\thiswillnotshow[inline]{
I think on its own, departed overlap doesn't have that big of an impact. However, it seems like the impact of being important is most negative in projects where there's high overlap. It also seems like projects with high overlap do best in periods prior to departure. There are also some trends wrt to \# of important people, but given that this is a hard metric to explain because of the NA i might pass. 
}

\takeaway[inline]{When a contributor in a less clustered project departs, the project experiences worse outcomes. It seems like everyone else's activity declines more. Is this because they don't know how to continue because they don't communicate? I will need to validate this by excluding the departed contributor though}
\askjesse[inline]{Confirm with him that I really don't want to include averages/\# of people opening PRs analysis with the departed contributor in the sample...}
\begin{enumerate}
    \item Next, I show how effects vary when we consider project clustering, but examine outcomes produced by various contributor subsets
    \todo[inline]{Define how I calculate these outcomes either here or in methods}
    \item First, I consider all contributors except the departed
    \begin{enumerate}
        \item Aggregate activity declines more in the short term when projects are clustered, which is not what we've been observing. In the long term the effects are similar/less clustered projects do worse, or the unclustered project does worse, which is what we previously observed.  What is this driven by?
        \begin{itemize}
            \item In the short run, average activity and activity participant count decline more in highly clustered projects
            \item In the long run, average activity and activity participant count decline more in less clustered projects
        \end{itemize}
    \takeaway[inline]{We have to separate the impact of the contributor departing from the activities of others. The contributor departing has a stronger effect in less clustered projects, whereas the contributor's effect on other contributors has a stronger effect in clustered projects. }
    \end{enumerate}
    \item I then consider all contributors who were present when the departed was there. Note that we might naturally expect a decline in activity due to attrition, but comparing relative rates will be informative. 
    \begin{enumerate}
        \item The observed decline in proportion of PRs opened by other contributors is much smaller. We see the same small effect when subsetting to just important or unimportant contributors present prior to departure. 
        \item What's noticeable (in the all subset) is that the trends over time described in the proportion data match up with the count data, which suggests that in projects of all sizes, a departure has an effect
        \todo[inline]{is it just the trends over time or also the implied proportions}
    \end{enumerate}
    \takeaway[inline]{In all projects existing contributors both experience declines in activity. However, new contributors contribute much less in clustered projects}
    \todo[inline]{How to visualize this? Do I want each subset on the same plot (2 plots)? Or do I still want subset specific plots, with org structure (4 plots) - ask jesse once I have the plots}
    \longterm[inline]{The detailed analysis might go in the appendix}
\end{enumerate}


\subsubsection{}
\begin{enumerate}
    \item Next, I explore how clustering interacts with departed contributor importance. 
    \item Although on it's own, clustering may not explain how projects are affected differently by organizational structure. However, it's likely that clustering is confounded with contributor importance; as we saw earlier, less clustered projects were more affected directly by the contributor's departure. 
    \begin{enumerate}
        \item We can see that less clustered projects have a higher distribution of departed contributors who were important.
        \todo[inline]{Add statistic visualizing? Maybe a table }
        \item When considering proportions: When the departed contributor is important, highly clustered projects are more negatively affected in both the short and long-term. Clustering also reduces the negative effect of less important contributors. 
        There's one confusing result - in unclustered projects, there's no or a slight improvement in outcomes when the departed is more important.  
        \item When considering counts: When the departed contributor was important, there is a much clearer and negative effect of departures on outcomes. This is likely because projects of all sizes in the bin are affected by important departures. 
        \todo[inline]{check the implied versus actual proportion}
    \end{enumerate}
    \item When I subset to examine the aggregate activity of all other contributors, I observe that
    \begin{enumerate}
        \item As before, when the departed contributor is important, the activity of other contributors in highly clustered projects are more negatively affected in both the short and long-term. 
        \item As before, but only in the long-term, clustering reduces the negative impact of less important departures. 
        \item Unclustered projects that undergo less important departures are more negatively affected because of aggregate activity declines by everyone else. 
        \end{enumerate}
    \item When examining average activity and activity participation, I see that
    \begin{enumerate}
        \item Worse aggregate outcomes for clustered vs. unclustered (fixing importance) is driven by greater decreases in average activity and activity participation in the short-term for clustered projects. 
        \item Worse aggregate outcomes for unclustered vs. clustered (fixing less importance) is driven by greater decreases in average activity and activity participation in the long-term for unclustered projects. 
    \end{enumerate}
    \item When examining all contributors present prior to departure, I see that
    \begin{enumerate}
        \item We now see a substantial difference in outcomes between less and more important departures in clustered projects. The same is true when you compare clustered and unclustered projects that experience less important departures. What explains this in the short term is less activity participant declines; what explains this in the long run is higher average activity increases. Interestingly, in the short term, we don't see differences in how clustering affects the activity of all contributors when someone important departs. 
    \end{enumerate}
\end{enumerate}
\takeaway[inline]{Less important departures in clustering have less negative effects, and more important projects in clusters have no worse effects?}
\textbf{What do I need to see in order to be more convinced that there's an effect. I'm not really convinced there is an effect right now}


\subsection{}
Causal validation 


1) The difference in effect is driven by two things. First, short term decline in prs opened in clustered projects, irrespective of departure importance is the same. Same with the short decline in the average activity. Second, after 1.5 years, further decline in people opening PRs and average, which causes the long and short term effects to diverge. 

\textbf{Not immediately clear how the departure affects # of people participating/their activity levels down the road in 1.5 years - will think more about this?}

Another thing to keep in mind is that we haven't considered individual departed contributor's impact on outcome yet. 

\textbf{all time analysis}: \\
When we look at the impact on everyone else, we see that important projects are much more affected, which is interesting - recall this is the activity of everyone els and while the outcome improves slightly in the short term the long term effect is the same as the original short term. 
note that this suggests that even outside of the fact that more important contributors might have larger impacts on PRs opened, their departures have larger repercussions on their fellow contributors
The difference in activity in the short-term is driven by a larger decrease in average activity. In the short-term there's minimal difference in departures. In the long-term there is difference in departures. 


\textbf{all (already there)}: \\
It might make sense for the departed to have the most impact on people who were already there
\iffalse
\textbf{old stuff}
I need to figure out what my results actually are first...
- Weak evidence that higher individual overlap is bad (prs_opened, prs_merged), especially in the short term. I'm looking at outcomes

- Stronger evidence that high overlap is good. This is particularly evident when you consider the top third vs. everyone else. 
  - \textbf{Notes on how high/middle/low overla projects differ}: Projects with high overlap are bigger, have more important individuals, and each individual is less important. Also, there's probably a case to excluding the bottom third because they're just too different. 
- Given that there's pretty clear differences in importance between high vs. overlap projects, and intuitively, importance seems like something important to account for, I account for importance next in the event study

- When I account for importance, I find that
  - important departures in high overlap projects are very harmful. surprisingly important departures in low overlap projects aren't harmful
    - High overlap projects have higher "prop. of imp people", the departed tend to be less important which is all counterintuitive (if we're thinking about magnitude of departure impact)
    - The two things distinguishing high from low overlap projects (with important departures) that might be important
      - departed tends to communicate much more with other important contributors
      - departed is much more integrated 
      - there's more activity to begin with 
    - Both projects tend to be smaller than projects with unimportant departures
  - The results are robust when I examine average activity as well, where average is determined across all contributors who engage in that activity, as well as contributor count in each activity group. I will note that the key difference between the middle and high overlap project in contributor count is that the latter experiences a permanent decrease, whereas the former experiences only a temporary one period decrease. It's kind of a double whammy - less people and average decreased = way less activity... 
    
  - unimportant departures in low overlap projects are very harmful, aren't harmful in high overlap projects. 
    - I also want to look into this but perhaps more down the road. 

  - The key things that counteract departures in small overlap big importance projects is that
    - everyone ups their average activity post-departure
    - new people have slightly more positive effects
    - More driven by unimp peole than imp people (less hierarchical suggests more opportunity for growth???) 


  - I think I want to 
    - see the distribution of importance \textbf{done}
    - see the distribution of clustering \textbf{done}
    - learn more about the characteristics of the projects in each bin \textbf{done}
    - how average activity changes \textbf{done}
      - (also subset by imp/unimp)
    - change in the number of people in the project \textbf{done}
      - (also subset by imp/unimp)
    - Examine how the activity of all other contributors changes post-departure
        
  - I think there's a story about how it's really the importnt contributors in tihgtly integrated projects that cause the most damage. This is interesting and I dno't know why... perhaps because they're the leader??? 


- Validate causal effects by
  - Ruling out obvious hypotheses
  - Testing hypotheses implied by heterogeneous treatments


The first causal effect I want to validate: 
Why are projects with low modularity (those with overlapping responsibilities among key contributors) less adversely affected?

Some potential hypotheses that I'd be concerned about
- Is this because in low modularity projects, departed contributors are less important? \textcolor{blue}{Using the same threshold for "importance", in low modularity projects, a higher proportion of departures (62.5\%) are by important individuals compared to in high modularity rojects (37.5\%). Departures by important individuals have equally sized effects in irrespective of modularity; it seems like it might even have a slightly larger effect in highly modular projects. }
  - If so, can I control for importance somehow? And does the effect still hold? 
  - If I look at the activity of everyone else (who was already there), does it differ between the projects?
    - Can also subsample by other important vs. unimportant
  - As a follow up to the above, does activity by new entrants differ?
- Is everyone leaving in high modularity projects?
  - How does average activity (for everyone, important and other contributors) change?
  - What proportion of existing (all, important and other contributors) leave and never return
    - What does entrance look like?
  - What does quantity of contributors in each group look like?
Downstream Hypotheses
- In low modularity projects, is the leftover work of departed important contributors more likely to be handled by other important contributors (controlling for size)
- In low modularity projects, are other important contributors more likely to (continue interacting, start interacting) with people that the departed contributor (and them, but not them) interacted with?

The second causal effect I want to validate: 
The extent of communication between the departed contributor and other project members also causes projects to experience differential effects. Projects where the departed contributor engaged in limited communication with other project members (key and otherwise) were disproportionately affected. The same is true for projects where the departed contributor engaged in extensive communication with other project members (key and otherwise). These adverse effects are particularly prescient in projects with high modularity although this final example of heterogeneity is only observed for GitHub-related activity. 

1) I think the clues for high imp->imp, imp->other communication will follow from the previous execises. My strategy will think about how low imp->imp and low imp->other relates high imp->imp or high imp->other. 
- I think the previous ones will be super revealing, nothing in particular here for now. 

- If little communication is really harmful, then why do we expect others to also have their activity drop off as well? A little confusing...
  - Are there unfinished PRs/issues that remain unfinished?
  - If the organization is falling apart, how is low communication related to that?

  
What is it about both being high that harms projects? The concern is that high communication is just an importance measure
- Testing implied hypotheses
  - The interesting hypothesis is that people are just listening, or are they learning? Before I test this, I'd want to see that the departed is involved in threads (or other actions) that could potentially indicate communication actually has an effect. It would also be good to learn what types of threads these are. 
\fi