\documentclass[12pt,notitlepage]{article}

% --------------------------------------------------
% Core math & symbols
% --------------------------------------------------
\usepackage{amssymb, amsmath, amsthm, bm, mathtools}
\usepackage{dsfont, bbm}

% --------------------------------------------------
% Figures & tables
% --------------------------------------------------
\usepackage{graphicx}
\usepackage{epstopdf}
\usepackage{float}
\usepackage{booktabs}
\usepackage{tabularx, longtable, array, multirow, diagbox}
\usepackage{arydshln}
\usepackage{subcaption}
\captionsetup[sub]{subrefformat=parens}
\DeclareCaptionLabelFormat{subpanel}{Panel~(#2):}
\captionsetup[sub]{labelformat=subpanel, labelsep=space}
\usepackage{pdflscape}

% required by modelsummary
\usepackage{tabularray}
\UseTblrLibrary{booktabs}
\UseTblrLibrary{siunitx}
\newcommand{\tinytableTabularrayUnderline}[1]{\underline{#1}}
\newcommand{\tinytableTabularrayStrikeout}[1]{\sout{#1}}
\NewTableCommand{\tinytableDefineColor}[3]{\definecolor{#1}{#2}{#3}}

% --------------------------------------------------
% Formatting & utilities
% --------------------------------------------------
\usepackage{xcolor}
\usepackage[colorinlistoftodos,prependcaption,textsize=small]{todonotes}
\usepackage{xargs}
\usepackage{setspace}
\usepackage{epigraph}
\usepackage{textcomp}
\usepackage{verbatim}
\usepackage{scalerel, stackengine}
\usepackage[framemethod=tikz]{mdframed}
\usepackage[hyphens]{url}
\usepackage[colorlinks,allcolors=blue]{hyperref}
\usepackage{cleveref}
\usepackage[shortlabels]{enumitem}
\usepackage{subfiles} % Best loaded last in the preamble
\usepackage[normalem]{ulem}
\usepackage[authoryear]{natbib}

% --------------------------------------------------
% Page setup
% --------------------------------------------------
\setlength{\marginparwidth}{2cm}
\setlength{\epigraphrule}{0pt}
\renewcommand{\baselinestretch}{1.25}
\topmargin=-1.5cm \textheight=23cm \oddsidemargin=0.5cm
\evensidemargin=0.5cm \textwidth=15.5cm
\onehalfspacing
\setcounter{MaxMatrixCols}{10}

% --------------------------------------------------
% Columns
% --------------------------------------------------
\newcolumntype{L}[1]{>{\raggedright\let\newline\\arraybackslash\hspace{0pt}}m{#1}}
\newcolumntype{C}[1]{>{\centering\let\newline\\arraybackslash\hspace{0pt}}m{#1}}
\newcolumntype{R}[1]{>{\raggedleft\let\newline\\arraybackslash\hspace{0pt}}m{#1}}

% --------------------------------------------------
% Custom commands
% --------------------------------------------------
\newcommand{\I}{\mathbb{I}}
\newcommand{\E}{\mathbb{E}}
\newcommand{\R}{\mathbb{R}}
\newcommand{\Prob}{\mathbb{P}}
\newcommand{\Var}{\mathrm{Var}}
\newcommand{\Cov}{\mathrm{Cov}}
\newcommand{\Corr}{\mathrm{Corr}}
\newcommand{\Bias}{\mathrm{Bias}}
\newcommand{\MSE}{\mathrm{MSE}}
\newcommand{\supp}{\mathrm{supp}}
\newcommand{\notimplies}{\mathrel{{\ooalign{\hidewidth$\not\phantom{=}$\hidewidth\cr$\implies$}}}}
\newcommand\dapprox{\stackrel{\mathclap{\tiny \mbox{d}}}{\approx}}
\newcommand\papprox{\stackrel{\mathclap{\tiny \mbox{p}}}{\approx}}
\newcommand\pconverge{\stackrel{\mathclap{\tiny \mbox{p}}}{\to}}
\newcommand\dconverge{\stackrel{\mathclap{\tiny \mbox{d}}}{\to}}
\newcommand\independent{\protect\mathpalette{\protect\independenT}{\perp}}
\def\independenT#1#2{\mathrel{\rlap{$#1#2$}\mkern2mu{#1#2}}}

\newcommand{\red}[1]{{\color{red} #1}}
\newcommand{\blue}[1]{{\color{blue} #1}}

\renewcommand{\eqref}[1]{Equation~\ref{#1}}
\newcommand{\assref}[1]{Assumption~\ref{#1}}

% Todos
\let\OldTodo\todo
\RenewDocumentCommand{\todo}{O{} m}{\OldTodo[#1]{\textbf{TODO}: #2}}
\newcommandx{\thiswillnotshow}[2][1=]{\OldTodo[disable,#1]{#2}}
\newcommandx{\askjesse}[2][1=]{\OldTodo[linecolor=Plum,backgroundcolor=Plum!25,bordercolor=Plum,#1]{\textbf{{Ask Jesse:}} #2}}
\newcommandx{\longterm}[2][1=]{\OldTodo[linecolor=Blue,backgroundcolor=Blue!25,bordercolor=Blue,#1]{\textbf{{Long-term:}} #2}}
\newcommandx{\donow}[2][1=]{\OldTodo[linecolor=Green,backgroundcolor=Green!25,bordercolor=Green,#1]{\textbf{{Do Now:}} #2}}

% --------------------------------------------------
% Theorems and assumptions
% --------------------------------------------------
\newtheorem{theorem}{Theorem}
\newtheorem{corollary}[theorem]{Corollary}
\newtheorem{proposition}{Proposition}
\newtheorem{lemma}{Lemma}
\newtheorem{cor}{Corollary}
\newtheorem{conjecture}{Conjecture}
\newtheorem{remark}{Remark}
\newtheorem{assumption}{Assumption}
\newtheorem{definition}{Definition}
\newtheorem{hyp}{Hypothesis}
\newtheorem{subhyp}{Hypothesis}[hyp]
\renewcommand{\thesubhyp}{\thehyp\alph{subhyp}}

% --------------------------------------------------
% Section numbering
% --------------------------------------------------
\renewcommand{\thesubsection}{\arabic{section}.\arabic{subsection}}
\renewcommand{\thesubsubsection}{\arabic{section}.\arabic{subsection}.\arabic{subsubsection}}

% --------------------------------------------------
\begin{document}

\title{Generalized Random Forest Event Studies}

\author{
  Chris Liao\thanks{I am grateful to Jesse Shapiro and Scott Nelson for helpful comments and support. Email: chrisliao@uchicago.edu} \\
}
\date{\today}
\maketitle

\begin{abstract}
\noindent 
This note introduces a procedure for estimating heterogeneous dynamic treatment effects in event studies with staggered adoption, allowing heterogeneity to vary by treatment adoption date and with observation-level time-invariant covariates.
The approach reformulates the problem as estimating a conditional linear model with binary regressors and estimates all cohort–event time conditional treatment effects jointly using a single generalized random forest, with propensity scores adapted to staggered treatment adoption in panel settings. 
\vspace{0in}\\
\end{abstract}

\section{Motivation}
Event studies are widely used to estimate the causal effects of policies over multiple periods. 
A survey by \cite{roth_pretest_2022} found 70 papers in three leading economics journals between 2014 and 2018 that employ event study plots.
Researchers may also be interested in treatment effect heterogeneity within event study designs.
The literature has developed methods to estimate treatment effects that are robust to heterogeneity arising from variation in treatment timing \citep{GOODMANBACON2021254, sun_estimating_2021, callaway_difference--differences_2021, athey_design_2022}, but largely omits potential heterogeneity arising from other sources. 
A separate literature  leverages machine learning methods for heterogeneous treatment effect estimation \citep{wager_estimation_2018, athey_generalized_2019, nie_quasi-oracle_2021, chernozhukov2025generic}, but these methods are primarily designed for cross-sectional settings rather than panel data.
\cite{chernozhukov2019inference} develops estimators of heterogeneous treatment effects that vary across units and over time using a low-rank factor structure but their framework is not designed for causal inference in staggered treatment adoption settings. 

I present a procedure for estimating heterogeneous dynamic treatment effects in staggered treatment adoption event studies with unit-level time-invariant covariates.
I show that this procedure is equivalent to estimating a special case of the conditional linear model with binary regressors and propensity scores adapted for staggered treatment adoption and panel settings. 
Solving the conditional linear model using the generalized random forest from \cite{athey_generalized_2019} is known to yield consistent and unbiased estimates.

I build on existing work combining estimation on event studies and heterogeneous treatment effects using machine learning.
\cite{wang_effect_2022} and \cite{miao_eects_2023}  both embed the causal forest from \cite{athey_generalized_2019} within a differences-in-difference framework but only consider a single pre- and post-treatment period, so their analysis omits treatment effect dynamics. 
\cite{miller2020causal} use causal forests to estimate heterogeneous and time-varying policy effects but rely on conditional exogeneity for identification, whereas my procedure relies on conditional parallel trends. 


The most related work is \cite{gavrilova_difference--difference_2025}.
Both are interested in estimating the effect of unit-level time-invariant covariates on treatment effects in an event-study framework with potentially multiple pre- and post-treatment periods and staggered treatment adoption using generalized random forests \citep{athey_generalized_2019}. 
Generalized random forests estimate conditional average treatment effects using locally weighted averages of observation-level treatment effects, with weights determined by forest-based similarity in covariates.
\cite{gavrilova_difference--difference_2025} allows these weights to vary by covariate values, treatment cohorts, and event time, whereas my approach restricts variation to the covariate value level, as they estimate separate forests for each cohort-event time effect, whereas I only use one forest. 
Permitting weights to only vary by covariate values enhances interpretability, as it preserves a uniform relationship between covariates and treatment effects across cohorts and event times, while still enabling dynamic treatment effect estimation.

\section{Setup}
I observe units $i=1,\dots,n$, and each unit $i$ is observed from time $t=s_i, \cdots, e_i$ where $1 \leq s_i <e_i\leq T$ represent the first and last period $i$ is observed, respectively.  
Observations are assumed independent across units but may be serially correlated within units.
Each unit $i$ is either treated in period $G_i \in \{2, \dots, T-1\} \equiv G$ or never treated ($G_i=\infty$). 
Treatments are irreversible and binary. 
Units are separated into cohorts $G_i = g$ based on treatment timing. 
Outcomes are denoted by $Y_{i,t}$ and $Y_{i,t}^\infty$ is the potential outcome for unit $i$ in period $t$ if it were never treated.

\cite{sun_estimating_2021} define the cohort-specific average treatment effect on the treated (CATT) $k$ periods after treatment. 
For a treated cohort $g \in G$,
\begin{equation}\label{eq:ccatt}
    CATT_{g,k} = \E[Y_{i,g+k} - Y_{i,g+k}^\infty \mid G_i=g].
\end{equation}
$CATT_{g,k}$ captures the average treatment effect at event time $k =  t - G_i$ for units in treatment cohort $g$. 

Define the cohort-event time treatment indicator as
$$Z_{i,t}^{g,k} = \mathbf{1}\{G_i=g\}\cdot \mathbf{1}\{t=g+k\}$$
where $Z_{i,t}^{g,k} = 1$ if and only if unit $i$ belongs to cohort $g$ and at time $t$ is at event time $k$.
I define the reference period as $k=-1$. 
\cite{sun_estimating_2021} show that when the saturated fixed-effects regression below is estimated,
\begin{equation}\label{eq:SA-reg}
Y_{i,t} = \alpha_i + \gamma_t + \sum_{g \in G}\sum_{k \neq -1} \delta_{g,k}\, Z_{i,t}^{g,k} + \varepsilon_{i,t}
\end{equation}
$\delta_{g, k}$ identifies the $CATT_{g, k}$ from \eqref{eq:ccatt}.
$\alpha_i$ and $\gamma_t$ are unit and time effects, respectively.  

For ease of exposition, I define
\begin{itemize}
  \item $\delta = (\delta_{g,k})_{g \in G, \, k \neq -1}$ as the vector of cohort-event time coefficients, with one entry for each cohort $g$ and event time $k$,
  \item $Z_{i,t} = (Z_{i,t}^{g,k})_{g \in G, \, k \neq -1}$ as the corresponding vector of cohort-event time treatment indicators.
\end{itemize}
I can then rewrite \eqref{eq:SA-reg} as
\begin{equation}\label{eq:SA-reg-vec}
Y_{i,t} = \alpha_i + \gamma_t + \langle \delta, Z_{i,t} \rangle + \varepsilon_{i,t}
\end{equation}
\section{Heterogeneous Treatment Effects} \label{sec:het_te}
\subsection{The Model}
Each unit $i$ has a set of time-invariant covariates $X_i \in \mathcal{X}$.
I am interested in how $X_i$ affects the CATT, which motivates my estimand of inference: the conditional CATT, defined as 
\begin{equation}\label{eq:catt_hte}
    CATT_{g,k}(x) = \E[Y_{i,g+k} - Y_{i,g+k}^\infty \mid G_i=g, X_i=x]. 
\end{equation}
This is the cohort-event time equivalent of Equation 1 in \cite{gavrilova_difference--difference_2025}.

The model I will estimate is 
\begin{equation}\label{eq:SA-reg-ml}
Y_{i,t} = \alpha_i + f_t(X_i) 
+ \langle \delta(X_i), Z_{i,t} \rangle + \varepsilon_{i,t}. 
\end{equation}
The estimated equation differs from \eqref{eq:SA-reg-vec} in two ways: 
\begin{enumerate}
    \item I replace time fixed effects with flexible, time-varying functions of $f_t(x)$ that capture baseline heterogeneity in the outcome,
    \item I define conditional cohort-event time coefficients $\delta_{g,k}(x)$ as functions of covariates. Furthermore, define the vector $\delta(X_i) = \big(\delta_{g,k}(X_i)\big)_{g \in G, \, k \neq -1}$ 
\end{enumerate}

I retain the linear fixed effects from \eqref{eq:SA-reg} because my covariates are time-invariant.
See \citealt{johannemann_sufficient_2021} for work on implementing alternative, lower-dimensional representations of categorical variables into generalized random forests. 
\eqref{eq:SA-reg-ml} embeds the possibility of a model with time fixed effects if $f_t(x)$ is constant across all $x$ and Appendix Section~\ref{sec:time_fe} extends derivations to this setting, 

\subsection{Quasi-Treatment Dates and First Differences}
My goal is to rewrite \eqref{eq:SA-reg-ml} in the centered regression form of \cite{robinson_root-n-consistent_1988}.
This requires partialing out the unit fixed effect, so that the (residualized) outcome is a function of the dot product of the (residualized) treatment vector and $\delta(x)$. 

I will first-difference all observations with respect to the reference period to eliminate the unit-level fixed effects. 
Since the notion of a reference period is not well-defined for never-treated observations, to enable first-differencing, I introduce the notion of the \emph{quasi-treatment date}.

Each never-treated unit first appears at time $s_i$. 
Define 
\begin{equation}\label{eq:FGx}
    F_G(g \mid s, x) = 
    \frac{\sum_{j: s_j = s} \mathbbm{1}\{G_j = g,\, G_j < \infty\} K(x_j, x)}
         {\sum_{j: s_j = s} \mathbbm{1}\{G_j < \infty\} K(x_j, x)}
\end{equation}
which is the conditional cumulative distribution function of treatment dates among treated units conditional on first appearance date $s$ and covariates $x$.
$K(\cdot, \cdot)$ is a weighting function over covariates. 
I assign all never-treated units $i$ a quasi-treatment date $Q_i$, drawn from  $F_G(\cdot \mid s_i, x_i)$. 
For treated units, the quasi-treatment date equals the actual treatment date.

I will next show how first-differencing eliminates the unit fixed effect and the resulting first-differenced reformulation of \eqref{eq:SA-reg-ml}.
The first-differenced outcomes, baseline heterogeneity function and error are 
\begin{align*}
   Y_{i,t}^{FD} &= Y_{i,t} - Y_{i,Q_i-1} \\ 
   f_{t}^{FD}(X_i) &= f_{t}(X_i) - f_{Q_i - 1}(X_i)\\
\varepsilon_{i,t}^{FD} &= \varepsilon_{i,t} - \varepsilon_{i,Q_i-1}
\end{align*}
First-differenced treatment indicators are unchanged because the reference period is already omitted. 
The first-differenced form of \eqref{eq:SA-reg-ml} is hence 
\begin{align} \label{eq:SA-reg-ml-fd}
Y_{i,t}^{FD} &= f_t(X_i) - f_{Q_i - 1}(X_i) + \langle \delta(X_i) , Z_{i,t} \rangle + \varepsilon_{i,t}^{FD} \notag \\ 
&= \langle f(X_i) , T_{i,t} \rangle  + \langle \delta(X_i) , Z_{i,t} \rangle + \varepsilon_{i,t}^{FD}.
\end{align}
For ease of exposition,
\begin{itemize}
    \item $f(X_i) = \left(f_s(X_i)\right)_{s = 1}^{T}$ is the vector of baseline heterogeneity functions
    \item $T_{i,t} = \left(T_{i,t}^s\right)_{s = 1}^{T}$ is the vector of indicators, where row $T_{i,t}^s = 1$ if $s=t$, $T_{i,t}^s = -1$ if $s=Q_i - 1$ and $0$ otherwise. 
\end{itemize}
Henceforth, the observation for event time $-1$ is omitted. 

The consequence of my first differencing procedure is that units will only ever be compared to other units with the same quasi-treatment date. 
\cite{gavrilova_difference--difference_2025} also difference the outcome relative to a reference period, but do not need to use a quasi-treatment date. 
This is because \cite{gavrilova_difference--difference_2025} estimate each cohort–event time parameter using a separate forest.
The treated observations for that forest is a single treatment cohort observed at the relevant event time and the control observations are the never-treated cohort observed at that same calendar time. 
Since each forest includes only one treatment cohort, all treated units share a single reference period, allowing outcomes to be differenced relative to that period without assigning quasi-treatment dates. 

\subsection{Identifying the Conditional CATT}
The following assumptions enable $\delta_{g,k}$ to identify the $CATT_{g, k}(x)$. 
The proof is in Appendix Section~\ref{sec:ccatt_identification}.


\begin{assumption}[Conditional Parallel Trends]\label{ass:parallel_trends_cond}
For all time periods $t$, $x \in \mathcal{X}$, and treatment cohorts $g \in \{G \cup \{\infty\}\}$
\[
\E[Y_{i,t}^{\infty} - Y_{i,Q_i - 1}^{\infty}\mid G_i=g, X_i = x] = f_t(X_i) - f_{Q_i - 1}(X_i) 
\]
\end{assumption}
It then follows that $ \E[\epsilon_{i, t}^{FD}  \mid X_i = x] = 0$
\begin{assumption}[Conditional No Anticipation]\label{ass:no_ant_cond}
For all $k<0$ and $x \in \mathcal{X}$,
\[
\E[Y_{i,g+k} \mid X_i = x] = \E[Y_{i,g+k}^\infty \mid X_i = x] 
\quad \text{for all } g \in G.
\]
\end{assumption}
I also require that the overlap assumption and regularity conditions from \cite{athey_generalized_2019} hold. 
\subsection{Centered Regression Form} \label{sec:estimation_framework}
Define the conditional means
\begin{align*}
e(x) &= \E[Z_{i,t}\mid X_i=x] \\
t(x) &= \E[T_{i,t}\mid X_i=x] \\
m(x) &= \E[Y_{i,t}^{FD} \mid X_i=x]  = \langle f(x), t(x)\rangle  + \langle \delta(x), e(x)\rangle 
\end{align*}

Subtracting the conditional means from the model in \eqref{eq:SA-reg-ml-fd} yields the centered regression from \cite{robinson_root-n-consistent_1988}. 
\begin{equation} \label{eq:centered_reg}
Y_{i,t}^{FD} - m(X_i) = \langle f(X_i), T_{i, t} - t(X_i)\rangle  + \langle \delta(X_i), Z_{i, t} - e(X_i)\rangle  + \varepsilon_{i,t}^{FD} 
\end{equation}
Define the centered treatment, baseline heterogeneity assignment and outcome as
\begin{align*}
    \widetilde{Z}_{i,t} &= Z_{i, t} - e(X_i)\\
    \widetilde{T}_{i,t} &= T_{i,t} - t(X_i)\\
    \widetilde{Y}_{i,t}^{FD} &= Y_{i,t}^{FD} - m(X_i)
\end{align*}
The key insight from \cite{robinson_root-n-consistent_1988} and \cite{athey_generalized_2019} is that if there is some neighborhood $S$ where for all $x \in S$, $\delta(x)$ and $f(x)$ are constant, then we can estimate $\delta(x)$ and $f(x)$ by estimating \eqref{eq:centered_reg} on the subsample of observations $j$ where $X_j \in S$.

Hence, \cite{athey_generalized_2019} show that for any covariate values $x$, $\delta(x)$ is identified by the solution to
\begin{equation} \label{eq:ccatt_soltion}
(\delta(x), f(x)) = \Var[(\widetilde{Z}_{i,t} + \widetilde{T}_{i,t}) \mid X_i \in S]^{-1} \Cov[(\widetilde{Z}_{i,t} + \widetilde{T}_{i,t}), \widetilde{Y}_{i,t}^{FD} \mid X_i \in S] 
\end{equation}

\subsection{Estimation Framework}
The finite-sample equivalent to \eqref{eq:ccatt_soltion} is
\begin{align*} \label{eq:catt-x-est}
(\hat{\delta}(x), \hat{f}(x)) = \left(\sum_{i,t} \alpha_i(x) \left( Z_{i, t} - e^{(-i)}(x) + T_{i,t} - t^{(-i)}(x)\right)^{\otimes 2}\right)^{-1} \cdot \\ \sum_{i,t} \alpha_i(x) \left( Z_{i, t} - e^{(-i)}(x) + T_{i,t} - t^{(-i)}(x)\right)\left(Y_{i,t} - m^{(-i)}(x)\right).
\end{align*}
Here, $\alpha_i(x)$ are generalized random forest kernel weights measuring the relevance of observation $i$ for estimating the $\delta(x)$ and are the observation-specific sample analog of the neighborhood $S$ \citep{athey_generalized_2019}.

Since \cite{gavrilova_difference--difference_2025} estimate a separate forest for each cohort–event time estimate, each cohort-event time estimate will rely on upon cohort-event time kernel weights $\alpha_i^{g, k}(x)$.
In the shared setting of this note and \cite{gavrilova_difference--difference_2025}, unit-level covariates are time-invariant. 
Because each $\delta_{g, k}(x)$ is a weighted average of other observations $i$'s treatment effects, the setting of \cite{gavrilova_difference--difference_2025} implies that even for identical covariate values $x$, the same observation $i$ may be weighted differently depending on the treatment cohort or event time of the parameter being calculated. 
Since my approach uses a single forest to estimate all parameters, the weighting structure $\alpha_i(x)$ is shared across all treatment cohort and event time combinations, ensuring that unit $i$'s contribution to the conditional treatment effect remains consistent conditional on covariate value $x$. 
Note that my approach does not restrict treatment effects to be the same across cohort-event times conditional on covariates - it only restricts the weights used in the weighted average of other observations to be the same. 

Suppose that the data is divided into $k$ equally sized folds. 
$e^{(-i)}(\cdot)$, $t^{(-i)}(\cdot)$, and $m^{(-i)}(\cdot)$ are the treatment, baseline heterogeneity and outcome function estimated via $k$-fold cross-fitting, using the $k-1$ folds that omit observation $i$. 
Cross-fitting enables valid inference on $\delta(X_i)$ and $f(X_i)$ even when the estimated outcome and propensity functions have small errors because the conditional means functions are not trained on unit $i$ \citep{chernozhukov_doubledebiased_2018}.

Estimation of the propensity scores $e(x)$ and $t(x)$ requires modifying existing approaches from the literature because of the panel data setting. 
Since treatment is at the cohort-event time level, in a given time period $t$, if an observation belongs to treatment cohort $g$, it is immediately known that for all cohort-event time pairs that $Z_{i,t}^{g, k} = 0$, for $k \neq t-g$. 
A similar phenomenon occurs with $T_{i, t}$.
Appendix Section~\ref{sec:treatment_propensity} describes the mathematical formulation to estimate $e(x)$ and $t(x)$.
The outcome function $m(x)$ can be estimated following standard procedure. 
Since \cite{gavrilova_difference--difference_2025} use separate forests for each cohort-event time estimate, their treatment vectors, and hence propensity scores, do not run into the co-determinism problem above. 

Appendix Section~\ref{sec:doubly_robust} shows how doubly robust scores for $\hat{\delta}(x)$ are estimated.

\clearpage
\bibliographystyle{source/paper/aea}
\bibliography{source/paper/references}


\appendix

\clearpage
\section{The linear time fixed effects model}\label{sec:time_fe}
A special case of Section~\ref{sec:het_te} arises when $f_t(X_i)$ affects outcomes uniformly within each period $t$, implying a time fixed effect. The model from Section~\ref{eq:SA-reg-ml} then simplifies to
\begin{equation*} 
Y_{i,t} = \alpha_i + \gamma_t
+ \langle \delta(X_i), Z_{i,t} \rangle + \varepsilon_{i,t}. \label{eq:SA-reg-ml-fe}
\end{equation*}
with first difference
\begin{equation*}
Y_{i,t}^{FD} = \gamma_t  - \gamma_{Q_i - 1} + \langle \delta(X_i), Z_{i,t} \rangle + \varepsilon_{i,t}^{FD}. \label{eq:SA-reg-ml-fd-fe}
\end{equation*}
Note that the average of the first-differenced outcome across all never-treated individuals at time $t$ belonging to the quasi-treatment cohort $Q_i = g$ is $\gamma_t - \gamma_{g - 1}$. Hence, I can define the demeaned outcome and residual as
\begin{align*}
    \widetilde{Y}_{i, t}^{FD} &= Y_{i,t}^{FD} -  \frac{1}{|j \in Q_i, G_j = \infty|} \sum_{j \in Q_i, G_j = \infty} Y_{j,t}^{FD} \\
    \widetilde{\varepsilon}_{i, t}^{FD}&=  \varepsilon_{i,t}^{FD} - \frac{1}{|j \in Q_i, G_j = \infty|} \sum_{j \in Q_i, G_j = \infty} \varepsilon_{j,t}^{FD}
\end{align*}
and rewrite the regression as  
\begin{equation*}
\widetilde{Y}_{i,t}^{FD} = \langle \delta(X_i) , Z_{i,t} \rangle + \widetilde{\varepsilon}_{i,t}^{FD} .
\end{equation*}
The framework in Section~\ref{sec:estimation_framework} can then be used to estimate $\delta(x)$, except that the baseline heterogeneity functions are omitted. 

\section{Proof of conditional CATT identification} \label{sec:ccatt_identification}
From \eqref{eq:SA-reg-ml-fd}, taking expectations conditional on cohort membership $G_i$ and covariates $X_i = x$  gives
\begin{align*}
\E[Y_{i,g+k}^{FD} \mid  Q_i = g, G_i=g, X_i = x] &= f_{g+k}(x) - f_{g-1}(x) + \delta_{g, k}(x) \\
\E[Y_{i,g+k}^{FD} \mid Q_i = g, G_i=\infty, X_i = x] &= f_{g+k}(x) - f_{g-1}(x).
\end{align*}
Then, 
\begin{align*}
& \E[Y_{i,g+k}^{FD} \mid  Q_i = g, G_i=g, X_i = x]  - \E[Y_{i,g+k}^{FD} \mid Q_i = g, G_i=\infty, X_i = x] \\
&= \delta_{g, k}(x) 
\end{align*}
Finally, 
\begin{align*}
   \delta_{g, k}(x) &= \E[Y_{i,g+k}^{FD} \mid  Q_i = g, G_i=g, X_i = x]  - \E[Y_{i,g+k}^{FD} \mid Q_i = g, G_i=\infty, X_i = x]\\
   &= \E[Y_{i,g+k}^{FD} \mid  Q_i = g, G_i=g, X_i = x]  - \E[Y_{i,g+k}^{FD, \infty} \mid Q_i = g, G_i=\infty, X_i = x]\\
   &\text{ by the definition of being untreated} \\
   &= \E[Y_{i,g+k}^{FD} \mid  Q_i = g, G_i=g, X_i = x]  - \E[Y_{i,g+k}^{FD, \infty} \mid Q_i = g, G_i=g, X_i = x]\\
   &\text{ by the parallel trends assumption in \assref{ass:parallel_trends_cond}} \\
   &= \E[Y_{i,g+k}^{FD} \mid   G_i=g, X_i = x]  - \E[Y_{i,g+k}^{FD, \infty} \mid  G_i=g, X_i = x]\\
   &= \E[Y_{i,g+k}^{FD} - Y_{i,g+k}^{FD, \infty} \mid  G_i=g, X_i = x]\\
   &= \E[Y_{i,g+k} - Y_{i,g-1} - (Y_{i,g+k}^{\infty} - Y_{i,g-1}^{\infty}) \mid  G_i=g, X_i = x]\\
   &= \E[Y_{i,g+k} - Y_{i,g-1}^\infty - (Y_{i,g+k}^{\infty} - Y_{i,g-1}^{\infty}) \mid  G_i=g, X_i = x]\\
   &\text{ by the no anticipation assumption in \assref{ass:no_ant_cond}}\\
   &= \E[Y_{i,g+k} - Y_{i,g+k}^{\infty} \mid  G_i=g, X_i = x]\\
   &= CATT_{g, k}(x)
\end{align*}

\section{Propensity scores in panel data settings}\label{sec:treatment_propensity}
Estimating the treatment propensity $E[Z_{i,t} \mid X_i]$ is a nontrivial task because of the panel data setting and staggered treatment adoption. 
Recall $Z_{i,t}^{g,k}=\mathbf{1}\{G_i=g\}\mathbf{1}\{t=g+k\}$. 
Now, for some $(g, k)$ any for observation $(i, t)$, conditional on covariates $X_i$, 
$$
\E[Z_{i,t}^{g,k}\mid X_i]
= \E\big[\mathbf{1}\{G_i=g\}\mathbf{1}\{t=g+k\}\mid X_i=x\big]
= \Pr(G_i=g, t=g+k \mid X_i).
$$
By the chain rule for probabilities,
$$
\Pr(G_i=g, t=g+k \mid X_i)
=\Pr(G_i=g\mid X_i)\cdot \Pr(t=g+k \mid G_i=g, X_i).
$$
$\Pr(G_i=g\mid X_i)$, is the probability of treatment time conditional on covariates. 
For $\Pr(t=g+k \mid G_i=g, X_i)$, notice that conditional on $g$ and $k$, for any observation $(i, t)$, we immediately know whether $t = g+k$. 
Hence, 
$$\Pr(G_i=g\mid X_i)\cdot \mathbf{1}\{t=g+k\}.$$
There are many known procedures for estimating $\Pr(G_i=g\mid X_i)$, including using a logit or probability forest. 

The estimation of $\E[T_{i,t}\mid X_i=x]$ follows a similar logic. 
Note that the $s$-th row of $T_{i,t}$ can be redefined as
\begin{equation*}
    T_{i,t}^s = \mathbf{1}\{s=t\} + \mathbf{1}\{s = Q_i - 1\}.
\end{equation*}
The two indicators will never both be $1$ because all observations at event time $-1$ are omitted. 
Hence, 
\begin{align*}
\E[T_{i,t}^s \mid X_i] 
&= \E[\mathbf{1}\{s = t\} \mid X_i] + \E[\mathbf{1}\{s = Q_i - 1\} \mid X_i] \\[4pt]
&= \E[\mathbf{1}\{t = s\} \mid X_i] + \E[\mathbf{1}\{Q_i = s + 1\} \mid X_i] \\[4pt]
&= \mathbf{1}\{t = s\} + \Pr(Q_i = s + 1 \mid X_i).
\end{align*}
since the time period $t$ and row $s$ are known by definition. 

$\Pr(Q_i = s + 1 \mid X_i)$ is also already known. 
I can integrate over the conditional empirical distribution of treatment dates conditional on an observation's first appearance in the data to obtain the conditional distribution of quasi-treatment dates. 

\section{Calculating Doubly Robust Scores} \label{sec:doubly_robust}
Since all my treatment variables are indicators and for all treated units $i$, $\sum_{g, k}Z_{i, t} = 1$, I can reframe staggered treatment adoption treatment as a single multivalued treatment, where each treatment arm corresponds to a treatment cohort-event time. 
I can then estimate doubly robust treatment effects using existing work on calculating doubly robust scores when treatments are multi-valued (\citealt{uysal_dr_2015}).

Define the average baseline outcome as
$$\widehat Y_{i,t}^{\mathrm{baseline}} = m^{(-i)}(X_i) - \sum_{g,k} \widehat P(G_i = g \mid X_i) \mathbf{1}\{t=g+k\} \widehat\delta_{g,k}(X_i).$$
The predicted baseline outcome, accounting for which units are treated and untreated is hence
\[
\widehat\mu_{i,t}^{g, k} =
\begin{cases}
\widehat Y_{i,t}^{FD,\mathrm{baseline}} + \widehat\delta_{g,k}(X_i), & \text{if } G_i \neq \infty, \\[6pt]
\widehat Y_{i,t}^{FD,\mathrm{baseline}}, & \text{if } G_i = \infty.
\end{cases}
\]
Define the outcome residual as 
$$\widetilde{Y}_{i,t}^{g,k} = Y_{i,t}^{FD} - \widehat\mu_{i,t}^{g, k}.$$
For a given column $(g, k)$ of observation $(i, t)$, the inverse propensity estimate score is
$$\mathrm{IPW}_{i,t}^{g, k} = \dfrac{\mathbf{1}\{G_i=g\} \mathbf{1}\{t=g+k\}}{ P(G_i = g \mid X_i) }- \dfrac{\mathbf{1}\{G_i=\infty\}}{ P(G_i = \infty \mid X_i) }. $$
Hence, the doubly robust score for $\hat{\delta}_{g, k}(X_i)$ is 
\begin{align*}
\widehat\varphi_{i,t}^{g, k} &= \widehat \delta_{g,k}(X_i) + \mathrm{IPW}_{i,g} \cdot \widetilde{Y}_{i,t}^{g,k} \\ 
&= \widehat \delta_{g,k}(X_i)
  + \left( \dfrac{\mathbf{1}\{G_i=g\} \mathbf{1}\{t=g+k\}}{ P(G_i = g \mid X_i) }
  - \dfrac{\mathbf{1}\{G_i=\infty\}}{ P(G_i = \infty \mid X_i) } \right) \cdot \left( Y_{i,t}^{FD}  - \widehat\mu_{i,t}^{\,G_i,k_i} \right) .
\end{align*}



\section{Notes on Implementation}
The procedure allows for consistent and unbiased estimation of CATT(x), following \citealt{athey_generalized_2019}. 
One can use the conditionally linear model forest (\texttt{lm\_forest}) from the R package \texttt{grf} and should cluster by observations when using panel data. 
Alternatively, one can use the multi-arm causal forest, which is the multivariate extension of the R-learner from \citealt{nie_quasi-oracle_2021} if they partial out the baseline heterogeneity functions from \eqref{eq:SA-reg-ml-fd}.\footnote{Note that \citealt{nie_quasi-oracle_2021} solves a global minimization function to estimate $\delta(x)$, as opposed to estimating a local parametric model of $\delta(x)$.}
The literature on event study and difference-in-differences also  provides weights that can be used to consistently aggregate $CATT(x)$ into more aggregate quantities of interest (\citealt{callaway_difference--differences_2021}, \citealt{sun_estimating_2021}).
By default, for each observation, \texttt{lm\_forest} will produce $CATT(x)$ values for many treatment cohorts. 
Only the treatment cohort-event time estimate pertaining to that $(i, t)$ should be used when estimating doubly robust scores. 

\end{document}