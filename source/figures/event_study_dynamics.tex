\begin{figure}[t]
    \caption{Average Organizational Resilience Using the Forest Split and Continuous Practice Scores}
    \label{fig:event_study_forest_dynamics}
    \centering
    \begin{minipage}[b]{0.325\textwidth}
        \centering
        \subcaption{Pull requests opened} \label{fig:event_study_prs_opened_forest_dynamics}
        \includegraphics[width=\textwidth]{output/analysis/event_study_personalization/important_topk_exact1/rolling5_pc/nevertreated/pull_request_opened_norm_split_pull_request_merged_lm_forest_nonlinear_observed_att_dr_kcombo_pc.png}
    \end{minipage}
    \hfill
    \begin{minipage}[b]{0.325\textwidth}
        \centering
        \subcaption{Pull requests merged} \label{fig:event_study_prs_merged_forest_dynamics}
        \includegraphics[width=\textwidth]{output/analysis/event_study_personalization/important_topk_exact1/rolling5_pc/nevertreated/pull_request_opened_norm_split_pull_request_merged_lm_forest_nonlinear_observed_att_dr_kcombo_pc.png}
    \end{minipage}
    \hfill
    \begin{minipage}[b]{0.325\textwidth}
        \centering
        \subcaption{New software releases} \label{fig:event_study_releases_forest_dynamics}
        \includegraphics[width=\textwidth]{output/analysis/event_study_personalization/important_topk_exact1/rolling5_pc/nevertreated/overall_new_release_count_norm_split_pull_request_merged_lm_forest_nonlinear_observed_att_dr_kcombo_pc.png}
    \end{minipage}
  \begin{minipage}{1\textwidth}
    \textbf{Figure notes:}  
    Panel~\subref{fig:event_study_prs_opened_forest_dynamics} presents event study estimates of the impact of key member departures on the number of pull requests opened for organizations in the early high, early low, late high and low low resilience subset.
    Organizations are assigned to the early (late) high resilience subset when their estimated doubly robust first (fifth) post-period treatment effect--computed using continuous organizational practice scores--exceeds the cross-organizational median.
    Organizations with below-median effects belong to the early (late) low resilience subset. 
    Outcomes are standardized using each organization’s mean and standard deviation over the five pre-periods, and each period spans six months. 
    Estimates, expressed in standard-deviation units, are obtained using \cite{sun_estimating_2021}. 
    Confidence intervals are 95\% intervals based on asymptotic standard errors clustered at the organization level. 
    The two reported pre-trend p-values are Wald tests of the null that all pre-period coefficients equal zero for the high- and low-collaboration subsets.
    The final p-value is from a Wald test of the null that post-period event-study coefficients do not differ between the two subsets.
    Panels~\subref{fig:event_study_prs_merged_forest_dynamics} and~\subref{fig:event_study_releases_forest_dynamics} repeat this analysis using the number of pull requests merged and the number of new software releases as outcomes.
    \end{minipage}
  
  

\end{figure}